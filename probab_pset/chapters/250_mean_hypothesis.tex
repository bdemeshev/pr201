% !Mode:: "TeX:UTF-8"
\section{Гипотезы о среднем и сравнении среднего при большом $n$}


\problem{Предположим, что исходные наблюдения $X_{i}$ нормальны $N(\mu,\sigma^{2})$ и независимы. Константы $\mu$ и $\sigma$ неизвестны. Вовочка строит доверительный интервал для $\mu$ по первой половине доступных наблюдений. Петя --- по всем наблюдениям. Может ли получится у Вовочки интервал меньшей ширины, чем у Пети?}
\solution{Да. Например, если первая половина наблюдений попала рядом с $\mu$, а вторая --- далеко. Ну не повезло Пете.}

% untyp
\problem{Вася и Петя выясняют, кто лучше умеет знакомиться с девушками. Вася попытался познакомиться с 100 девушек, из них 54 девушки дали ему номер своего телефона. Петя  попытался познакомиться с 900 девушек, из них 495 дали ему номер своего телефона. 

Вася и Петя изучили курс матстата и начали спорить. Петя утверждает, что ему в среднем чаще девушки дают свой номер телефона и аргументирует это так: давай проверим гипотезу, что в среднем ровно половина девушек даёт номер своего телефона, против альтернативной гипотезы, что больше половины. По твоим данным эта гипотеза не отвергается, а по моим --- отвергается.

Вася утверждает, что он лучше убеждает девушек. Аргументирует это так: давай проверим гипотезу, что в среднем 60\% девушек даёт номер своего телефона, против альтернативной гипотезы, что меньше 60\%. По твоим данным эта гипотеза отвергается, а по моим --- не отвергается. 

Кто из них прав?}
\solution{Оба они делают одну ошибку: если $H_0$ не отвергается, это не значит, что она --- верна. Корректнее было бы провести тест на сравнение средних. }
Идея: Кирилл Фурманов


\problem{ Расчёт вероятностей с неизвестным параметром
\begin{enumerate}
\item Величина $X$ --- равномерна на $[\mu-1;\mu+1]$, причем $\mu$ --- неизвестна. Найдите $\E(X)$. С какой вероятностью случайный интервал $[X-0.5;X+0.5]$ накрывает неизвестное $\mu$?
\item Пусть $X_{1}$, $X_{2}$ - нормальны $N(\mu,1)$, причем $\mu$ --- неизвестна. Найдите $E(X_{1})$, $E(\bar{X})$. С какой вероятностью случайный интервал $[X_{1}-0.5;X+0.5]$ накрывает неизвестное $\mu$? А интервал $[\bar{X}-0.5;\bar{X}+0.5]$? Почему вторая вероятность не равна первой? 
\end{enumerate} 
}
\solution{ b) У $\bar{X}$ меньше дисперсия, поэтому она в среднем ближе к $\mu$, чем $X_{1}$.}

\problem{
До проведения рекламной компании в среднем 7 из 10
посетителей художественного магазина-салона делали покупки. После
рекламной компании из 200 посетителей покупки сделали 163. Можно
ли считать, что рекламная компания имела эффект на 5\%-ом уровне
значимости? 
} 
\solution{} 

\problem{
Дневные расходы электроэнергии на предприятии составляли
1400 КВт со стандартным отклонением 50 КВт. За 50 дней прошедших
после ремонта и наладки оборудования средние расходы за день
составили 1340 КВт. Можно ли считать, что ремонт способствовал
экономии электроэнергии на 10\% уровне значимости? 
} 
\solution{} 


\problem{
Монету подбросили 1000 раз, при этом 519 раз она выпала на орла.
Проверьте гипотезу о том, что монета <<правильная>> на уровне
значимости 5\%.
} 
\solution{} 

\problem{
Вася отвечает на 100 тестовых вопросов. В каждом вопросе
один правильный вариант ответа из пяти возможных. На 5\%-ом уровне
значимости проверьте гипотезу о том, что Вася ставит ответы
наугад, если он ответил правильно на 26 вопросов из теста.\par
} 
\solution{} 

\problem{
Некоторых студентов спросили, на какую оценку они рассчитывает по
теории вероятностей, 30 человек надеются на 4 балла, 20 человек ---
на 6 баллов, 30 человек --- на 8 баллов, 10 человек --- на 10 баллов.
Проверьте гипотезу о том, что медиана равна 7 баллам на уровне
значимости 10\%.
} 
\solution{} 

\problem{
Кубик подбросили 160 раз, из них 29 он выпал на шестерку.
Проверьте гипотезу о том, что вероятность выпадения шестерки
правильная на уровне значимости 10\%.
} 
\solution{} 

\problem{
Двести домохозяек попробовали новый <<{\it Вовсе не обычный
порошок}>>, 110 из них получили более удачный результат, чем
раньше. На уровне значимости 5\% проверьте гипотезу о том, что
<<{\it Вовсе не обычный порошок}>> по эффективности не отличается
от старого средства (против альтернативной гипотезы о большей
эффективности).
} 
\solution{} 

\todo{Может включить неизвестную дисперсию?}
\problem{
Величины $x_{1}$, $x_{2}$, ..., $x_{n}$ независимы и распределены $N(10,16)$. Вася знает дисперсию, но не знает среднего. Поэтому он строит 60\% доверительный интервал для истинного среднего значения. У него получаются две границы --- левая и правая. \par
Какова вероятность того, что:
\begin{enumerate}
\item левая меньше 9?
\item левая и правая лежат по разные стороны от 9? 
\item левая и правая лежат по разные стороны от настоящего среднего? 
\end{enumerate}
}
\solution{} 


\problem{
По предварительному опросу 10000 человек на выборах в Думу 462
человека будут голосовать за партию <<{\it Обычная партия}>>. На
уровне значимости 0,05 проверьте гипотезу о том, что <<{\it
Обычная партия}>> преодолеет 5\% барьер.
} 
\solution{} 

\problem{
Вася и Петя метают дротики по мишени. Каждый из них сделал
по 100 попыток. Вася оказался метче Пети в 59 попытках. На уровне
значимости 5\% проверьте гипотезу о том, что меткость Васи и Пети
одинаковая, против альтернативной гипотезы о том, что Вася метче
Пети. 
} 
\solution{} 

\problem{
По 820 посетителям супермаркета средние расходы на одного
человека составили 340 рублей. Из достоверных источников известно,
что дисперсия равна 90000 руб. Постройте $95\%$ доверительные
интервалы для средних расходов одного посетителя,
двусторонний и два односторонних. 
} 
\solution{} 

\problem{
В прошлом году средняя длина ушей зайцев в темно-синем
лесу была 20 см, $\sigma=4$. В этом году у случайно попавшихся 15
зайцев средняя длина оказалась 24 см. Предполагая нормальность
распределения, проверьте гипотезу о том, что средняя длина ушей не
изменилась против альтернативной гипотезы о росте длины. 
} 
\solution{} 

\problem{
Стандартное отклонение количества иголок у ежа равно 130.
По выборке из 12 ежей было получено среднее количество иголок
5120. Допустим, что количество иголок на одном еже можно считать
нормально распределенным. 
\begin{enumerate}
\item Постройте $90\%$-ый доверительный интервал для среднего
количества иголок.  
\item На $5\%$-ом уровне значимости проверьте гипотезу о том, что
среднее количество иголок равно 5000.
\end{enumerate}
} 
\solution{} 

\problem{
Средний бал по диплому студента --- нормальная случайная величина, $N(\mu;0.04)$.
Средний бал, рассчитанный по выборке из 25 абитуриентов этого
года, составил $4.30$. По данной выборке был построен
доверительный интервал для $\mu$: $(4.2424; 4.3576)$. Какой
уровень доверия соответствует этому
интервалу? 
} 
\solution{} 

\problem{
Вася очень любит играть в преферанс. Предположим, что Васин
выигрыш распределен нормально. За последние 5 партий средний
выигрыш составил 1560 рублей, при оценке стандартного отклонения
равной 670 рублям. Постройте 90\%-ый доверительный интервал для
математического ожидания Васиного выигрыша. 
} 
\solution{} 

\problem{
In 1882 Michelson performed experiments to measure the speed of
light. 23 trials gave an average of 299756.2 km/sec with a
standard deviation of 107.12. Find a $95\%$ confidence interval
for the speed of light. The correct answer is 299710.5 so there
must have be some bias in his experiments. \par
} 
\solution{} 

\problem{
An English biologist named Weldon was interested in the
'pip effect' in dice --- the idea that the spots, or 'pips', which
on some dice are produced by cutting small holes in the surface,
make the sides with more spots lighter and more likely to turn up.
Weldon threw 12 dice 26306 times for a total of 315672 throws and
observed that a 5 or 6 came up on 106602 throws. Find a $95\%$
confidence interval for the true probability of getting 5
or 6 on a dice. 
} 
\solution{} 

\problem{
On 384 out of 600 randomly selected farms, the operator was
also the owner. Find a $95\%$ confidence interval for the true
proportion of owner operated farms. 
} 
\solution{} 

\problem{
During a two week period, 10 weekdays, a parking garage
collected an average of \$126 with a standard deviation of \$15.
Find a 95\% confidence interval for the mean revenue. \par
Problems are borrowed from \url{www.math.cornell.edu/~durrett/ep4a/ep4a.html} } \solution{} 

\problem{
In their last 100 chess games played against each other, Bill has
won 46 and Monica has won 54. Using this information and a 95\%
confidence level, what is the probability that Bill will win a
<<best of seven>> series with Monica? The first one to win 4 games
is the winner and no more games are played. \par

Hints: First determine a 95\% confidence interval for the
probability that Bill will win a game. Then, using the two
extremes of this interval, determine the probability that Bill
will win the series. This is a binomial experiment. Bill could win
the series in 4 games, 5 games, 6 games, or 7 games. Calculate the
probability of each and add them up. Do this for each of the two
interval extremes. \par
Source: (?):\par
\url{http://www.artofproblemsolving.com/Forum/viewtopic.php?highlight=probability+game\&t=87203}
} 
\solution{} 

\problem{
Имеются две монетки. Одна правильная, другая --- выпадает орлом с
вероятностью $0<q<0.5$, значение $q$ известно. Монетки неотличимы
по внешним признакам. Одну из них, неизвестно какую, подкинули $n$
раз и сообщили Вам, сколько раз выпал орел. 
\begin{enumerate}
\item Опишите процедуру тестирования гипотезы $H_{0}$: <<подбрасывалась правильная
монетка>> против $H_{a}$: <<подбрасывалась неправильная монетка>>. 
\item Каким должно быть $n$ чтобы вероятность ошибок первого и
второго рода не превышала 10 процентов, если $q=0.4$ 
\item Ответьте на предыдущий вопрос при произвольном $q$ \par
\end{enumerate}
} 
\solution{} 


\problem{
Школьник Вася аккуратно замерял время, которое ему требовалось, чтобы добраться от школы до дома. По результатам 90 наблюдений, среднее выборочное оказалось равным 14 мин, а несмещенная оценка дисперсии --- 5 мин$^{2}$. 
\begin{enumerate}
\item  Постройте 90\% доверительный интервал для среднего времени на дорогу \par
\item  На уровне значимости 10\% проверьте гипотезу о том, что среднее время равно 14,5 мин, против альтернативной гипотезы о меньшем времени. Найдите точное $P$-значение.
\end{enumerate} }
\solution{ a) $[13.61;14.39]$ \par
b) Отвергается ($Z_{observed}=-2.12$, $Z_{critical}=-1.28$) \par
c) $P_{value}=0.017$ 
} 



\problem{
На днях Левада-Центр опубликовал итоги опроса, согласно которым 2/3 россиян поддерживают Путина и 2/5 россиян доверяют опросам Левада-Центра. Доверяющие опросам всегда отвечают искренне, а недоверяющие могли соврать в ответе на любой вопрос или оба. Исходя из этих данных, оцените реальную поддержку Путина россиянами. Постройте 95\% доверительный интервал (для поддерживающих Путина и для верящих в опрос), если было опрошено 1000 человек. \par
source: лента ru-math } 
\solution{} 

