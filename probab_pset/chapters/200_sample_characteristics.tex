% !Mode:: "TeX:UTF-8"

\section{Выборочные характеристики, общая интуиция}


\problem{ Имеется пять чисел: $x$, $4$, $5$, $7$, $9$. При каком значении $x$ медиана будет равна среднему? }
\solution{}

\problem{ Измерен рост 100 человек. Средний рост оказался равным 160
см. Медиана оказалась равной 155 см. Машин рост в 163 см был
ошибочно внесен как 173 см. Как изменятся медиана и среднее после
исправления ошибки? } 
\solution{} 

\problem{ Возможно ли, что риск катастрофы в расчете на 1 час пути
больше для самолета, чем для автомобиля, а в расчете на 1 километр
пути --- наоборот? } 
\solution{} 

\problem{ Деканат утверждает, что если студента N перевести из группы
А в группу В, то средний рейтинг каждой группы возрастет. Возможно
ли это? } 
\solution{} 


\problem{ Есть три группы по 10 человек, две группы по 20 человек и
одна группа по 40 человек. У каждой из групп свой преподаватель.
\begin{enumerate}
\item Каков средний размер группы, для которой читает лекции наугад
выбранный профессор? 
\item Каков средний размер
группы, в которой учится наугад выбранный студент?
\item Творческий вопрос. Мы ловим студентов наугад и спрашиваем каждого размер группы, в которой он учится. Можно ли как-то восстановить средний размер группы с точки зрения преподавателя? 
\end{enumerate} }
\solution{} 


\problem{Приведите примеры, когда $\Med(X+Y)=\Med(X)+\Med(Y)$ и $\Med(X+Y)\neq \Med(X)+\Med(Y)$, где $\Med$ --- медиана}
\solution{ два независимых симметричных распределения; практически любая сумма несимметричных распределений, например, два независимых с $p(x)=2-2x$ на $[0;1]$.}

\problem{ Пусть $X$ и $Y$ --- две независимые случайные величины. Верно ли, что $Med(X+Y)=Med(X)+Med(Y)$, где $Med$ - это медиана.}
\solution{нет}

\problem{Исследователь Вениамин измерил рост пяти случайно выбранных человек. Какова вероятность того, что истинная медиана роста лежит между минимумом и максимумом из этих пяти наблюдений? Предположим, что рост имеет непрерывное распределение.
}

\solution{Исключим те варианты, когда все пять наблюдений оказались или синхронно выше, или синхронно ниже медианы, получаем, $p=1-2\cdot 0.5^5=1-0.5^4$.}


\problem{Во время Второй Мировой войны американские военные собрали статистику попаданий пуль в фюзеляж самолёта.  По самолётам, вернувшимся из полёта на базу, была составлена карта повреждений среднестатистического самолёта. С этими данными военные обратились к статистику Абрахаму Вальду с вопросом, в каких местах следует увеличить броню самолёта.

Что посоветовал Абрахам Вальд и почему?}

\solution{Броню следует увеличить в тех местах, где меньше всего следов от пуль. Отсутствие следов пуль в некотором месте означает, что попадание в это место приводит к тому, что самолёт не возвращается на базу.}

\problem{Два лекарства испытывали на мужчинах и женщинах. Каждый
человек принимал только одно лекарство. Общий процент людей,
почувствовавших улучшение, больше среди принимавших лекарство А.
Процент мужчин, почувствовавших улучшение, больше среди мужчин, принимавших лекарство В. Процент женщин, почувствовавших улучшение, больше среди женщин, принимавших лекарство В. 

Возможно ли это? }
\solution{Да, Парадокс Симпсона}
