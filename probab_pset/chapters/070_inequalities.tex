% !Mode:: "TeX:UTF-8"
\section{Неравенства, связанные с ожиданием}

\subsection{Чебышёв/Марков/Кантелли/Чернов}
\problem{
У последовательности неотрицательных случайных величин $X_1$, $X_2$,\ldots дисперсия постоянна, а  математическое ожидание стремится к бесконечности, $\lim \E(X)=+\infty$. Найдите $\lim \P(X_n>a)$. }
\solution{ $\P(X_n\leq q)=\P(X_n-\E(X_n)\leq a-\E(X_n))$. При больших $n$ величина $a-\E(X_n)$ отрицательна, поэтому $\P(X_n-\E(X_n)\leq a-\E(X_n))\leq \P(|X_n-\E(X_n)|\geq |a-\E(X_n)|)\leq c/(a-\E(X_n))^2$.}


\subsection{Йенсен}


\problem{Вася забрасывает удочку $100$ раз.
Вероятность поймать рыбку при одном забрасывании равна $p$. Петя забрасывает удочку случайное количество раз, $N$, под настроение. Известно, что $\E(N)=100$. У кого шансы поймать хотя бы одну рыбку выше?  }
\solution{ У Васи }

\problem{ Тысяча зайцев требует спасения. Дед Мазай выбирает между двумя стратегиями: 
\begin{itemize}
\item[A.] Перевозить зайцев равными партиями по 10 за заход.  
\item[B.] Перевозить зайцев случайными партиями от 1 до 19 зайцев за заход.
\end{itemize}
В каком случае ожидаемое количество заходов будет меньшим? }
\solution{ }



\problem{ При каких условиях $\E\left(\frac{1}{X}\right)=\frac{1}{\E(X)}$? }
\solution{Только если $X$ с вероятностью 1 константа. }


\subsection{Коши"--~Шварц}


\section{Сходимости}

\problem{Определите, к какому распределению сходятся указанные последовательности
\begin{enumerate}
\item $X_n \sim N(\frac{n-1}{n+1},9)$
\item $X_n \sim N(7,\frac{5+n}{n^2})$
\item $X_n \sim t_n$
\item $X_n \sim \frac{Y_n}{n}$, где $Y_n \sim \chi_n^2$
\item $X_n \sim \frac{Y_n}{n+5}$, где $Y_n \sim \chi_n^2$
\item $X_n \sim \frac{Y_n-n}{\sqrt{n}}$, где $Y_n \sim \chi_n^2$
\item $X_n \sim 2011 Y_n$, где $F_{2011,n}$
\end{enumerate}
}
\solution{}