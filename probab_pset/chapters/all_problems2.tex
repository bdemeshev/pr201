% !Mode:: "TeX:UTF-8"

% есть идея проще:
% возле неоттипографленной (новой) задачи я ставлю такой комментарий

% untyp

% когда задача "готова" комментарий "untyp" рядом с ней можно убрать
% так проще потому, что при нажатии ctrl-f "untyp" мы сразу попадаем к нужному месту

% чтобы облегчить совместное редактирование - файл разделен на несколько более мелких частей
% кстати, у texmaker есть сквозной поиск по файлам. т.е. я могу искать "untyp" сразу во всех файлах



% прочее...


%про две шкатулки - вставить в стохан:
%E(Y|X=x_{i}) существует,
%но E(Y|X) - нет

%раздракониванию задач (полный дубляж условия) но вопросы разные - в разные разделы (да!)
%автоматические ссылки туда-сюда ???
%викифицирование ?

%метки
% die - про кубик
% coin - про монетку

%binomial
%uniform
%geom_d
%poisson

%circle_trick
%wrong_class - возможно неправильно классифицирована

%gen_fun - производящие функции


%
%Идеи решения (?):
% Превратить одношаговый эксперимент в двухшаговый: сначала выбрать предметы, затем выбрать их порядок

% !Mode:: "TeX:UTF-8"
\section{Простые эксперименты}
% simple_experiments

\subsection{Дискретные простые эксперименты}
%Эксперимент состоит из одного "этапа"

%Правило сложения вероятностей.
%Если события несовместны, то
%P(хотя бы одно)= сумма
%Р(все сразу)=0

%1.1. дискретные случайные величины (P, E)

\problem{
Подбрасываются два кубика. Какова вероятность выпадения хотя бы
одной шестёрки? Какова вероятность того, что шестёрка не выпадет
ни разу? }
\solution{
$\PP(N\geq 1)=1-\frac{5}{6}^2$; $\PP(N=0)=\frac{5}{6}^2$. }
\cat{die}

\problem{
$\Omega =\{a, b, c\}$, $\PP\ofbr{a, b}=0{,}8$, $\PP\ofbr{b,c}=0{,}7$. Найдите
$\PP\ofbr{a}$, $\PP\ofbr{b}$, $\PP\ofbr{c}$.}
\solution{$\PP\ofbr{b}=0{,}2$, $\PP\ofbr a=0{,}6$, $\PP\ofbr{c}=0{,}5$.  }

\problem{
$A$  и  $B$  несовместны,  $\PP(A)=0{,}3$, $\PP(B)=0{,}4$. Найдите
$\PP(A^{c} \cap B^{c} )$.}
\solution{ $\PP(A^{c} \cap B^{c} )=1-0{,}3-0{,}4$.}

\problem{
$\PP(A)=0{,}3$,  $\PP(B)=0{,}8$. В каких пределах
может лежать  $\PP(A\cap B)$? }
\solution{ $\PP(A\cap B)\in[0{,}1;0{,}3]$.}


\problem{
Кубик подбрасывается два раза. Какова вероятность того, что результат
второго броска будет строго больше, чем результат первого?
Какова вероятность того, что в сумме будет 6? Что в сумме будет 9? Что максимум равен
5? Что минимум равен 3? Что разница будет равна 1 или 0? }
\solution{ $\PP\ofbr{N_{2}>N_{1}}=\frac{15}{36}$; \par
$\PP\ofbr{N_{1}+N_{2}=6}=\frac{5}{36}$; \par
$\PP\ofbr{N_{1}+N_{2}=9}=\frac{4}{36}$; \par
$\PP\ofbr{\max\{N_{1},N_{2}\}=5}=\frac{9}{36}$; \par
$\PP\ofbr{\min\{N_{1},N_{2}\}=3}=\frac{7}{36}$; \par
$\PP\ofbr{|N_{1}-N_{2}|\leq 1}=\frac{16}{36}$. }
\cat{die}

\problem{ \label{shokoladnie konfeti}
На подносе лежит 20 шоколадных конфет, одинаковых с виду. В
четырёх из них есть орех внутри. Маша съела 5 конфет. Какова
вероятность того, что в наугад выбранной оставшейся конфете будет
орех? }
\solution{$\frac{4}{20}$. }

%%%%% пошло применение ожидания

\problem{ \label{ojidanie ot bernulli}
Пусть  $X$  принимает два значения, причём $\PP\ofbr{X=1}=p$ и
$\PP\ofbr{X=0}=1-p$. Найдите $\E(X)$.}
\solution{ $\E(X)=p$.}



\problem{
Пусть существует всего два момента времени, $t = 0$ и $t =
1$. Cтоимости облигаций и акций в момент времени $t$ обозначим соответственно
$B_{t}$ (bond) и $S_{t}$ (share). Известно, что $B_{0}=1$,
$B_{1}=1{,}1$, $S_{0}=5$, $S_{1}=
\begin{cases}
10, & p_{\text{high}}=0{,}7; \\
2, & p_{\text{low}}=0{,}3.
\end{cases}$ \\
Индивид может покупать акции и облигации по указанным ценам без
ограничений. Например, можно купить минус одну акцию: это
означает, что в момент времени $t=0$ индивид получает 5 рублей, а в момент $t = 1$ в
зависимости от состояния природы должен заплатить 10 рублей или 2
рубля.
\begin{enumerate}
\item Чему равна безрисковая процентная ставка за период?
\item Найдите дисконтированные математические ожидания будущих цен
акций и облигаций. Совпадают ли они с ценами нулевого периода?
\item Найдите такие вероятности $q_{\text{high}}$ и $q_{\text{low}}$, чтобы
дисконтированное математическое ожидание будущих цен
совпало с ценами нулевого периода.
\item Индивиду предлагают купить некий актив, который приносит 8
рублей в состоянии мира $\omega_{\text{high}}$ и 11 рублей в состоянии
мира $\omega_{\text{low}}$. Посчитайте ожидание стоимости этого актива с
помощью вероятностей $p$ и с помощью вероятностей $q$. Придумайте
такую комбинацию акций и облигаций, которая в будущем приносит 8 и
11 рублей соответственно, и найдите её стоимость.\end{enumerate} }
\solution{ }



\problem{
Игральный кубик подбрасывается два раза. Пусть  $X_{1}$ и $X_{2} $
"--- результаты подбрасывания. Найдите вероятности $\PP(\min
\left\{X_{1},X_{2} \right\}=4)$  и $\PP(\min
\left\{X_{1},X_{2} \right\}=2)$. }
\solution{ }


\problem{  \label{simple third}
На десяти карточках написаны числа от 1 до 9. Число 8 фигурирует
два раза, остальные числа "--- по одному разу. Карточки извлекают в
случайном порядке. Какова вероятность того, что девятка появится позже обеих
восьмёрок? }
\solution{ Устно: $\frac{1}{3}$.}

\problem{
17 заключённых, 5 камер. Заключённых распределяют по камерам по очереди, равновероятно в каждую. Какова вероятность, что Петя и Вася сидят в одной камере? }
\solution{ $0{,}2$. }
% решабельна ли более сложная задача, где конфигурации рассадок равновероятны?


\problem{
Кость подбрасывается два раза. Пусть  $X$  и  $Y$  "---
результаты
подбрасываний. Найдите  $\E\parb{|X\hm-Y|}$. }
\solution{ }

\problem{
\foreignlanguage{british}{We throw 3 dices one by one. What is the probability that we obtain 3 points in strictly increasing order?} }
\solution{ $\frac{C_{6}^{3}}{6^{3}}$. }

\problem{ \label{tri chisla}
Из 10 цифр (от 0 до 9) выбираются 3 наугад (возможны повторения).
Обозначим числа (в порядке появления): $X_{1}$, $X_{2}$, $X_{3}$.
Какова вероятность того, что $X_{1}>X_{2}>X_{3}$? }
\solution{ $\frac{C_{10}^{3}}{10^{3}}$, т.\,к. каждый способ выбрать три разных числа соответствует благоприятной
комбинации. }



\problem{
Кубик подбрасывается 3 раза. Какова вероятность того, что сумма первых двух подбрасываний будет больше третьего? }
\solution{ }

\problem{ \zdt{<<Масть>> при игре в бридж }

Часто приходится слышать, что некто при игре в бридж получил на
руки 13 пик. Какова вероятность (при условии, что карты хорошо
перетасованы) получить 13 карт одной масти?
\begin{note}
Каждый из четырёх игроков в бридж получает 13 карт из колоды в 52 карты.
\end{note}
\begin{ist}
Mosteller.
\end{ist}
}
\solution{ }


\problem{ \label{maksimum iz kartochek}
На карточках написаны числа от
1 до 100. В левую руку Маша берёт одну карточку, в правую "--- $k$~карточек.
Какова вероятность того, что число на карточке в левой руке
окажется больше числа на любой карточке из
правой руки? }
\solution{$\frac{1}{k+1}$, т.\,к. одна из $k+1$ карточек должна быть наибольшей.  }



\problem{ \label{sleeping beauty} \zdt{Спящая красавица}

Спящая красавица согласилась принять участие в научном
эксперименте. В воскресенье её специально уколют веретеном. Как
только она заснёт, будет подброшена правильная монетка. Если
монетка выпадет орлом, то спящую красавицу разбудят в понедельник
и спросят о том, как выпала монетка. Если монетка выпадет решкой,
то спящую царевну разбудят в понедельник, спросят о монетке, снова
уколют веретеном, разбудят во вторник и снова спросят о монетке.
Укол веретена вызывает легкую амнезию, и красавица не сможет
определить, просыпается ли она в первый раз или во второй.
Красавица только что проснулась.
\begin{enumerate}
\item Какова вероятность того, что сегодня понедельник?
\item Как следует отвечать красавице, если за каждый верный ответ ей
дарят молодильное яблоко?
\item Как следует отвечать красавице, если за неверный ответ её тут
же превращают в тыкву?
\end{enumerate}
\begin{note}
Осторожно! Некорректные вопросы!
\end{note}
 }
\solution{<<Сегодня понедельник>> "--- это \textbf{не} событие. Вероятность не
определена. Это функция от времени.

Вероятность того, что монетка выпала орлом, равна $0{,}5$. Поэтому ей
всё равно, как отвечать, если наказанием является превращение в
тыкву, и нужно отвечать: <<Решка!>> "--- если наградой является
молодильное яблоко. Предполагается, что красавица максимизирует
ожидаемое количество молодильных яблок.  }



\problem{
Пусть события $A_{0}$, $A_{1}$ и $A_{2}$ несовместны и вместе
покрывают всё $\Omega$. Обозначим $p_{0}=\P(A_{1}\cup A_{2})$, $p_{1}=\P(A_{0}\cup A_{2})$,
$p_{2}=\P(A_{0}\cup A_{1})$. Перечислите все условия, которым удовлетворяют $p_{0}$, $p_{1}$,
$p_{2}$. }
\solution{ }

\problem{
Найдите вероятность того, что произойдёт ровно одно событие из $A$ и $B$, если $\P(A)=0{,}3$, $\P(B)=0{,}2$, $\P(A\cap B)=0{,}1$.    }
\solution{ }

\problem{
Вася наугад выбирает два разных натуральных числа от 1 до 4.
\begin{enumerate}
\item Какова вероятность того, что будет выбрано число 3?
\item Какова вероятность того, что сумма выбранных чисел будет чётная?
\item Каково математическое ожидание суммы выбранных чисел?
\end{enumerate}
 }
\solution{ $\PP\ofbr{3}=\frac{1}{2}$, $\PP\ofbr{\Sigma\text{ чёт.}}=\frac{1}{3}$, $\E(\Sigma)=5$. }


\problem{
Известно, что когда соревнуются А и Б, то А побеждает с вероятностью $x$ (Б, соотвественно, с вероятностью $(1-x)$). Когда соревнуются А и В, то А побеждает с вероятностью $y$ (В, соответственно, с вероятностью $(1-y)$).
\begin{enumerate}
\item  Придумайте модель, которая бы позволяла узнать вероятность победы Б над В.
\item  Покажите, что можно придумать другую модель и получить другую вероятность.
\end{enumerate} }
\solution{
Если предположить, что у каждого игрока есть своя сила (константа), а вероятности победить в схватке для двух игроков относятся так же, как их силы, то $x=\frac{a}{a+b}$, $y=\frac{a}{a+c}$. Легко находим, что $\frac{b}{b+c}=\frac{y-xy}{x+y-2xy}$. }



\problem{  В клубе 25 человек.
\begin{enumerate}
\item  Сколькими способами можно выбрать комитет
из четырёх человек?
\item  Сколькими способами можно выбрать руководство, состоящее из
директора, зама и кассира?
\end{enumerate}
 }
\solution{ Комитет можно выбрать $C_{25}^{4}$ способами, руководство "--- $C_{25}^{3}3!$.}

\problem{ Сколькими способами можно расставить 5 человек в очередь?}
\solution{$5!$. }

\problem{ Сколькими способами можно покрасить 12 комнат, если требуется 4
покрасить жёлтым цветом, 5 "--- голубым и 3 "--- зелёным?}
\solution{ $C_{12}^{4}C_{8}^{5}$. }

\problem{ Шесть студентов (три юноши и три девушки), стоят в очереди за
пирожками в случайном порядке. Какова вероятность того, что юноши
и девушки чередуются?}
\solution{$2\cdot\frac{3!3!}{6!}$. }


\problem{
Где-то в начале 17 века Галилея попросили объяснить следующее:
количество троек натуральных чисел, дающих в сумме 9, такое же,
как количество троек, дающих в сумме 10; но при трёхкратном
подбрасывании кубика 9 в сумме выпадает реже, чем 10. Дайте корректное объяснение. }
\solution{ }


\problem{В классе 30 человек, и все разного роста. Учитель физкультуры хочет отобрать и поставить в порядке возрастания роста 5 человек. Сколькими способами это можно сделать?}
\solution{$C_{30}^{5}$. Расположить по росту можно только в одном порядке. }

\subsection{Непрерывные простые эксперименты}
%1.2. непрерывные случайные величины (P, E для равномерной)
\problem{
Поезда метро идут регулярно с интервалом 3 минуты. Пассажир
приходит на платформу в случайный момент времени. Пусть $X$
"--- время ожидания поезда в минутах.

Найдите $\P(X<1)$, $\E(X)$. }
\solution{$\frac{1}{3}$, $1{,}5$. }


\problem{
Светофор 40 секунд горит зелёным светом, 3 секунды "--- жёлтым, 30
секунд "--- красным, затем цикл повторяется. Петя подъезжает к светофору. На жёлтый свет Петя предпочитает остановиться.
\begin{enumerate}
\item  Какова вероятность, что Петя сможет проехать сразу?
\item  Какова средняя задержка Пети на светофоре?
\item  Вася, стоящий рядом со светофором, смотрит на него в течение 3
секунд. Какова вероятность того, что он увидит смену цвета?
\end{enumerate}
 }
\solution{ }

\problem{
Случайные величины $X$, $Y$, и $Z$ независимы и равномерны на $[0;1]$. Какова вероятность того, что $X+Y>Z$? }
\solution{ }


\problem{
\foreignlanguage{british}{At a bus stop you can take bus \#1 and bus \#2. Bus \#1 passes 10 minutes after bus \#2 has passed whereas bus \#2 passes 20 mins after bus \#1 has passed. What is the average waiting time to get on a bus at that bus stop?}

\begin{ist}
Wilmott forum, \texttt{catid=26\&threadid=55617}.
\end{ist}
 }
\solution{ $\frac{25}{3}$. }


\problem{
На множестве $A:=\{x\geq 0,\ 0\leq y\leq e^{-x}\}$ случайно (равномерно) выбирается точка. Пусть $X$ "--- абсцисса этой точки. Найдите следующие вероятности: $\PP(X>1)$, $\PP(X\in (1;5))$, $\PP(X \in [1;5])$. }
\solution{ $\int_{1}^{\infty}e^{-x}\,dx$; $\PP(X\in (1;5))=\PP(X \in [1;5])=\int_{1}^{5}e^{-x}\,dx$. }


\subsection{Смешанные простые эксперименты, или содержание эксперимента неясно}
%1.3. смешанные случайные величины (P, E для смеси с равномерной)
\problem{
Как связаны между собой $\PP(A)$ и $\E(\inds{A})$? }
\solution{Равны.}

\problem{Случайная величина $X$ равновероятно принимает одно из пяти значений: 1, 2, 3, 8 и 9. 
\begin{enumerate}
\item Найдите математическое ожидание и медиану $X$
\item Найдите значение $u$ при котором функция $f(u)=\E(|X-u|)$ достигает минимума
\item Найдите значение $u$ при котором функция $g(u)=\E((X-u)^2)$ достигает минимума
\item Сделайте выводы
\end{enumerate}
}
\solution{$\min f(u)=\med(X)$, $\min g(u)=\E(X)$}
\input{chapters/020_complex_experiments.tex}
% !Mode:: "TeX:UTF-8"
\section{Условные вероятности и ожидания. Дополнительная информация}
%Правило умножения вероятностей:
%Если A B независимы, то

\subsection{Условная вероятность}

\problem{ \ENGs
A bag contains a counter, known to be either white or black. A white counter is put in, the bag is shaken, and a counter is drawn out, which proves to be white. What is now the chance of drawing a white counter? \RUSs}
\solution{ }

\problem{ \ENGs
You have a hat in which there are three pancakes: one is golden on both sides, one is brown on both sides, and one is golden on one side and brown on the other. You withdraw one pancake, look at one side, and see that it is brown. What is the probability that the other side is brown? \RUSs}
\solution{ }

\problem{ \ENGs
The inhabitants of an island tell truth one third of the time. They lie with the probability of $\frac{2}{3}$. On an occasion, after one of them made a statement, another fellow stepped forward and declared the statement true. What is the probability that it was indeed true? \RUSs }
\solution{ }


\problem{
На кубиках написаны числа от 1 до 100. Кубики свалены в кучу. Вася выбирает наугад из кучи по очереди три кубика.
\begin{enumerate}
\item Какова вероятность, что полученные три числа будут идти в возрастающем порядке?
\item Какова вероятность, что полученные три числа будут идти в возрастающем порядке, если известно, что первое меньше последнего?
\end{enumerate}
 }
\solution{ $\frac{1}{6}$; $\frac{1}{3}$. }


\problem{ (дописать)

Наследование группы крови контролируется аутосомным геном. Три его аллеля обозначаются буквами А, В и 0. Аллели А и В доминантны в одинаковой степени, а аллель 0 рецессивен по отношению к ним обоим. Поэтому существует четыре группы крови. Им соответствуют следующие генотипы:
\begin{itemize}
\item Первая (I) "--- 00;
\item Вторая (II) "--- АА, А0;
\item Третья (III) "--- ВВ, В0;
\item Четвёртая (IV) "--- АВ.
\end{itemize}

Наследование резус-фактора кодируется тремя парами генов и происходит независимо от наследования группы крови. Наиболее значимый ген имеет два аллеля, аллель D доминантный, аллель d рецессивный. Таким образом, получаем следующие генотипы:
\begin{itemize}
\item Резус-положительный "--- DD, Dd;
\item Резус-отрицательный "--- dd.
\end{itemize}

Если у беременной женщины резус"=отрицательная кровь, а у плода резус"=положительная, то есть риск возникновения гемолитической болезни (у матери образуются антитела к резус фактору, безвредные для неё, но вызывающие разрушение эритроцитов плода).

Перед нами два семейства: Монтекки и Капулетти. \\
...}
\solution{ }


\problem{ \label{rekordnaia volna}
Пусть $X_{i}$ "--- НОРСВ, такие, что $\PP(X_{i}=X_{j})=0$. Обозначим за
$E_{k}$ событие, состоящее в том, что $X_{k}$ оказалась
<<рекордом>>, т.\,е. больше, чем все предыдущие $X_{i}$ ($i<k$). Для
определённости будем считать, что $E_{1}=\Omega$.
\begin{enumerate}
\item Найдите $\PP(E_{k})$.
\item  Верно ли, что $E_{k}$ независимы?
\item  Какова вероятность того, что второй рекорд будет зафиксирован в $n$-й момент времени?
\item  Сколько в среднем времени пройдёт до второго рекорда?
\end{enumerate}

\begin{ist}
Williams, 4.3.
\end{ist}
 }
\solution{ Какая-то из первых $k$ величин будет наибольшей. В силу \iid{}
получаем, что $\PP(E_{k})=\frac{1}{k}$. $E_k$ независимы: например, если известно,
что 10-е наблюдение было рекордом, это ничего не говорит о рекордах в первых 9-ти
наблюдениях. Вероятность второго рекорда в $n$-й момент равна $\frac{1}{n(n-1)}$,
а в среднем времени до второго рекорда пройдёт $\infty$. }

\problem{Известно, что $\PP(A \mid B)=\PP(A \mid B^{c})$. Верно ли, что $A$ и $B$ независимы?}
\solution{Да.}


\problem{ \zdt{Randomized response technique}

В анкету для чиновников включён скользкий вопрос: <<Берёте ли Вы
взятки?>>. Чтобы стимулировать чиновников отвечать правдиво,
используется следующий прием. Перед ответом на вопрос чиновник втайне от анкетирующего подкидывает специальную монетку, на гранях
которой написано <<правда>>, <<ложь>>. Если монетка выпадает
<<правдой>>, то предлагается отвечать на вопрос правдиво, если
монетка выпадает на <<ложь>>, то предлагается солгать. Таким
образом, ответ <<да>> не обязательно означает, что чиновник берёт
взятки.

Допустим, что треть чиновников берёт взятки, а монетка
неправильная и выпадает <<правдой>> с вероятностью 0{,}2.
\begin{enumerate}
\item Какова вероятность того, что чиновник ответит <<да>>?
\item  Какова вероятность того, что чиновник берёт взятки, если он
ответил <<да>>? Если ответил <<нет>>?
\end{enumerate}
\todo[inline]{Вставить построение несмещённой оценки?}
}
\solution{ }

\problem{
Пусть события  $A$  и  $B$  независимы и $\PP(B)>0$.
Чему равна  $\PP(A \mid B)$? }
\solution{ $ \PP(A \mid B)=\PP(A)$. }

\problem{
Из колоды в 52 карты извлекается одна карта наугад. Верно ли, что
события <<извлечён туз>> и <<извлечена пика>> являются
независимыми? }
\solution{ Да. }

\problem{
Из колоды в 52 карты извлекаются по очереди две карты наугад.
Верно ли, что события <<первая карта "--- туз>> и <<вторая карта "---
туз>> являются независимыми? }
\solution{ Нет. }

\problem{
Известно, что $\PP(A)=0{,}3$, $\PP(B)=0,{4}$, $\PP(C)=0{,}5$. События
$A$ и $B$ несовместны, события $A$ и $C$ независимы и
$\PP(B\mid C)=0{,}1$.
Найдите $\PP(A\cup B\cup C)$. }
\solution{ }

\problem{
Имеется три монетки. Две <<правильных>> и одна "--- с орлами по
обеим сторонам. Петя выбирает одну монетку наугад и подкидывает её
два раза. Оба раза выпадает орёл. Какова вероятность того, что
монетка <<неправильная>>? }
\solution{ }

\problem{
Самолёт упал либо в горах, либо на равнине. Вероятность того, что самолёт упал в горах, равна 0{,}75. Для поиска пропавшего самолёта выделено 10 вертолётов. Каждый вертолёт можно использовать только в одном месте. Как распределить имеющиеся вертолёты, если вероятность обнаружения пропавшего самолёта отдельно взятым вертолётом равна: $0{,}95$? $0,6$ (пасмурно)? $0{,}1$ (туман)? }
\begin{ist}
Айвазян, экзамен РЭШ.
\end{ist}
\solution{ }

\problem{
Предположим, что социологическим опросам доверяют 70\,\% жителей. Те, кто доверяет опросам, всегда отвечают искренне; те, кто не доверяет, отвечают наугад, равновероятно выбирая <<да>> или <<нет>>. Социолог Петя  в анкету очередного опроса включил вопрос: <<Доверяете ли Вы социологическим опросам?>>
\begin{enumerate}
\item Какова вероятность, что случайно выбранный респондент ответит <<Да>>?
\item  Какова вероятность того, что он действительно доверяет, если известно, что он ответил <<Да>>?
\end{enumerate}
 }
\solution{ }

\problem{
Два охотника выстрелили в одну утку. Первый попадает с
вероятностью 0{,}4, второй "--- с вероятностью 0{,}6. В утку попала ровно
одна пуля. Какова вероятность того, что утка была убита первым
охотником? }
\solution{
$p=\frac{0{,}4\cdot 0{,}4}{0{,}4\cdot 0{,}4+0{,}6\cdot 0{,}6}=\frac{4}{13}$.}

\problem{
С вероятностью 0{,}3 Вася оставил конспект в одной из 10
посещённых им сегодня аудиторий. Вася осмотрел 7 из 10 аудиторий и
конспекта в них не нашёл.
\begin{enumerate}
\item  Какова вероятность того, что конспект будет найден в следующей
осматриваемой им аудитории?
\item  Какова (условная) вероятность того, что конспект оставлен
где-то в другом месте?
\end{enumerate}
 }
\solution{ }

\problem{
Вася гоняет на мотоцикле по единичной окружности с центром в
начале координат. В случайный момент времени он останавливается.
Пусть случайные величины  $X$  и  $Y$  "--- это Васины абсцисса и
ордината в момент остановки. Найдите  $\PP\left(X>\frac{1}{2} \right)$,
$\PP\left(X>\frac{1}{2} \bigm| Y<\frac{1}{2} \right)$. Являются ли события
$A=\left\{X>\frac{1}{2} \right\}$  и
$B=\left\{Y<\frac{1}{2} \right\}$  независимыми?
\begin{hint}
$\cos\left(\frac{\pi }{3} \right)=\frac{1}{2}$, длина окружности $l=2\pi r$.
\end{hint}
}
\solution{ }

\problem{
Пусть  $\PP(A)=1/4$,  $\PP(A \mid B)=\frac{1}{2}$  и $\PP(B\mid A)=\frac{1}{3}$. Найдите $\PP(A\cap
B)$, $\PP(B)$  и  $\PP(A\cup B)$.}
\solution{ }

\problem{
Примерно\footnote{Цифры условные. Celui qui ne mange pas de
bifsteak au cause de la vache folle --- il est fou! Jolivaldt.} 4\,\%
коров заражены <<коровьим бешенством>>. Имеется тест, позволяющий
с определённой степенью достоверности установить, заражено ли мясо
прионом или нет. С вероятностью $0{,}9$ заражённое мясо будет признано
заражённым. <<Чистое>> мясо будет признано заражённым с
вероятностью 0{,}1. Судя по тесту, эта партия мяса заражена. Какова
вероятность того, что она действительно заражена?}
\solution{ }

\problem{
\emph{Роме Протасевичу, искавшему со мной у Мутновского
вулкана в
тумане серую палатку...}

Есть две тёмные комнаты, $A$ и $B$. В одной из них сидит чёрная кошка.
Первоначально предполагается, что вероятность нахождения кошки в
комнате $A$ равна $\alpha$. Вероятность найти чёрную кошку в темной
комнате (если она там есть) с одной попытки равна $p$.  Допустим,
что вы сделали $a$ неудачных попыток поиска кошки в комнате $A$ и
$b$ неудачных попыток в комнате $B$.
\begin{enumerate}
\item Чему равна условная вероятность нахождения кошки в комнате $A$?
\item  При каком условии на $(a-b)$ эта вероятность будет больше
$0{,}5$?
\end{enumerate}
 }
\solution{ }

\problem{
Кубик подбрасывается два раза. Найдите вероятность
получить сумму, равную 8, если на первом кубике выпало 3.}
\solution{ $\frac{1}{6}$. }

\problem{
В коробке 10 пронумерованных монеток, $i$-я монетка выпадает
орлом с вероятностью $\frac{i}{10}$. Из коробки была вытащена одна
монетка наугад. Она выпала орлом. Какова вероятность того, что это
была пятая монетка? }
\solution{
$\frac{1}{11}$.  }

\problem{ Вы играете две партии в шахматы против незнакомца. Равновероятно
незнакомец может оказаться новичком, любителем или профессионалом.
Вероятности вашего выигрыша в отдельной партии, соответственно,
будут равны 0{,}9; 0{,}5; 0{,}3.
\begin{enumerate}
\item Какова вероятность выиграть первую партию?
\item Какова вероятность выиграть вторую партию, если вы выиграли
первую?
\end{enumerate}
 }

\solution{ $p_{a}=\frac{1}{3}(0{,}9+0{,}5+0{,}3)=\frac{17}{30}$, $p_{b}=\frac{1}{3}(0{,}9^{2}+0{,}5^{2}+0{,}3^{2})/p_{a}=\frac{115}{170}$. }

\problem{
В каких из перечисленных случаев вероятность наличия флэша (см. \hyperref[combo]{карточные комбинации} на стр.~\pageref{combo}) больше, чем при полном отсутствии информации:
\begin{enumerate}
\item Первая карта из имеющихся "--- это туз;
\item Первая карта из имеющихся "--- это туз бубей;
\item На руках имеется хотя бы один туз;
\item На руках имеется туз бубей.
\end{enumerate}
 }

\solution{ \ENGs Unverified, but no calculation:

An arbitrary hand can have two aces but a flush hand can't.  The
average number of aces that appear in flush hands is the same as the
average number of aces in arbitrary hands, but the aces are spread out
more evenly for the flush hands, so set (3) contains a higher fraction
of flushes.

Aces of spades, on the other hand, are spread out the same way over
possible hands as they are over flush hands, since there is only one of
them in the deck.  Whether or not a hand is flush is based solely on a
comparison between different cards in the hand, so looking at just one
card is necessarily uninformative.  So the other sets contain the same
fraction of flushes as the set of all possible hands. \RUSs }


\problem{\ENGs A man has 3 equally favorite seats to fish at. The probability with which the man can succeed at catching at each seat is 0.6, 0.7, 0.8 respectively. It is known that the man dropped the hint at one seat three times and just caught one fish. Find the probability that the fish was caught at the first seat. \RUSs}
\solution{$\approx 0{,}503$. }


\subsection{Условное среднее}


\problem{ \label{dve shkatulki} \zdt{Две шкатулки}

Васе предлагают две шкатулки и обещают, что в одной из них денег
в два раз больше, чем в другой. Вася открывает наугад одну из них
"--- в ней $a$ рублей. Вася может взять либо деньги, либо
оставшуюся шкатулку.
\begin{enumerate}
\item Правильно ли Вася считает, что ожидаемое количество денег в
неоткрытой шкатулке равно $\frac{1}{2}\left( {\frac{1} {2}a}
\right)+\frac{1}{2}( {2a} ) = 1\frac{1} {4}a$ и что
поэтому нужно изменить свой выбор?
\item Пусть известно, что в пару шкатулок кладут $3^k$ и $3^{k+1}$
рублей с вероятностью $p_k  = ( {\frac{1} {2}} )^k $. Стоит ли
Васе изменить свой выбор после того, как он открыл
первую шкатулку? Почему?
\end{enumerate}
 }
\solution{Вася считает неправильно: условное распределение суммы можно определить, только зная безусловное.

Концепция условного ожидания неприменима? Вставить это в иллюстрацию условного ожидания? При заданном безусловном распределении Васе следует сменить
свой выбор вне зависимости от того, что он увидел в первой шкатулке. Вторая открытая лучше первой открытой. Это возможно
из-за того, что безусловная ожидаемая сумма равна бесконечности
для обеих шкатулок.  }


\problem{В кабинет бюрократа скопилась очередь ещё до его открытия. Пусть время обслуживания страждущих "--- независимые экспоненциальные случайные величины. Посетитель, пришедший через $t$ минут после открытия, узнал, что первый посетитель уже ушёл, а второй ещё сидит в кабинете. Найдите ожидаемое время обслуживания первого посетителя, $\E(X_{1} \mid X_{1}\le t < X_{1}+X_{2}) $.}
\solution{$\frac{t}{2}$.}
\cat{poisson} \cat{exp} % может перекинуть в Пуассоновский процесс?


\problem{
Пете и Васе предложили одну и ту же задачу. Они могут правильно решить её с вероятностями 0{,}7 и 0{,}8 соответственно. К задаче предлагается 5 ответов на выбор, поэтому будем считать, что выбор каждого из пяти ответов равновероятен, если задача решена неправильно.
\begin{enumerate}
\item Какова вероятность несовпадения ответов Пети и Васи?
\item  Какова вероятность того, что Петя ошибся, если ответы совпали?
\item  Каково ожидаемое количество правильных решений, если ответы совпали?
\end{enumerate}
 }
\solution{ }

\problem{Автобусы ходят регулярно с интервалом в 10~минут. Вася приходит на остановку в случайный момент времени и ждёт автобуса не больше $a$ минут. Величина $a$ "--- константа из интервала $(0;10)$. Если автобус приходит меньше чем за $a$ минут, то Вася уезжает на нём. Если автобуса нет в течение $a$ минут, то Вася заходит в ближайшую кафешку перекусить и через случайное время возвращается на остановку. На второй раз он ждёт до прихода автобуса.
\begin{enumerate}
\item Какое время Вася в среднем проводит в ожидании автобуса?
\item  Постройте график получившейся функции от $a$.
\end{enumerate}
}
\solution{$f(a)=5+\frac{a(10-a)}{20}$.}



% !Mode:: "TeX:UTF-8"
\section{$\Var$, $\sigma$, conditional $\Var$, conditional $\sigma$}
\subsection{Дискретный случай}

\problem{ \label{do 2-h v summe}
Кубик подбрасывают до тех пор, пока накопленная сумма очков на
гранях не превысит 2. Пусть  $X$  "--- число подбрасываний кубика.
Найдите  $\E(X)$, $\Var(X)$,
$\Var(36X-5)$, $\E(36X-17)$. }
\solution{ \begin{tabular}{|c|c|c|c|} \hline
  % after \\: \hline or \cline{col1-col2} \cline{col3-col4} ...
  $X$ & 1 & 2 & 3 \\ \hline
  $\PP$ & $\frac{24}{36}$ & $\frac{11}{36}$ & $\frac{1}{36}$ \bigstrut \\  \hline
\end{tabular}

$\E(X)=\frac{49}{36}$, $\E(36X-17)=32$. $\Var(X)=\frac{371}{1296}$, $\Var(36X-5)=371$. }


\problem{ \label{iz 5 detalei 2 brakovannih}
Из 5-ти деталей 3 бракованных. Сколько потребуется в среднем
попыток, прежде чем обнаружится первая дефектная деталь? Какова
дисперсия числа попыток?}
\solution{\begin{tabular}{|c|c|c|c|} \hline
  % after \par: \hline or \cline{col1-col2} \cline{col3-col4} ...
  $X$ & 1 & 2 & 3 \\  \hline
 $\PP$ & $\frac{12}{20}$ & $\frac{6}{20}$ & $\frac{2}{20}$ \bigstrut \\
  \hline
\end{tabular} \par
$\E(X)=\frac{3}{2}$, $\Var(X)=\frac{9}{20}$.  }


\problem{
Бросают два правильных игральных кубика. Пусть  $X$  "--- наименьшая
из выпавших граней, а  $Y$  "--- наибольшая.
\begin{enumerate}
\item  Рассчитайте  $\PP(X=3\cap Y=5)$;
\item  Найдите  $\E(X)$,  $\Var(X)$, $\E(3X-2Y)$.
\end{enumerate}
\begin{ist}
Cut the knot.
\end{ist}
 }
\solution{ }

\problem{
Из колоды в 52 карты извлекаются две. Пусть  $X$  "--- количество
тузов. Найдите закон распределения  $X$, $\E(X)$, $\Var(X)$.}
\solution{ }

\problem{  \label{Iska priglasil 3 druzei}
Иська пригласил трёх друзей навестить его. Каждый из них появится
независимо от другого с вероятностью $0{,}9$, $0{,}7$ и $0{,}5$
соответственно. Пусть $N$ "--- количество пришедших гостей.
\begin{enumerate}
\item Рассчитайте вероятности $\PP(N=0)$, $\PP(N=1)$, $\PP(N=2)$ и $\PP(N=3)$;
\item Найдите $\E(N)$ и $\Var(N)$.
\end{enumerate}
 }
\solution{$\PP(N=0)=0{,}1\cdot 0{,}3\cdot 0{,}5$; $\PP(N=3)=0{,}9\cdot 0{,}7\cdot 0{,}5$;
$\PP(N=1)=0{,}9\cdot 0{,}3\cdot 0{,}5+0{,}1\cdot 0{,}7\cdot 0{,}5+0{,}1\cdot0{,}3\cdot 0{,}5$. $\E(N)=0{,}9+0{,}7+0{,}5$; $\Var(N)=0{,}9\cdot 0{,}1+0{,}7\cdot 0{,}3+0{,}5\cdot
0{,}5$.  }


\problem{
В коробке лежат три монеты: их достоинство "--- 1, 5 и 10 копеек соответственно. Они извлекаются в случайном порядке. Пусть $X_{1}$,  $X_{2}$  и  $X_{3}$  "--- достоинства монет в порядке их
появления из коробки.
\begin{enumerate}
\item Верно ли, что  $X_{1}$  и  $X_{3}$ одинаково распределены?
\item  Верно ли, что  $X_{1}$  и  $X_{3}$ независимы?
\item  Найдите $\E(X_{2})$
\item  Найдите дисперсию $\bar{X}_{2} =\frac{X_{1} +X_{2} }{2} $.
\end{enumerate}
 }
\solution{ }

\problem{  \zdt{Easy}

Пусть $X$ "--- сумма очков, выпавших в результате двукратного подбрасывания кубика. Найдите $\E(X)$, $\Var(X)$. }
\solution{ }

\problem{
Охотник, имеющий 4 патрона, стреляет по дичи до первого
попадания или до израсходования всех патронов. Вероятность
попадания при первом выстреле равна 0{,}6, а при каждом последующем
уменьшается на 0{,}1. Найдите
\begin{enumerate}
\item  Закон распределения числа патронов, израсходованных охотником;
\item  Математическое ожидание и дисперсию этой случайной величины.
\end{enumerate}
 }
\solution{ \small

\begin{tabular}{|c|c|c|c|c|}  \hline
  % after \\: \hline or \cline{col1-col2} \cline{col3-col4} ...
  $x_{i}$ & 1 & 2 & 3 & 4 \\  \hline
  $p_i=\PP(X=x_{i})$ & $0{,}6$& $(1-0{,}6)\cdot 0{,}5 = 0{,}2$ & $(1-0{,}6)\cdot(1-0{,}5)\cdot 0{,}4 = 0{,}08$ & $1-p_{1}-p_{2}-p_{3} = 0{,}12$ \\ \hline
\end{tabular}

\normalsize $\E(X)=1{,}7$, $\Var(X)\approx 1{,}08$. }


\subsection{Непрерывный случай}
%Здесь появляется:
%Е когда задана функция плотности

\problem{  \ENGs
A large quantity of pebbles lies scattered uniformly over a
circular field; compare the labour of collecting them on by one
\begin{enumerate}
\item At the center $O$ of the field;
\item At a point $A$ on the circumference.
\end{enumerate}\RUSs
\begin{ist}
Grimmett, экзамен 1858 года в St John's College, Кембридж.
\end{ist}
 }
\solution{ }

\subsection{? Способ расчёта ожидания и дисперсии через условную}

\problem{
Число $x$ выбирается равномерно на отрезке $[0; 1]$. Затем случайно выбираются числа из отрезка $[0; 1]$ до тех пор, пока не появится число больше $x$.
\begin{enumerate}
\item  Сколько в среднем потребуется попыток?
\item  Сколько в среднем потребуется попыток, если $x$ выбирается равномерно на отрезке $[0;r]$, $r<1$?
\item  Сколько в среднем потребуется попыток, $x$ не выбирается равномерно, а имеет функцию плотности $p(t)=2(1-t)$ для $t\in[0;1]$?
\end{enumerate}
 }
\solution{ }




% !Mode:: "TeX:UTF-8"
\section{Связь между случайными величинами, Cov, Corr}
\subsection{Дискретный случай}
%(частная/маржинальная/... ф распределения)


\problem{ \label{vtoroie podkidivanie}
 У Васи есть $n$ монеток,
каждая из которых выпадает орлом с вероятностью $p$. В первом
раунде Вася подкидывает все монетки, во втором раунде Вася
подкидывает только те монетки, которые выпали орлом в первом
раунде. Пусть $R_{i}$ "--- количество монеток,
подкидывавшихся и выпавших орлом во $i$-м раунде.
\begin{enumerate}
\item Каков закон распределения величины $R_{2}$?
\item  Найдите $\Corr(R_{1},R_{2})$.
\item  Как ведет себя корреляция при $p\to 0$ и $p\to 1$? Почему?
\end{enumerate}
 }
\solution{ Закон распределения по сути эксперимента: $\bin(n; p^{2}$); $\Corr(R_{1},R_{2}) =\sqrt{\frac{p}{1+p}}$. При $p\to 0$ корреляция равна 0; при  $p\to 1$ составляет 0{,}5. Почему "--- чез = чёрт его знает.
}



\problem{<<Корреляция --- это мера линейной связи>>

Найдите все случайные величины $X$ такие, что $corr(X,X^2)=1$.


Источник: Алексей Суздальцев}

\solution{Случайные величины, принимающие два значения $x_1<x_2$, такие, что $|x_1|<|x_2|$.}



\subsection{Непрерывный случай (Cov, Corr)}


\problem{Пусть $ X $ равномерно на $ [0;1] $. Если известно, что $ X=x $, то случайная величина $ Y $ равномерна на $ [x;x+1] $. Найдите $ P(X+Y<1) $ и $ f_{X|Y}(x|y) $. Как распределено $ Y-X $?}
\solution{$ P(X+Y<1)=1/4 $, $ f_{X|Y}(x|y)=1/f(y) $ при $ y\in [x;x+1], x\in[0;1] $. Равномерно.}


\subsection{Общие свойства Cov и Var}

\problem{Верно ли, что $X$ и $Y$  независимы, если известно, что
\begin{enumerate}
\item Величина $X$ и любая функция $g(Y)$ некоррелированы?
\item Любая функция $f(X)$ и любая функция $g(Y)$ некоррелированы?
\end{enumerate}
}
\solution{В первом случае "--- нет, например, $X \sim \mN(0;1)$ и $Y=X^2$. Во втором "--- да: возьмём индикаторы и получим стандартное определение независимости}


\subsection{Преобразование случайных величин (преобразования)}


\problem{ \label{simmetria razbitia otrezka}
На отрезке равномерно и независимо выбираются две точки. Верно ли,
что длины получающихся трёх отрезков распределены одинаково? }
\solution{Да. Возьмём окружность. Наугад отметим три точки. Одну будем
трактовать как разрезающую окружность на отрезок. Две других "---  как
разрезающие отрезок на три части. }
\cat{uniform} \cat{circle_trick}


\problem{ \zdt{Птички на проводе-1}

На провод, отрезок $[0; 1]$, равномерно и независимо друг от друга
садятся $n$ птичек. Пусть $Y_{1}$,\ldots, $Y_{n+1}$ "--- расстояния
от левого столба до первой птички, от первой птички от второй и т.\,д.
\begin{enumerate}
\item Найдите функцию плотности $Y_{1}$;
\item Верно ли, что все $Y_{i}$ одинаково распределены?
\item  Верно ли, что все $Y_{i}$ независимы?
\item Найдите $\Cov(Y_{i},Y_{j})$ (вроде бы ковариации равны?);
\item  Как распределена величина $n\cdot Y_{1}$ при больших $n$? Почему?
\end{enumerate}
 }
\solution{ Пусть $X_{i}$ "--- координата $i$-ой птички. $\PP(Y_{1}\le t)=1-\PP(\min\{X_{i}\}>t)=1-(1-t)^{n}$. Далее, $\lim \PP(nY_{1}\le t)=\lim 1-\left(1-\frac{t}{n}\right)^{n}=1-e^{-t}$. Распределение экспоненциальное с параметром $\lambda=1$. }
\cat{uniform} \cat{circle_trick} \cat{exponential}


\problem{ \zdt{Птички на проводе-2}

На провод, отрезок $[0;1]$, равномерно и независимо друг от друга
садятся $n$ птичек. Мы берем ведро жёлтой краски и для каждой
птички красим участок провода от неё до ближайшей к ней соседки.

Какая часть провода будет окрашена при больших $n$? }
\solution{ Пусть $n$ велико, тогда $Y_{i}$ можно считать независимыми и
$nY_{i}$ "--- экспоненциально распределёнными. Не красятся только <<большие>> интервалы, т.\,е. интервалы, чья
длина больше, чем каждого из двух соседних интервалов. <<Больших>>
интервалов примерно треть. Находим $\E(B)=\E(\max\{Y_{1},Y_{2},Y_{3}\})$. $\delta=1-\E(B)\frac{n}{3}=\frac{7}{18}$. }
\begin{ist}
Marcin Kuczma.
\end{ist}

\problem{Длительность разговора клиента в минутах $X$ "--- экспоненциальная случайная величина со средним две минуты. Стоимость разговора, $Y$, составляет $5$ рублей за весь разговор, если разговор короче двух минут, и $2{,}5$ рубля за минуту, если разговор длиннее двух минут. Неполные минуты оплачиваются пропорционально, например, за 3{,}5 минуты нужно заплатить $2{,}5\cdot 3{,}5$ рублей. Найдите $\E(Y)$, $\Var(Y)$, постройте функцию распределения $Y$. }
\solution{}

\problem{Вася пришёл на остановку. Ему нужен 42-й или 21-й автобус. Время до прихода 42-го равномерно на отрезке $[0;10]$ минут, время до прихода 21-го равномерно на отрезке $[0;20]$ минут. Время прихода 42-го и время прихода 21-го "--- независимые величины. Обозначим за $Y$  время, которое Вася проведёт в ожидании на остановке. Найдите функцию плотности $Y$, $\E(Y)$, $\Var(Y)$. }
\solution{}


\subsection{Прочее про несколько случайных величин}

\problem{\zdt{Парадокс голосования}
\par
Пусть $X$, $Y$, $Z$ "--- дискретные
случайные величины, их значения попарно различны с вероятностью 1.
Докажите, что $\min\left\{\PP(X>Y),\PP(Y>Z),\PP(Z>X)\right\}\le \frac{2}{3}$.
Приведите пример, при котором эта граница точно достигается. }
\solution{}


% !Mode:: "TeX:UTF-8"
\section{Приёмы решения}
\subsection{Разложение в сумму}
\problem{ \label{sudba-don-juan-2}\zdt{Судьба Дон-Жуана-2} (см. тж. с.~\pageref{sudba-don-juan-1})

У Васи $n$  знакомых девушек (их всех зовут по-разному). Он пишет
им $n$  писем, но по рассеянности раскладывает их в конверты
наугад. Случайная величина $X$ обозначает количество девушек, получивших письма,
написанные лично для них. Найдите $\E(X)$, $\Var(X)$. }
\solution{$\E(X)=1$, $\Var(X)=1$. }



\problem{ \label{cube-cut-2}(см. тж. с.~\pageref{cube-cut-1}) \ENGs

A wooden cube that measures 3 cm along each edge is painted red. The painted cube is then cut into 27 pieces of 1-cm cubes. If I tossed all the small cubes in the air, so that they landed randomly on the table, how many cubes should I expect to land with a painted face up? \RUSs}
\solution{ $9$.}


\problem{Вокруг новогодней ёлки танцуют хороводом 27 детей. Мы считаем, что ребенок высокий, если он выше обоих своих соседей. Сколько высоких детей в среднем танцует вокруг елки? Вероятность совпадания роста будем считать равной нулю.}
\solution{Для трёх детей вероятность того, что тот, что посередине "--- самый высокий, равна $ \frac{1}{3} $, значит математическое ожидание равно $ \frac{27}{3}=9$. }

\problem{Маша собирает свою дамскую сумочку. Есть $n$ различных предметов, которые она туда может положить. Каждый предмет она кладёт независимо от других с вероятностью $p$.
\begin{enumerate}
\item Пусть $X$ "--- количество положенных предметов. Найдите $\E(X)$ и $\Var(X)$.
\item При каком $p$ вероятность положить в сумку любой заданный набор вещей не будет зависеть от конкретного набора?
\end{enumerate}}
\solution{ Биномиальное распределение, $\E(X)=np$, $\Var(X)=np(1-p) $. При $p=0{,}5$ все подмножества будут равновероятны.}


\subsection{Первый шаг}

\problem{Илье Муромцу предстоит дорога к камню. И от камня начинаются ещё три дороги. Каждая из тех дорог снова оканчивается камнем. И от каждого камня начинаются ещё три дороги. И каждые те три дороги оканчиваются камнем\ldotst{} И так далее до бесконечности. На каждой дороге можно встретить живущего на ней трёхголового Змея Горыныча с вероятностью (хм, вы не поверите!) одна третья. Какова вероятность того, что Илья Муромец пройдет свой бесконечный жизненный путь, так ни разу и не встретив Змея Горыныча?}
\solution{$p=\frac{2}{3}(1-(1-p)^{3})$, нам подходит решение $ p=\frac{3-\sqrt{3}}{2} $. }


\problem{У Пети "--- монетка, выпадающая орлом с вероятностью $ p\in (0;1) $. У Васи "--- с вероятностью $ q\in (0;1) $. Они одновременно подбрасывают свои монетки до тех пор, пока у них не окажется набранным одинаковое количество орлов. В частности, они останавливаются после первого подбрасывания, если оно дало одинаковые результаты. Сколько в среднем раз им придётся подбросить монетку?}
\solution{}

\problem{Сколько в среднем нужно взять из колоды в 52 карты, чтобы насобирать подряд 5 карт одной масти?
\begin{hint}
Ответ имеет вид произведения дробей очень простого вида.
\end{hint}
}
\solution{Если у нас $m=13$ достоинств и $n=4$ масти, то ответ имеет вид: $mn\prod\limits_{i=1}^{}\frac{in}{in+1}\approx 45{,}3$.}

\problem{Вася прыгает на один метр вперёд с вероятностью $p$ и на два метра вперёд с вероятностью $1-p$. Как только он пересечёт дистанцию в 100~метров, он останавливается. Получается, что он может остановиться на отметке либо в 100~метров, либо в 101~метр. Какова вероятность того, что он остановится ровно на отметке в 100~метров?}
% копия в задачах на остановку мартингала
\solution{ Обозначим за $P_n$ вероятность остановиться ровно на $n$ метрах. Мы ищем $P_{100}$.

\textit{Решение 1.} По методу первого шага:  $P_n=pP_{n-1}+(1-p)P_{n-2}$.

\textit{Решение 2.} Попасть ровно в $n$ можно двумя способами: перелетев $n-1$ или попав в $n-1$ и сделав шаг в один метр. Значит $P_n=(1-P_{n-1})+pP_{n-1}$.

\textit{Решение 3.} Обозначим Васину координату в момент времени $t$ как $X_t$. Можно найти $a$ так, чтобы процесс $Y_t=a^{X_t}$ был мартингалом. Момент остановки $T=\min\{t \min X_t\geq n\}$. Мартингал $Y_{t\wedge T}$ ограничен, теорема Дуба применима. $\E(Y_T)=\E(Y_0)=1$. Получаем уравнение $P_n a^{n}+(1-P_n) a^{n+1}=1$.}

% untyp
\problem{
Испытания по схеме Бернулли проводятся до первого успеха, вероятность успеха в
отдельном испытании равна $p$ \par
а) Чему равно ожидаемое количество испытаний?   \par
б) Чему равно ожидаемое количество неудач? \par
в) Чему равна дисперсия количества неудач? }
\solution{ $\frac{1}{p}$, $\frac{q}{p}$ \par
в) $E(X^{2})=p\cdot 1+q\cdot E((X+1)^{2})$, $Var(X)=\frac{q}{p^2}$ }

% untyp
\problem{ Отрицательное биномиальное \par
Испытания по схеме Бернулли проводятся до $r$-го успеха, вероятность успеха в
отдельном испытании равна $p$ \par
а) Чему равно ожидаемое количество неудач? \par
б) Чему равна дисперсия количества неудач? }
\solution{ (устно, при сделанной предыдущей задаче) $\frac{rq}{p}$, $Var(X)=\frac{rq}{p^2}$ }

% untyp
\problem{
Саша и Маша по очереди подбрасывают кубик. Посуду будет
мыть тот, кто первым выбросит шестерку. Маша бросает первой.
Какова вероятность того, что Маша будет мыть посуду? }
\solution{ }

% untyp
\problem{
Саша и Маша решили, что будут рожать нового ребенка, до тех
пор, пока в их семье не будут дети обоих полов. Каково ожидаемое
количество детей? }
\solution{ }

% untyp
\problem{
Четыре человека играют в игру <<белая ворона платит>>. Они
одновременно подкидывают монетки. Если три монетки выпали одной
стороной, а одна - по-другому, то <<белая ворона>> оплачивает всей
четверке ужин в ресторане. Если <<белая ворона>> не определилась,
то монетки подбрасывают снова. Сколько в среднем нужно
подбрасывания для определения <<белой вороны>>? }
\solution{ }

% untyp
\problem{
Саша и Маша каждую неделю ходят в кино. Саша доволен
фильмом с
вероятностью 1/4, Маша - с вероятностью 1/3. \par
a) Сколько недель в среднем пройдет до тех пор, пока кто-то не
будет доволен? \par
b) Какова вероятность того, что первым будет доволен Саша? \par
c) Сколько недель в среднем пройдет до тех пор, пока каждый не
будет доволен хотя бы одним просмотренным фильмом? }
\solution{ }

% untyp
\problem{
По ответу студента на вопрос преподаватель может сделать
один из трех выводов: ставить зачет, ставить незачет, задать еще
один вопрос. Допустим, что знания студента и характер
преподавателя таковы, что при ответе на отдельный вопрос зачет
получается с вероятностью $p_{1}=3/8$, незачет - с вероятностью
$p_{2}=1/8$. Преподаватель задает вопросы до тех пор, пока не
определится
оценка. \par
а) Сколько вопросов в среднем будет задано? \par
б) Какова вероятность получения зачета? }
\solution{ }

% untyp
\problem{
Вы играете в следующую игру. Кубик подкидывается неограниченное число раз. Если на кубике выпадает 1, 2 или 3, то соответствующее количество монет добавляется на кон. Если выпадает 4 или 5, то игра оканчивается и Вы получаете сумму, лежащую на кону. Если выпадает 6, то игра оканчивается, а Вы не получаете ничего. \par
а) Чему равен ожидаемый выигрыш в эту игру? \par
б) Изменим условие: если выпадает 5, то набранная сумма сгорает, а игра начинается заново. Чему будет равен ожидаемый выигрыш? }
\solution{ 
a) $V(x)=\frac{1}{6}(V(x+1)+V(x+2)+V(x+3)+2x+0)$ \par
Ищем линейную $V(x)$, получаем $V(x)=\frac{2}{3}x+\frac{4}{3}$ \par
б) $V(x)=\frac{1}{6}(V(x+1)+V(x+2)+V(x+3)+x+V(0)+0)$ }

% untyp
\problem{ Вася подкидывает кубик. Если выпадает единица, или Вася говорит
<<стоп>>, то игра оканчивается, если нет, то начинается заново.
Васин выигрыш - последнее выпавшее число. Как выглядит оптимальная
стратегия? Как выглядит оптимальная стратегия, если за каждое
подбрасывание Вася платит 35 копеек?\cite{stirzaker:otep}}
\solution{ }

% untyp
\problem{
Саша и Маша подкидывают монетку бесконечное количество раз. Если сначала появится
РОРО, то выигрывает Саша, если сначала появится ОРОО, то - Маша. \par
а) У кого какие шансы выиграть? \par
b) Сколько в среднем времени ждать до появления РОРО? До ОРОО?
с) Сколько в среднем времени ждать до определения победителя? }
\solution{ }

% untyp
\problem{ \label{mishka ishet sir}
Есть три комнаты. В первой из них лежит сыр. Если мышка
попадает в первую комнату, то она находит сыр через одну минуту.
Если мышка попадает во вторую комнату, то она ищет сыр две минуты
и покидает комнату. Если мышка попадает в третью комнату, то она
ищет сыр три минуты и покидает комнату. Покинув комнату, мышка
выходит в коридор и выбирает новую комнату наугад (т.е. может
зайти в одну и ту же). Сейчас мышка в коридоре. Сколько времени ей
в среднем потребуется, чтобы найти сыр? }
\solution{ $m=\frac{1}{3}+\frac{1}{3}(2+m)+\frac{1}{3}(3+m)$, $m=6$ }

% untyp
\problem{
Иська и Еська по очереди подбрасывают два кубика. Иська
бросает первым. Иська выигрывает, если при своем броске получит 6
очков в сумме на двух кубиках. Еська выигрывает, если при своем
броске получит 7 очков в сумме на двух кубиков. Кубики
подбрасываются до
тех пор, пока не определится победитель. \par
а) Верно ли, что события $A=\{$на двух кубиках в сумме выпало
больше 5 очков$\}$ и $B=\{$на одном из кубиков выпала 1$\}$ являются независимыми? \par
б) Какова вероятность того, что Еська выиграет? }
\solution{ }

% untyp
\problem{
Players A and B play a (fair) dice game. <<A>> deposits one coin and
they take turns rolling a single dice, <<B>> rolling first. \par
If <<B>> rolls an even number, he collects a coin from the pot. If
he rolls an odd number, he put a coin (coins with same values
always). If <<A>> (plays and) rolls an even number, he collects a
coin but if he rolls an odd number, he does NOT add a coin. The
game continues until the pot is exhausted. \par
Question: what is the probability that <<A>> wins this game (that
is, exhaust the pot) ? \par
t=138358}
\solution{ }



\subsection{\textit{o}-малое}

\problem{Случайные величины $ X_{1} ,\ldots, X_{n} $ одинаково распределены с функцией плотности $ p(t) $ и независимы. Найдите функцию плотности третьего по величине $ X_{(3)}$.}
\solution{$ \PP(X_{(3)}\in [x;x+dx])= C_{n}^{2}C_{n-2}^{1} \bigl((F(x)+o(x)\bigr)^{n-3}\bigl(1-F(x)+o(x)\bigr)^{2}\bigl((f(x)+o(x)\bigr)dx $. Значит, искомая функция плотности равна $f_{X_{(3)}}(t)=f(t)F(t)^{n-3}\bigl(1-F(t)\bigr)^{2}$.}

\subsection{Вероятностный метод}
% задачи не по теории вероятностей, которые решаются с помощью теории вероятностей

% untyp
\problem{ На потоке 200 студентов. На контрольной было 6 задач. Известно, что каждую задачу решило не менее 120 человек.
Всегда ли преподаватель может выбрать двух студентов из потока так, что эти двое могут решить всю контрольную совместными усилиями?}
\solution{Выберем двух студентов из потока наугад. Вероятность того, что ни один из них не решил задачу \No\,1, не превосходит $\br{\frac{80}{200}}^2=0{,}16$.
Вероятность того, что ни один из них не решил задачу \No\,2, не превосходит 0{,}16 (по тем же причинам), и это справедливо для каждой из шести задач. Вероятность того, что хотя
бы одну задачу они на пару не решили, не превосходит суммы этих вероятностей, т.\,е. $0{,}16\cdot6=0{,}96$.
Значит, вероятность выбора пары студентов, которые совместными усилиями могут решить экзамен, не менее $0{,}04$. Значит, хотя бы одна такая пара существует.}


\subsection{Склеивание отрезка}


\problem{Машина может сломаться равновероятно в любой точке на дороге от города А до города Б. Когда машина сломается мы будем толкать ее до ближайшего сервиса. Где должны быть расположены три автосервиса чтобы минимизировать ожидаемую продолжительность толкания? А если автосервисов будет $n$?}
\solution{Проверить. Разбиваем отрезок на $n$ частей, ставим автосервис в центр каждой части
\url{http://math.stackexchange.com/questions/37254/} }


% !Mode:: "TeX:UTF-8"
\section{Неравенства, связанные с ожиданием}

\subsection{Чебышёв/Марков/Кантелли/Чернов}
\problem{
У последовательности неотрицательных случайных величин $X_1$, $X_2$,\ldots дисперсия постоянна, а  математическое ожидание стремится к бесконечности, $\lim \E(X)=+\infty$. Найдите $\lim \P(X_n>a)$. }
\solution{ $\P(X_n\leq q)=\P(X_n-\E(X_n)\leq a-\E(X_n))$. При больших $n$ величина $a-\E(X_n)$ отрицательна, поэтому $\P(X_n-\E(X_n)\leq a-\E(X_n))\leq \P(|X_n-\E(X_n)|\geq |a-\E(X_n)|)\leq c/(a-\E(X_n))^2$.}


\subsection{Йенсен}


\problem{Вася забрасывает удочку $100$ раз.
Вероятность поймать рыбку при одном забрасывании равна $p$. Петя забрасывает удочку случайное количество раз, $N$, под настроение. Известно, что $\E(N)=100$. У кого шансы поймать хотя бы одну рыбку выше?  }
\solution{ У Васи }

\problem{ Тысяча зайцев требует спасения. Дед Мазай выбирает между двумя стратегиями: 
\begin{itemize}
\item[A.] Перевозить зайцев равными партиями по 10 за заход.  
\item[B.] Перевозить зайцев случайными партиями от 1 до 19 зайцев за заход.
\end{itemize}
В каком случае ожидаемое количество заходов будет меньшим? }
\solution{ }



\problem{ При каких условиях $\E\left(\frac{1}{X}\right)=\frac{1}{\E(X)}$? }
\solution{Только если $X$ с вероятностью 1 константа. }


\subsection{Коши"--~Шварц}


\section{Сходимости}

\problem{Определите, к какому распределению сходятся указанные последовательности
\begin{enumerate}
\item $X_n \sim N(\frac{n-1}{n+1},9)$
\item $X_n \sim N(7,\frac{5+n}{n^2})$
\item $X_n \sim t_n$
\item $X_n \sim \frac{Y_n}{n}$, где $Y_n \sim \chi_n^2$
\item $X_n \sim \frac{Y_n}{n+5}$, где $Y_n \sim \chi_n^2$
\item $X_n \sim \frac{Y_n-n}{\sqrt{n}}$, где $Y_n \sim \chi_n^2$
\item $X_n \sim 2011 Y_n$, где $F_{2011,n}$
\end{enumerate}
}
\solution{}
% !Mode:: "TeX:UTF-8"
\section{Компьютерные (use R or Python!)}

\subsection{Нахождение сложных сумм/поиск оптимальных стратегий}
\problem{Перед нами 10 коробок. Изначально в 1-й коробке 1 шар, во 2-й "--- 2 шара и т.\,д. Мы равновероятно выбираем одну из коробок, вытаскиваем из неё шар и кладём его равновероятно в одну из девяти оставшихся коробок. Мы повторяем это перекладывание до тех пор, пока одна из коробок не станет пустой. Пусть $N$ "--- число перекладываний. С помощью компьютера оцените $\E(N) $, $\Var(N)$.}
\solution{ $ \E(N)\approx 12{,}15$.}

\problem{В классе учатся $n$ человек. Нас интересует вероятность того, что хотя бы у двух из них дни рождения будут в соседние дни. (31 декабря и 1 января будем считать соседними). При каком $n$ эта вероятность впервые достигнет 0{,}5?}
\solution{ $\PP_{16}=0{,}482\,390\,182$, $\PP_{17}=0{,}525\,836\,596$.}

\problem{В классе 30 человек. Какова вероятность того, что есть три человека, у которых совпадают дни рождения? Найдите ответ с помощью симуляций и с помощью пуассоновского приближения. При каком количестве человек эта вероятность впервые превысит 50\,\%?}
\solution{Симуляции $p=0{,}028\,5 $, Пуассон: $p=0{,}03$.}

\problem{Сколько нужно людей, чтобы вероятность того, что в каждый день года у кого-то день рождения, впервые превысила 50\,\%?}
\solution{$\PP(T\leq k)= n^{-k}n!\left\{ \begin{array}{c} k \\ n \end{array} \right\}$ (число Стирлинга второго рода); 2287.}

\problem{Сколько в среднем нужно взять из колоды в 52 карты, чтобы насобирать подряд 5 карт одной масти? Не обязательно одной масти?}
\solution{Если у нас $m=13$ достоинств и $n=4$ масти, то ответ имеет вид: $mn\prod\limits_{i=1}^{}\frac{in}{in+1}\approx 45{,}3$; $\approx 28{,}0$.}

% untyp
\problem{Маленький мальчик торгует на улице еженедельной газетой. Покупает
он ее по 15 рублей, а продает по 30 рублей. Количество потенциальных покупателей --- случайная величина с распределением Пуассона и средним
значением равным 50. Нераспроданные газеты ничего не стоят. Пусть $n$ --- количество газет, максимизирующее ожидаемую прибыль мальчика.
\begin{enumerate}
\item Чему примерно должно быть равно значение функции распределения в
точке  $n$?
\item  С помощью компьютера найдите  $n$ и ожидаемую прибыль.
\end{enumerate}}
\solution{$n=50$, $665.51$

\inputminted{python}{src_python/newspapers_notext.py}
}



\problem{Подбрасывается правильная монетка. В любой момент вы можете сказать <<Хватит>>. Ваш выигрыш равен доле орлов на момент остановки. С помощью компьютера определите, чему равен ожидаемый выигрыш при использовании оптимальной стратегии? При решении на компьютере можно считать, что число подбрасываний ограничено скажем 500.}
\solution{Около 0.7925}

\problem{У Васи есть 100 рублей. Вася открывает карты из колоды одну за одной в случайном порядке. В колоде 26 красных и 26 чёрных карт. Перед открытием каждой карты Вася может поставить на цвет любую целую сумму рублей в пределах своего капитала. Если он угадал цвет, то его ставка возвращается удвоенной, если нет, то он теряет ставку. Задача Васи --- максимизировать ожидаемый финальный выигрыш. С помощью компьютера определите, как выглядит оптимальная стратегия и какую сумму он в среднем выигрывает? }
\solution{Ожидаемая сумму в концу игры - 808 рублей}



\subsection{Проведение симуляций}



% (к статистике) проверка простых гипотез на РЕАЛЬНЫХ (исторических) данных

\subsection{Statistics}
\problem{Петя подбрасывал две монетки неправильные монетки. Результаты подбрасывания:

Число подбрасываний первой. Число подбрасываний второй. Общее число орлов.
... ... ...

... ... ...
... ... ...


Оцените с помощью компьютера вероятности выпадения орлом для каждой монетки. Постройте доверительные интервалы.
(Красивого решения в явном виде нет).

Можно использовать нормальное приближение


}
\solution{}


\problem{Голосовать можно за трёх кандидатов: А, Б и В. Из 100 опрошенных 20 хотят голосовать за А, 40 "--- за Б, остальные "--- за В. В осях $p_{A}$, $p_{B}$ постройте 90-процентную доверительную двумерную область.}
\solution{ }




% !Mode:: "TeX:UTF-8"
\section{Пуассоновский поток и экспоненциальное распределение}

\subsection{Пуассоновский поток}

\problem{Саша красит стены в своей комнате, а Алёша "--- в своей. У каждой комнаты четыре стены. Предположим, что время покраски одной стены и для Саши, и для Алёши "--- экспоненциальная случайная величина с параметром $\lambda$. Какова вероятность того, что Саша успеет покрасить 3 стены раньше, чем Алёша "--- две?}

\solution{Каждая следующая стена равновероятно покрашена Сашей и Алёшей. Значит, нам нужны $\PP(SSS)+\PP(SSAS)+\PP(SASS)+\PP(ASSS)=\frac{5}{16}$. По другому: для простоты положим $\lambda=1$. Пусть $T$ "--- время, когда Саша закончит 3 стены. Функция плотности гамма-распределения (сумма трёх экспоненциальных) $f(t)=0{,}5t^{2}e^{-t}$. Нам нужна вероятность того, что к тому времени Алёша успеет меньше двух стен: $\int_{0}^{\infty} \PP(N_{t}<2 \mid T=t)\,dt =\ldots=\frac{5}{16}$.}

\problem{Машины подъезжают к светофору пуассоновским потоком с интенсивностью $\lambda $. Для простоты будем считать, что первая машина подъезжает в $ t=0 $. Светофор горит зелёным только в целые моменты времени, и этого достаточно чтобы пропустить одну машину, т.\,е. светофор горит красным при $ t\in(0;1) $, $ t\in(1;2) $, $ t\in(2;3) $ и т.\,д. Какой будет средняя длина очереди через продолжительное время? Чему будет равна вероятность, что очередь пуста?}
\solution{Производящая функция удовлетворяет соотношению:
\[ g(t)=\exp(\lambda (t-1))\frac{g(t)+tg(0)-g(0)}{t} \]
\[ g(t)=g(0)\frac{(t-1)\exp(\lambda (t-1))}{t-\exp(\lambda (t-1))} \]
Из условия $ g(1)\to 1 $ находим $ g(0)=1-\lambda $ и, помучившись, $\E(X_{\infty})=g'(1)=\frac{\lambda(2-\lambda)}{2\cdot(1-\lambda)} $.}

\cat{Poisson} \cat{gen_fun}

\problem{В офисе два телефона: зелёный и красный. Входящие звонки на красный "--- Пуассоновский поток событий с интенсивностью $\lambda_{1}=4$ звонка в час, входящие на зелёный "--- с интенсивностью $\lambda_2=5$ звонков в час. Секретарша Василиса Премудрая одна в офисе. Время разговора "--- случайная величина, имеющая экспоненциальное распределение со средним временем $5$ минут. Если Василиса занята разговором, то на второй телефон она не отвечает. Сколько звонков в час в среднем пропустит Василиса, потому что будет занята разговором по другому телефону? Являются ли пропущенные звонки Пуассоновским потоком? }
\solution{}

\problem{В офисе два телефона: зелёный и красный. Входящие звонки на красный "--- Пуассоновский поток событий с интенсивностью $\lambda_{1}=4$ звонка в час, входящие на зелёный "--- с интенсивностью $\lambda_2=5$ звонков в час. Секретарша Василиса Премудрая одна в офисе. Перед началом рабочего дня она подбрасывает монетку и отключает один из телефонов: зелёный "--- если выпала решка, красный "--- если выпал орёл. Обозначим за $Y$ время от начала дня до первого звонка. Найдите функцию плотности $Y$. }
\solution{}

\problem{Случайная величина $X$ имеет экспоненциальное распределение с параметром $\lambda$. Найдите медиану $X$. }
\solution{}

\subsection{Пуассоновское приближение}
% при замене на Poisson(\lambda=np) ошибка не превосходит
% min(1,1/\lambda)\sum p_{i}^{2}


% !Mode:: "TeX:UTF-8"
\section{Нормальное распределение и ЦПТ}
\subsection{Одномерное нормальное распределение}

\problem{Где находятся точки перегиба функции плотности для нормального распределения?}
\solution{$ \mu \pm \sigma $.}

\problem{Пусть $X\sim \mN(\mu;\sigma^{2})$ и $t>\mu$. В какой точке функция $\PP(X\in [t;t+dt])$ убывает быстрее всего?}
\solution{$ \mu + \sigma $.}

\problem{Имеются две акции с доходностями $X$ и $Y$ на один вложенный рубль. Доходности некоррелированы, $\E(X)=0{,}09$, $\E(Y)=0{,}04$, $\sigma_X=0{,}5$, $\sigma_Y=0{,}1$. У инвестора есть 1~рубль. Он покупает на $a\in (0;1)$ рубля первых акций и на $(1-a)$ вторых акций. Обозначим за $\mu(a)$ и $\sigma(a)$ ожидаемую доходность и стандартное отклонение доходности полученного портфеля.
\begin{enumerate}
\item Постройте множество возможных $\mu(a)$ и $\sigma(a)$
\item Найдите наименее рисковый портфель. Чему равна его ожидаемая доходность?
\end{enumerate} }
\solution{}

\problem{Имеются две акции с доходностями $X$ и $Y$ на один вложенный рубль. Доходности некоррелированы, $\E(X)=0{,}09$, $\E(Y)=0{,}04$, $\sigma_X=0{,}5$, $\sigma_Y=0{,}1$. У инвестора есть 1~рубль. Инвестор подкидывает неправильную монетку, выпадающую орлом с вероятностью $p$. Если монетка выпадает орлом, он покупает первые акции, если решкой, то вторые. Обозначим за $\mu(p)$ и $\sigma(p)$ ожидаемую доходность и стандартное отклонение доходности полученной стратегии.
\begin{enumerate}
\item Постройте множество возможных $\mu(p)$ и $\sigma(p)$.
\item Найдите функцию плотности доходности полученной стратегии.
\end{enumerate} }
\solution{}

\subsection{ЦПТ}
\problem{
Вася играет в компьютерную игру "--- <<стрелялку"=бродилку>>. По
сюжету ему нужно убить 60 монстров. На один выстрел уходит ровно 1
минута. Вероятность убить монстра с одного выстрела равна 0{,}25.
Количество выстрелов не ограничено.
\begin{enumerate}
\item Сколько времени в среднем Вася тратит на одного монстра?
\item  Найдите дисперсию этого времени.
\item  Какова вероятность того, что Вася закончит игру меньше, чем за
3 часа?
\end{enumerate}
 }
\solution{ }


\subsection{Многомерное нормальное распределение}

\problem{
Ермолай Лопахин решил приступить к вырубке вишневого сада. Однако выяснилось, что растут в нём не только вишни, но и яблони. Причём, по словам Любови Андреевны Раневской, среднее количество деревьев (а они периодически погибают от холода или жары, либо из семян вырастают новые) в саду распределено в соответствии с нормальным законом ($X$ "--- число яблонь, $Y$ "--- число вишен) со следующими параметрами:
\begin{equation}
\begin{pmatrix}	X \\ 	Y 	\end{pmatrix}
\sim \mN
\left(
\begin{pmatrix}
25 \\ 125
\end{pmatrix}
;
\begin{pmatrix}
	5 & 4 \\
	4 & 10
	\end{pmatrix}
\right)
\end{equation}

Найдите вероятность того, что Ермолаю Лопахину придется вырубить более 150~деревьев.
Каково ожидаемое число подлежащих вырубке вишен, если известно, что предприимчивый и последовательный Лопахин, не затронув ни одного вишнёвого дерева, начал очистку сада с яблонь и все 35~яблонь уже вырубил?

Автор: Кирилл Фурманов, Ира ...}
\solution{ }

\subsection{Распределения связанные с нормальным}

% задачи без статистики на свойства t, F, chi распределений

\problem{Сравните $\E(F_{k,n})$ и $Var(t_n)$.}
\solution{$\E(F_{k,n})=Var(t_n)$}
% !Mode:: "TeX:UTF-8"
\section{Случайное блуждание}
\subsection{Дискретное случайное блуждание}


\problem{Какова вероятность того, что трёхмерное случайное блуждание бесконечное количество раз пересечёт вертикальную ось?}
\solution{1. Про вертикальные шаги можно забыть, а вероятность бесконечное количество раз посетить 0 для двумерного случайного блуждания равна 1.}


\problem{Пусть $X_{n}$ "--- симметричное случайное блуждание. Сколько времени в среднем придётся ждать, пока остаток от деления $X_{n}$ на 183 окажется равным 39?}
\solution{$39^{2}$.}



\subsection{Принцип отражения}
\problem{  \zdt{Выборы} \par
После выборов, в которых участвуют два кандидата, A и B, за них
поступило $a$ и $b$ ($a>b$) бюллетеней соответственно, скажем, 3 и
2. Если подсчёт голосов производится последовательным извлечением
бюллетеней из урны, то какова вероятность того, что хотя бы один
раз число вынутых бюллетеней, поданных за A и B, было одинаково? Какова вероятность того, что A всё время лидировал?
\begin{ist}
Mosteller.
\end{ist}
}
\solution{ }

\problem{ \zdt{Ничьи при бросании монеты} \par
Игроки A и B в орлянку играют $n$ раз. После первого бросания
каковы шансы на то, что в течение всей игры их выигрыши не
совпадут?
\begin{ist}
Mosteller.
\end{ist}}
\solution{ }

\problem{Доходность акции следует симметричному дискретному случайному блужданию. Какова вероятность того, что в момент времени $2k+1$ доходность будет выше, чем когда-либо в прошлом?

Источник: Алексей Суздальцев}
\solution{$\frac{C_{2k}^{k}}{2^{2k+1}}$. Совсем простого решения не знаю, хотя ответ простой.}



\subsection{Броуновское движение}

