% !Mode:: "TeX:UTF-8"

% untyp
\problem{ 
При подбрасывании кубика грани выпали 234, 229, 240, 219,
236 и 231 раз соответственно. Проверьте гипотезу о том, что кубик
<<правильный>>. } 
\solution{} 

% untyp
\problem{
Проверьте независимость дохода и пола по таблице: \\
\begin{tabular}{|c|c|c|c|}
  \hline
   & $<500$ & $500-1000$ & $>1000$ \\
  \hline
  М & 112 & 266 & 34 \\
  Ж & 140 & 152 & 11 \\
  \hline
\end{tabular} } 
\solution{} 

% untyp
\problem{
Вася Сидоров утверждает, что ходит в кино в два раза чаще, чем в
спортзал, а в спортзал в два раза чаще, чем в театр. За последние
полгода он 10 раз был в театре, 17 раз - в спортзале и
39 раз - в кино. Правдоподобно ли Васино утверждение? } 
\solution{} 

% untyp
\problem{
Проверьте независимость пола респондента и предпочитаемого
им сока: \par
\begin{tabular}{|c|c|c|c|}
  \hline
   & Апельсиновый & Томатный & Вишневый \\
  \hline
  М & 69 & 40 & 23 \\
  Ж & 74 & 62 & 34 \\
  \hline
\end{tabular} } 
\solution{} 


% untyp
\problem{
У 200 человек записали цвет глаз и волос. На уровне значимости
10\% проверьте гипотезу о независимости этих признаков. \par
\begin{tabular}{|c|c|c|c|}
  \hline
  Цвет глаз/волос & Светлые & Темные & Итого \\
  \hline
  Зеленые & 49 & 25 & 74 \\
  Другие & 30 & 96 & 126 \\
  \hline
  Итого & 79 & 121 & 200 \\
  \hline
\end{tabular} } 
\solution{} 

% untyp
\problem{
Идея задачи на хи-квадрат. \par
Если предложить голосовать за 3 альтернативы... \par
Если предложить голосовать за 4 альтернативы... \par
Выполняется ли предпосылка независимости от посторонних
альтернатив? } 
\solution{} 



% untyp
\problem{Когда Пирсон придумал хи-квадрат тест на независимость признаков (около 1900 г.), он не был уверен в правильном количестве степеней свободы. Он разошелся во мнениях с Фишером. Фишер считал, что для таблицы сопряженности размера два на два хи-квадрат статистика будет иметь три степени свободы, а Пирсон - что одну. Чтобы выяснить истину, Фишер взял большое количество таблиц два на два с заведомо независимыми признаками и посчитал среднее значение хи-квадрат статистики. Чему оно оказалось равно? Почему этот эксперимент помог выяснить истину?}
\solution{Среднее значение хи-квадрат случайной величины равно числу степеней свободы. Единице. Historical Note (as told by Chris Olsen): 
The Chi-square statistic was invented by Karl Pearson about 1900. Pearson knew what the Chi-square distribution looks like, but he was unsure about the degrees of freedom. 

About 15 years later, Fisher got involved. He and Pearson were unable to agree on the degrees of freedom for the two-by-two table, and they could not settle the issue mathematically. Pearson believed there was 1 degree of freedom and Fisher 3 degrees of freedom. 

They had no nice way to do simulations, which would be the modern approach, so Fisher looked at lots of data in two-by-two tables where the variables were thought to be
independent. For each table he calculated the Chi-square statistic. Recall that the expected value for the Chi-square statistic is the degrees of freedom. After collecting many Chi-square values, Fisher averaged all the values and got a result he described as <<embarrassingly close to 1>> 

This confirmed that there is one degree of freedom for a two-by-two table. Some years later this result was proved mathematically. }

