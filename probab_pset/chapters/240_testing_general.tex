% !Mode:: "TeX:UTF-8"
% тестирование гипотез. общие свойства


\problem{Вовочка тестирует гипотезу $H_{0}$ против гипотезы $H_{a}$. Предположим, что $H_{0}$ на самом деле верна. По своей сути p-value является случайной величиной. Какое распределение оно имеет?}
\solution{Равномерное на $[0;1]$ }


\problem{Гражданин Фёдор решает проверить, не жульничает ли напёрсточник Афанасий, для чего предлагает Афанасию сыграть 5 партий в напёрстки. Фёдор решает, что в каждой партии будет выбирать один из трёх напёрстков наугад, не смотря на движения рук ведущего. Основная гипотеза: Афанасий честен, и вероятность правильно угадать напёрсток, под которым спрятан шарик, равна 1/3. Альтернативная гипотеза: Афанасий каким-то образом жульничает (например, незаметно прячет шарик), так что вероятность угадать нужный напёрсток меньше, чем 1/3. Статистический критерий: основная гипотеза отвергается, если Фёдор ни разу не угадает, где шарик.
\begin{enumerate}
\item Найдите уровень значимости критерия.
\item Найдите мощность критерия в том случае, когда Афанасий жульничает, так что вероятность угадать нужный напёрсток равна 1/5.
\end{enumerate}}
\solution{}


