% !Mode:: "TeX:UTF-8"
% тестирование гипотез. общие свойства
\section{Тестирование гипотез --- общие свойства}

\problem{Вовочка тестирует гипотезу $H_{0}$ против гипотезы $H_{a}$. Предположим, что $H_{0}$ на самом деле верна. По своей сути p-value является случайной величиной. Какое распределение оно имеет?}
\solution{Равномерное на $[0;1]$ }


\problem{Гражданин Фёдор решает проверить, не жульничает ли напёрсточник Афанасий, для чего предлагает Афанасию сыграть 5 партий в напёрстки. Фёдор решает, что в каждой партии будет выбирать один из трёх напёрстков наугад, не смотря на движения рук ведущего. Основная гипотеза: Афанасий честен, и вероятность правильно угадать напёрсток, под которым спрятан шарик, равна 1/3. Альтернативная гипотеза: Афанасий каким-то образом жульничает (например, незаметно прячет шарик), так что вероятность угадать нужный напёрсток меньше, чем 1/3. Статистический критерий: основная гипотеза отвергается, если Фёдор ни разу не угадает, где шарик.
\begin{enumerate}
\item Найдите уровень значимости критерия.
\item Найдите мощность критерия в том случае, когда Афанасий жульничает, так что вероятность угадать нужный напёрсток равна 1/5.
\end{enumerate}}
\solution{}

\problem{
Имеется две конкурирующие гипотезы: \par
$H_{0}$: Величина $X$ распределена равномерно на отрезке $[0;100]$ \par
$H_{a}$: Величина $X$ распределена равномерно на отрезке $[50;150]$ \par
Исследователь выбрал такой критерей: \par
Если $X<c$, то использовать $H_{0}$, иначе использовать $H_{a}$. 
\begin{enumerate}
\item Что такое <<ошибка первого рода>>, <<ошибка второго рода>>,
<<мощность теста>>?
\item Постройте графики зависимостей ошибок первого и второго рода от
$c$.  
\end{enumerate}
} 
\solution{} 

\problem{
Известно, что  $X_{i}$ iid $N\left(\mu ;900\right)$ .
Исследователь проверяет гипотезу $H_{0}$: $\mu =10$  против
$H_{A}$: $\mu =30$  по выборке из 20 наблюдений. Критерий выглядит
следующим образом: если  $\bar{X}>c$ , то выбрать  $H_{A} $ ,
иначе выбрать  $H_{0} $.
\begin{enumerate}
\item  Рассчитайте вероятности ошибок первого и второго рода, мощность критерия для $c=25$. 
\item Что произойдет с указанными вероятностями при росте количества
наблюдений ($c\in(10;30)$)? 
\item Каким должно быть $c$, чтобы вероятность ошибки второго рода
равнялась $0,15$? 
\item Как зависят от $c$ вероятности ошибок первого и второго рода
($c\in(10;30)$)? 
\end{enumerate} }
\solution{} 

\problem{
Дама утверждает, что обладает особыми способностями и безошибочно
отличает <<бонакву>> без газа от <<святого источника>> без газа.
$H_{0}$: дама не обладает особыми способностями, $H_{a}$: дама
обладает особыми способностями. При даме в 3 стаканчика из 8-ми
налили <<бонакву>>, а в 5 оставшихся --- <<святой источник>>. При
отгадывании стаканчики предлагаются даме в неизвестном ей порядке.
Критерий: принимается основная гипотеза, если дама ошиблась хотя
бы один раз и альтернативная иначе. 
\begin{enumerate}
\item Рассчитайте вероятности ошибок первого и второго рода, мощность
критерия.
\item Сколько из 8 стаканчиков надо наполнить <<бонаквой>> и сколько
<<святым источником>>, чтобы вероятность ошибки первого рода была
минимальной?
\end{enumerate}
Коммент: некоторые студенты утверждают, что отличить <<святой
источник>> от <<бонаквы>> --- элементарно, а вот отличить
<<бонакву>> от <<акваминерале>> --- трудно. } 
\solution{} 


\problem{Аня и Таня любят мыть посуду. Аня разбивает тарелку, которую моет, с вероятностью $ p_{a} $, Таня --- с вероятностью $ p_{t} $. Они хотят проверить гипотезу $ H_{0}: p_{a}=p_{t} $ против альтернативной $ H_{a}: p_{a}>p_{t} $. Способ проверки следующий. Они по очереди будут мыть по одной тарелки, до тех пор пока не разобьется две тарелки. Если обе эти тарелки будут разбиты Аней, то $ H_{0} $ будет отвергнута в пользу $ H_{a} $. Определите вероятность ошибки первого рода как функцию от $p_{a}$, если:
\begin{enumerate}
\item Первой моет тарелку Аня
\item Первой моет тарелку Таня
\end{enumerate} }
\solution{ Если начинает Аня, то $ \frac{1-p}{(2-p)^{2}} $; Если начинает Таня, то $ \frac{(1-p)^{2}}{(2-p)^{2}} $. Источник: Кирилл Фурманов}


