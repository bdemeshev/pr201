% !Mode:: "TeX:UTF-8"
\section{Пуассоновский поток и экспоненциальное распределение}

\subsection{Пуассоновский поток}

\problem{Саша красит стены в своей комнате, а Алёша "--- в своей. У каждой комнаты четыре стены. Предположим, что время покраски одной стены и для Саши, и для Алёши "--- экспоненциальная случайная величина с параметром $\lambda$. Какова вероятность того, что Саша успеет покрасить 3 стены раньше, чем Алёша "--- две?}

\solution{Каждая следующая стена равновероятно покрашена Сашей и Алёшей. Значит, нам нужны $\PP(SSS)+\PP(SSAS)+\PP(SASS)+\PP(ASSS)=\frac{5}{16}$. По другому: для простоты положим $\lambda=1$. Пусть $T$ "--- время, когда Саша закончит 3 стены. Функция плотности гамма-распределения (сумма трёх экспоненциальных) $f(t)=0{,}5t^{2}e^{-t}$. Нам нужна вероятность того, что к тому времени Алёша успеет меньше двух стен: $\int_{0}^{\infty} \PP(N_{t}<2 \mid T=t)\,dt =\ldots=\frac{5}{16}$.}

\problem{Машины подъезжают к светофору пуассоновским потоком с интенсивностью $\lambda $. Для простоты будем считать, что первая машина подъезжает в $ t=0 $. Светофор горит зелёным только в целые моменты времени, и этого достаточно чтобы пропустить одну машину, т.\,е. светофор горит красным при $ t\in(0;1) $, $ t\in(1;2) $, $ t\in(2;3) $ и т.\,д. Какой будет средняя длина очереди через продолжительное время? Чему будет равна вероятность, что очередь пуста?}
\solution{Производящая функция удовлетворяет соотношению:
\[ g(t)=\exp(\lambda (t-1))\frac{g(t)+tg(0)-g(0)}{t} \]
\[ g(t)=g(0)\frac{(t-1)\exp(\lambda (t-1))}{t-\exp(\lambda (t-1))} \]
Из условия $ g(1)\to 1 $ находим $ g(0)=1-\lambda $ и, помучившись, $\E(X_{\infty})=g'(1)=\frac{\lambda(2-\lambda)}{2\cdot(1-\lambda)} $.}

\cat{Poisson} \cat{gen_fun}

\problem{В офисе два телефона: зелёный и красный. Входящие звонки на красный "--- Пуассоновский поток событий с интенсивностью $\lambda_{1}=4$ звонка в час, входящие на зелёный "--- с интенсивностью $\lambda_2=5$ звонков в час. Секретарша Василиса Премудрая одна в офисе. Время разговора "--- случайная величина, имеющая экспоненциальное распределение со средним временем $5$ минут. Если Василиса занята разговором, то на второй телефон она не отвечает. Сколько звонков в час в среднем пропустит Василиса, потому что будет занята разговором по другому телефону? Являются ли пропущенные звонки Пуассоновским потоком? }
\solution{}

\problem{В офисе два телефона: зелёный и красный. Входящие звонки на красный "--- Пуассоновский поток событий с интенсивностью $\lambda_{1}=4$ звонка в час, входящие на зелёный "--- с интенсивностью $\lambda_2=5$ звонков в час. Секретарша Василиса Премудрая одна в офисе. Перед началом рабочего дня она подбрасывает монетку и отключает один из телефонов: зелёный "--- если выпала решка, красный "--- если выпал орёл. Обозначим за $Y$ время от начала дня до первого звонка. Найдите функцию плотности $Y$. }
\solution{}

\problem{Случайная величина $X$ имеет экспоненциальное распределение с параметром $\lambda$. Найдите медиану $X$. }
\solution{}

\subsection{Пуассоновское приближение}
% при замене на Poisson(\lambda=np) ошибка не превосходит
% min(1,1/\lambda)\sum p_{i}^{2}

