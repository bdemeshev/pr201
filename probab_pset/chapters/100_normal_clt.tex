% !Mode:: "TeX:UTF-8"
\section{Нормальное распределение и ЦПТ}
\subsection{Одномерное нормальное распределение}

\problem{Где находятся точки перегиба функции плотности для нормального распределения?}
\solution{$ \mu \pm \sigma $.}

\problem{Пусть $X\sim \mN(\mu;\sigma^{2})$ и $t>\mu$. В какой точке функция $\PP(X\in [t;t+dt])$ убывает быстрее всего?}
\solution{$ \mu + \sigma $.}

\problem{Имеются две акции с доходностями $X$ и $Y$ на один вложенный рубль. Доходности некоррелированы, $\E(X)=0{,}09$, $\E(Y)=0{,}04$, $\sigma_X=0{,}5$, $\sigma_Y=0{,}1$. У инвестора есть 1~рубль. Он покупает на $a\in (0;1)$ рубля первых акций и на $(1-a)$ вторых акций. Обозначим за $\mu(a)$ и $\sigma(a)$ ожидаемую доходность и стандартное отклонение доходности полученного портфеля.
\begin{enumerate}
\item Постройте множество возможных $\mu(a)$ и $\sigma(a)$
\item Найдите наименее рисковый портфель. Чему равна его ожидаемая доходность?
\end{enumerate} }
\solution{}

\problem{Имеются две акции с доходностями $X$ и $Y$ на один вложенный рубль. Доходности некоррелированы, $\E(X)=0{,}09$, $\E(Y)=0{,}04$, $\sigma_X=0{,}5$, $\sigma_Y=0{,}1$. У инвестора есть 1~рубль. Инвестор подкидывает неправильную монетку, выпадающую орлом с вероятностью $p$. Если монетка выпадает орлом, он покупает первые акции, если решкой, то вторые. Обозначим за $\mu(p)$ и $\sigma(p)$ ожидаемую доходность и стандартное отклонение доходности полученной стратегии.
\begin{enumerate}
\item Постройте множество возможных $\mu(p)$ и $\sigma(p)$.
\item Найдите функцию плотности доходности полученной стратегии.
\end{enumerate} }
\solution{}


\problem{ Известна функция плотности случайной величины,  $p_{X}(t)=c\cdot \exp (-8t^{2} +5t)$. Найдите $E(X)$,  $\sigma _{X} $. }
\solution{ выделяем полный квадрат, $\E(X)=\frac{5}{16}$, $\sigma_{X}=\frac{1}{4}$  }


\problem{Случайные величины $X$ и $Y$ имеют стандартное нормальное распределение. Закон распределения величины $Z$ неизвестен. Все три случайных величины независимы. Рассмотрим случайную величину $R=(X+ZY)/\sqrt{1+Z^2}$.
\begin{enumerate}
\item Какое условное распределение имеет величина $R$ при известной величине $Z$?
\item Как распределена величина $R$?
\end{enumerate} }
\solution{$N(0,1)$, $N(0,1)$}

\subsection{ЦПТ}
\problem{
Вася играет в компьютерную игру "--- <<стрелялку"=бродилку>>. По
сюжету ему нужно убить 60 монстров. На один выстрел уходит ровно 1
минута. Вероятность убить монстра с одного выстрела равна 0{,}25.
Количество выстрелов не ограничено.
\begin{enumerate}
\item Сколько времени в среднем Вася тратит на одного монстра?
\item  Найдите дисперсию этого времени.
\item  Какова вероятность того, что Вася закончит игру меньше, чем за
3 часа?
\end{enumerate}
 }
\solution{ }


\subsection{Многомерное нормальное распределение}

\problem{
Ермолай Лопахин решил приступить к вырубке вишневого сада. Однако выяснилось, что растут в нём не только вишни, но и яблони. Причём, по словам Любови Андреевны Раневской, среднее количество деревьев (а они периодически погибают от холода или жары, либо из семян вырастают новые) в саду распределено в соответствии с нормальным законом ($X$ "--- число яблонь, $Y$ "--- число вишен) со следующими параметрами:
\begin{equation}
\begin{pmatrix}	X \\ 	Y 	\end{pmatrix}
\sim \mN
\left(
\begin{pmatrix}
25 \\ 125
\end{pmatrix}
;
\begin{pmatrix}
	5 & 4 \\
	4 & 10
	\end{pmatrix}
\right)
\end{equation}

Найдите вероятность того, что Ермолаю Лопахину придется вырубить более 150~деревьев.
Каково ожидаемое число подлежащих вырубке вишен, если известно, что предприимчивый и последовательный Лопахин, не затронув ни одного вишнёвого дерева, начал очистку сада с яблонь и все 35~яблонь уже вырубил?

Автор: Кирилл Фурманов, Ира Чернухина}
\solution{ }


\problem{
В самолете пассажирам предлагают на выбор <<мясо>> или <<курицу>>. В самолет 250 мест. Каждый пассажир с вероятностью 0.6 выбирает курицу, и с вероятностью 0.4 - мясо. Сколько порций курицы и мяса нужно взять, чтобы с вероятностью 99\% каждый пассажир получил предпочитаемое блюдо, а стоимость <<мяса>> и <<курицы>> для компании одинаковая? \\
Как изменится ответ, если компания берет на борт одинаковое количество <<мяса>> и <<курицы>>? }
\solution{ 
$K=170$, $M=120$ (симметричный интервал) или $K=M=168$ (площадь с одного края можно принять за 0) \\
Вариант: театр, два входа, два гардероба а) только пары, б) по одному }


\problem{
Сэр Фрэнсис Гальтон ---  учёный XIX-XX веков, один из основоположников как
генетики, так и статистики --- изучал, среди всего прочего, связь между ростом детей и родителей.  Он исследовал данные о росте 928 индивидов. Обозначим $X_1$ --- рост случайного человека, а $X_2$ --- среднее арифметическое роста его отца и матери. По результатам исследования Гальтона:


\[
\begin{pmatrix}
X_1 \\
X_2
\end{pmatrix}
\sim
N\left[
\begin{pmatrix}
68.1 \\
68.3
\end{pmatrix};
\begin{pmatrix}
6.3 & 2.1 \\
2.1 & 3.2
\end{pmatrix}
\right]
\]

\begin{enumerate}
\item Обратите внимание на то, что дисперсия роста детей выше дисперсии среднего роста
родителей. С чем это
может быть связано? Учтите, что рост детей измерялся уже по достижении
зрелости, так что разброс не должен быть связан с возрастными различиями.
\item Рассчитайте корреляцию между $X_1$ и $X_2$
\item Один дюйм примерно равен $2.54$ сантиметра. Пусть $X_1'$ и $X_2'$ --- это те же $X_1$ и $X_2$, только измеренные в сантиметрах. Найдите вектор математических ожиданий и ковариационную матрицу вектора $X'=(X_1', X_2')$.
\item Определите, каков ожидаемый рост и дисперсия роста человека, средний рост родителей которого составляет 72 дюйма?
\item Найдите вероятность того, что рост человека превысит 68 дюймов, если средний рост его родителей равен 72 дюймам. Подсказка: используйте предыдущий пункт и нормальность распределения!
\end{enumerate} }

\solution{}

\subsection{Распределения связанные с нормальным}

% задачи без статистики на свойства t, F, chi распределений


\problem{Сравните $\E(F_{k,n})$ и $Var(t_n)$.}
\solution{$\E(F_{k,n})=Var(t_n)$}


\problem{Пусть $X\sim t_{n}$. Как распределена величина $Y=X^{2}$? }
\solution{ $F_{1,n}$ }


\problem{  На плоскости выбирается точка со случайными координатами. Абсцисса
и ордината независимы и распределены $N(0;1)$. Какова вероятность
того, что расстояние от точки до начала координат будет больше
2,45? }
\solution{ Квадрат расстояния имеет $\chi^2_2$ распределение  }


