% !Mode:: "TeX:UTF-8"
\section{Нормальное распределение и ЦПТ}
\subsection{Одномерное нормальное распределение}

\problem{Где находятся точки перегиба функции плотности для нормального распределения?}
\solution{$ \mu \pm \sigma $.}

\problem{Пусть $X\sim \mN(\mu;\sigma^{2})$ и $t>\mu$. В какой точке функция $\PP(X\in [t;t+dt])$ убывает быстрее всего?}
\solution{$ \mu + \sigma $.}

\problem{Имеются две акции с доходностями $X$ и $Y$ на один вложенный рубль. Доходности некоррелированы, $\E(X)=0{,}09$, $\E(Y)=0{,}04$, $\sigma_X=0{,}5$, $\sigma_Y=0{,}1$. У инвестора есть 1~рубль. Он покупает на $a\in (0;1)$ рубля первых акций и на $(1-a)$ вторых акций. Обозначим за $\mu(a)$ и $\sigma(a)$ ожидаемую доходность и стандартное отклонение доходности полученного портфеля.
\begin{enumerate}
\item Постройте множество возможных $\mu(a)$ и $\sigma(a)$
\item Найдите наименее рисковый портфель. Чему равна его ожидаемая доходность?
\end{enumerate} }
\solution{}

\problem{Имеются две акции с доходностями $X$ и $Y$ на один вложенный рубль. Доходности некоррелированы, $\E(X)=0{,}09$, $\E(Y)=0{,}04$, $\sigma_X=0{,}5$, $\sigma_Y=0{,}1$. У инвестора есть 1~рубль. Инвестор подкидывает неправильную монетку, выпадающую орлом с вероятностью $p$. Если монетка выпадает орлом, он покупает первые акции, если решкой, то вторые. Обозначим за $\mu(p)$ и $\sigma(p)$ ожидаемую доходность и стандартное отклонение доходности полученной стратегии.
\begin{enumerate}
\item Постройте множество возможных $\mu(p)$ и $\sigma(p)$.
\item Найдите функцию плотности доходности полученной стратегии.
\end{enumerate} }
\solution{}

\subsection{ЦПТ}
\problem{
Вася играет в компьютерную игру "--- <<стрелялку"=бродилку>>. По
сюжету ему нужно убить 60 монстров. На один выстрел уходит ровно 1
минута. Вероятность убить монстра с одного выстрела равна 0{,}25.
Количество выстрелов не ограничено.
\begin{enumerate}
\item Сколько времени в среднем Вася тратит на одного монстра?
\item  Найдите дисперсию этого времени.
\item  Какова вероятность того, что Вася закончит игру меньше, чем за
3 часа?
\end{enumerate}
 }
\solution{ }


\subsection{Многомерное нормальное распределение}

\problem{
Ермолай Лопахин решил приступить к вырубке вишневого сада. Однако выяснилось, что растут в нём не только вишни, но и яблони. Причём, по словам Любови Андреевны Раневской, среднее количество деревьев (а они периодически погибают от холода или жары, либо из семян вырастают новые) в саду распределено в соответствии с нормальным законом ($X$ "--- число яблонь, $Y$ "--- число вишен) со следующими параметрами:
\begin{equation}
\begin{pmatrix}	X \\ 	Y 	\end{pmatrix}
\sim \mN
\left(
\begin{pmatrix}
25 \\ 125
\end{pmatrix}
;
\begin{pmatrix}
	5 & 4 \\
	4 & 10
	\end{pmatrix}
\right)
\end{equation}

Найдите вероятность того, что Ермолаю Лопахину придется вырубить более 150~деревьев.
Каково ожидаемое число подлежащих вырубке вишен, если известно, что предприимчивый и последовательный Лопахин, не затронув ни одного вишнёвого дерева, начал очистку сада с яблонь и все 35~яблонь уже вырубил?

Автор: Кирилл Фурманов, Ира ...}
\solution{ }

\subsection{Распределения связанные с нормальным}

% задачи без статистики на свойства t, F, chi распределений

\problem{Сравните $\E(F_{k,n})$ и $Var(t_n)$.}
\solution{$\E(F_{k,n})=Var(t_n)$}