% !Mode:: "TeX:UTF-8"
\section{Приёмы решения}
\subsection{Разложение в сумму}
\problem{ \label{sudba-don-juan-2}\zdt{Судьба Дон-Жуана-2} (см. тж. с.~\pageref{sudba-don-juan-1})

У Васи $n$  знакомых девушек (их всех зовут по-разному). Он пишет
им $n$  писем, но по рассеянности раскладывает их в конверты
наугад. Случайная величина $X$ обозначает количество девушек, получивших письма,
написанные лично для них. Найдите $\E(X)$, $\Var(X)$. }
\solution{$\E(X)=1$, $\Var(X)=1$. }



\problem{ \label{cube-cut-2}(см. тж. с.~\pageref{cube-cut-1}) \ENGs

A wooden cube that measures 3 cm along each edge is painted red. The painted cube is then cut into 27 pieces of 1-cm cubes. If I tossed all the small cubes in the air, so that they landed randomly on the table, how many cubes should I expect to land with a painted face up? \RUSs}
\solution{ $9$.}


\problem{Вокруг новогодней ёлки танцуют хороводом 27 детей. Мы считаем, что ребенок высокий, если он выше обоих своих соседей. Сколько высоких детей в среднем танцует вокруг елки? Вероятность совпадания роста будем считать равной нулю.}
\solution{Для трёх детей вероятность того, что тот, что посередине "--- самый высокий, равна $ \frac{1}{3} $, значит математическое ожидание равно $ \frac{27}{3}=9$. }

\problem{Маша собирает свою дамскую сумочку. Есть $n$ различных предметов, которые она туда может положить. Каждый предмет она кладёт независимо от других с вероятностью $p$.
\begin{enumerate}
\item Пусть $X$ "--- количество положенных предметов. Найдите $\E(X)$ и $\Var(X)$.
\item При каком $p$ вероятность положить в сумку любой заданный набор вещей не будет зависеть от конкретного набора?
\end{enumerate}}
\solution{ Биномиальное распределение, $\E(X)=np$, $\Var(X)=np(1-p) $. При $p=0{,}5$ все подмножества будут равновероятны.}


\subsection{Первый шаг}

\problem{Илье Муромцу предстоит дорога к камню. И от камня начинаются ещё три дороги. Каждая из тех дорог снова оканчивается камнем. И от каждого камня начинаются ещё три дороги. И каждые те три дороги оканчиваются камнем\ldotst{} И так далее до бесконечности. На каждой дороге можно встретить живущего на ней трёхголового Змея Горыныча с вероятностью (хм, вы не поверите!) одна третья. Какова вероятность того, что Илья Муромец пройдет свой бесконечный жизненный путь, так ни разу и не встретив Змея Горыныча?}
\solution{$p=\frac{2}{3}(1-(1-p)^{3})$, нам подходит решение $ p=\frac{3-\sqrt{3}}{2} $. }


\problem{У Пети "--- монетка, выпадающая орлом с вероятностью $ p\in (0;1) $. У Васи "--- с вероятностью $ q\in (0;1) $. Они одновременно подбрасывают свои монетки до тех пор, пока у них не окажется набранным одинаковое количество орлов. В частности, они останавливаются после первого подбрасывания, если оно дало одинаковые результаты. Сколько в среднем раз им придётся подбросить монетку?}
\solution{}

\problem{Сколько в среднем нужно взять из колоды в 52 карты, чтобы насобирать подряд 5 карт одной масти?
\begin{hint}
Ответ имеет вид произведения дробей очень простого вида.
\end{hint}
}
\solution{Если у нас $m=13$ достоинств и $n=4$ масти, то ответ имеет вид: $mn\prod\limits_{i=1}^{}\frac{in}{in+1}\approx 45{,}3$.}

\problem{Вася прыгает на один метр вперёд с вероятностью $p$ и на два метра вперёд с вероятностью $1-p$. Как только он пересечёт дистанцию в 100~метров, он останавливается. Получается, что он может остановиться на отметке либо в 100~метров, либо в 101~метр. Какова вероятность того, что он остановится ровно на отметке в 100~метров?}
% копия в задачах на остановку мартингала
\solution{ Обозначим за $P_n$ вероятность остановиться ровно на $n$ метрах. Мы ищем $P_{100}$.

\textit{Решение 1.} По методу первого шага:  $P_n=pP_{n-1}+(1-p)P_{n-2}$.

\textit{Решение 2.} Попасть ровно в $n$ можно двумя способами: перелетев $n-1$ или попав в $n-1$ и сделав шаг в один метр. Значит $P_n=(1-P_{n-1})+pP_{n-1}$.

\textit{Решение 3.} Обозначим Васину координату в момент времени $t$ как $X_t$. Можно найти $a$ так, чтобы процесс $Y_t=a^{X_t}$ был мартингалом. Момент остановки $T=\min\{t \min X_t\geq n\}$. Мартингал $Y_{t\wedge T}$ ограничен, теорема Дуба применима. $\E(Y_T)=\E(Y_0)=1$. Получаем уравнение $P_n a^{n}+(1-P_n) a^{n+1}=1$.}

% untyp
\problem{
Испытания по схеме Бернулли проводятся до первого успеха, вероятность успеха в
отдельном испытании равна $p$ \par
а) Чему равно ожидаемое количество испытаний?   \par
б) Чему равно ожидаемое количество неудач? \par
в) Чему равна дисперсия количества неудач? }
\solution{ $\frac{1}{p}$, $\frac{q}{p}$ \par
в) $E(X^{2})=p\cdot 1+q\cdot E((X+1)^{2})$, $Var(X)=\frac{q}{p^2}$ }

% untyp
\problem{ Отрицательное биномиальное \par
Испытания по схеме Бернулли проводятся до $r$-го успеха, вероятность успеха в
отдельном испытании равна $p$ \par
а) Чему равно ожидаемое количество неудач? \par
б) Чему равна дисперсия количества неудач? }
\solution{ (устно, при сделанной предыдущей задаче) $\frac{rq}{p}$, $Var(X)=\frac{rq}{p^2}$ }

% untyp
\problem{
Саша и Маша по очереди подбрасывают кубик. Посуду будет
мыть тот, кто первым выбросит шестерку. Маша бросает первой.
Какова вероятность того, что Маша будет мыть посуду? }
\solution{ }

% untyp
\problem{
Саша и Маша решили, что будут рожать нового ребенка, до тех
пор, пока в их семье не будут дети обоих полов. Каково ожидаемое
количество детей? }
\solution{ }

% untyp
\problem{
Четыре человека играют в игру <<белая ворона платит>>. Они
одновременно подкидывают монетки. Если три монетки выпали одной
стороной, а одна - по-другому, то <<белая ворона>> оплачивает всей
четверке ужин в ресторане. Если <<белая ворона>> не определилась,
то монетки подбрасывают снова. Сколько в среднем нужно
подбрасывания для определения <<белой вороны>>? }
\solution{ }

% untyp
\problem{
Саша и Маша каждую неделю ходят в кино. Саша доволен
фильмом с
вероятностью 1/4, Маша - с вероятностью 1/3. \par
a) Сколько недель в среднем пройдет до тех пор, пока кто-то не
будет доволен? \par
b) Какова вероятность того, что первым будет доволен Саша? \par
c) Сколько недель в среднем пройдет до тех пор, пока каждый не
будет доволен хотя бы одним просмотренным фильмом? }
\solution{ }

% untyp
\problem{
По ответу студента на вопрос преподаватель может сделать
один из трех выводов: ставить зачет, ставить незачет, задать еще
один вопрос. Допустим, что знания студента и характер
преподавателя таковы, что при ответе на отдельный вопрос зачет
получается с вероятностью $p_{1}=3/8$, незачет - с вероятностью
$p_{2}=1/8$. Преподаватель задает вопросы до тех пор, пока не
определится
оценка. \par
а) Сколько вопросов в среднем будет задано? \par
б) Какова вероятность получения зачета? }
\solution{ }

% untyp
\problem{
Вы играете в следующую игру. Кубик подкидывается неограниченное число раз. Если на кубике выпадает 1, 2 или 3, то соответствующее количество монет добавляется на кон. Если выпадает 4 или 5, то игра оканчивается и Вы получаете сумму, лежащую на кону. Если выпадает 6, то игра оканчивается, а Вы не получаете ничего. \par
а) Чему равен ожидаемый выигрыш в эту игру? \par
б) Изменим условие: если выпадает 5, то набранная сумма сгорает, а игра начинается заново. Чему будет равен ожидаемый выигрыш? }
\solution{ 
a) $V(x)=\frac{1}{6}(V(x+1)+V(x+2)+V(x+3)+2x+0)$ \par
Ищем линейную $V(x)$, получаем $V(x)=\frac{2}{3}x+\frac{4}{3}$ \par
б) $V(x)=\frac{1}{6}(V(x+1)+V(x+2)+V(x+3)+x+V(0)+0)$ }

% untyp
\problem{ Вася подкидывает кубик. Если выпадает единица, или Вася говорит
<<стоп>>, то игра оканчивается, если нет, то начинается заново.
Васин выигрыш - последнее выпавшее число. Как выглядит оптимальная
стратегия? Как выглядит оптимальная стратегия, если за каждое
подбрасывание Вася платит 35 копеек?\cite{stirzaker:otep}}
\solution{ }

% untyp
\problem{
Саша и Маша подкидывают монетку бесконечное количество раз. Если сначала появится
РОРО, то выигрывает Саша, если сначала появится ОРОО, то - Маша. \par
а) У кого какие шансы выиграть? \par
b) Сколько в среднем времени ждать до появления РОРО? До ОРОО?
с) Сколько в среднем времени ждать до определения победителя? }
\solution{ }

% untyp
\problem{ \label{mishka ishet sir}
Есть три комнаты. В первой из них лежит сыр. Если мышка
попадает в первую комнату, то она находит сыр через одну минуту.
Если мышка попадает во вторую комнату, то она ищет сыр две минуты
и покидает комнату. Если мышка попадает в третью комнату, то она
ищет сыр три минуты и покидает комнату. Покинув комнату, мышка
выходит в коридор и выбирает новую комнату наугад (т.е. может
зайти в одну и ту же). Сейчас мышка в коридоре. Сколько времени ей
в среднем потребуется, чтобы найти сыр? }
\solution{ $m=\frac{1}{3}+\frac{1}{3}(2+m)+\frac{1}{3}(3+m)$, $m=6$ }

% untyp
\problem{
Иська и Еська по очереди подбрасывают два кубика. Иська
бросает первым. Иська выигрывает, если при своем броске получит 6
очков в сумме на двух кубиках. Еська выигрывает, если при своем
броске получит 7 очков в сумме на двух кубиков. Кубики
подбрасываются до
тех пор, пока не определится победитель. \par
а) Верно ли, что события $A=\{$на двух кубиках в сумме выпало
больше 5 очков$\}$ и $B=\{$на одном из кубиков выпала 1$\}$ являются независимыми? \par
б) Какова вероятность того, что Еська выиграет? }
\solution{ }

% untyp
\problem{
Players A and B play a (fair) dice game. <<A>> deposits one coin and
they take turns rolling a single dice, <<B>> rolling first. \par
If <<B>> rolls an even number, he collects a coin from the pot. If
he rolls an odd number, he put a coin (coins with same values
always). If <<A>> (plays and) rolls an even number, he collects a
coin but if he rolls an odd number, he does NOT add a coin. The
game continues until the pot is exhausted. \par
Question: what is the probability that <<A>> wins this game (that
is, exhaust the pot) ? \par
t=138358}
\solution{ }



\subsection{\textit{o}-малое}

\problem{Случайные величины $ X_{1} ,\ldots, X_{n} $ одинаково распределены с функцией плотности $ p(t) $ и независимы. Найдите функцию плотности третьего по величине $ X_{(3)}$.}
\solution{$ \PP(X_{(3)}\in [x;x+dx])= C_{n}^{2}C_{n-2}^{1} \bigl((F(x)+o(x)\bigr)^{n-3}\bigl(1-F(x)+o(x)\bigr)^{2}\bigl((f(x)+o(x)\bigr)dx $. Значит, искомая функция плотности равна $f_{X_{(3)}}(t)=f(t)F(t)^{n-3}\bigl(1-F(t)\bigr)^{2}$.}

\subsection{Вероятностный метод}
% задачи не по теории вероятностей, которые решаются с помощью теории вероятностей

% untyp
\problem{ На потоке 200 студентов. На контрольной было 6 задач. Известно, что каждую задачу решило не менее 120 человек.
Всегда ли преподаватель может выбрать двух студентов из потока так, что эти двое могут решить всю контрольную совместными усилиями?}
\solution{Выберем двух студентов из потока наугад. Вероятность того, что ни один из них не решил задачу \No\,1, не превосходит $\br{\frac{80}{200}}^2=0{,}16$.
Вероятность того, что ни один из них не решил задачу \No\,2, не превосходит 0{,}16 (по тем же причинам), и это справедливо для каждой из шести задач. Вероятность того, что хотя
бы одну задачу они на пару не решили, не превосходит суммы этих вероятностей, т.\,е. $0{,}16\cdot6=0{,}96$.
Значит, вероятность выбора пары студентов, которые совместными усилиями могут решить экзамен, не менее $0{,}04$. Значит, хотя бы одна такая пара существует.}


\subsection{Склеивание отрезка}


\problem{Машина может сломаться равновероятно в любой точке на дороге от города А до города Б. Когда машина сломается мы будем толкать ее до ближайшего сервиса. Где должны быть расположены три автосервиса чтобы минимизировать ожидаемую продолжительность толкания? А если автосервисов будет $n$?}
\solution{Проверить. Разбиваем отрезок на $n$ частей, ставим автосервис в центр каждой части
\url{http://math.stackexchange.com/questions/37254/} }

