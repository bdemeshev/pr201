% !Mode:: "TeX:UTF-8"
\section{Условные вероятности и ожидания. Дополнительная информация}
%Правило умножения вероятностей:
%Если A B независимы, то

\subsection{Условная вероятность}

\problem{ \ENGs
A bag contains a counter, known to be either white or black. A white counter is put in, the bag is shaken, and a counter is drawn out, which proves to be white. What is now the chance of drawing a white counter? \RUSs}
\solution{ }

\problem{ \ENGs
You have a hat in which there are three pancakes: one is golden on both sides, one is brown on both sides, and one is golden on one side and brown on the other. You withdraw one pancake, look at one side, and see that it is brown. What is the probability that the other side is brown? \RUSs}
\solution{ }

\problem{ \ENGs
The inhabitants of an island tell truth one third of the time. They lie with the probability of $\frac{2}{3}$. On an occasion, after one of them made a statement, another fellow stepped forward and declared the statement true. What is the probability that it was indeed true? \RUSs }
\solution{ }


\problem{
На кубиках написаны числа от 1 до 100. Кубики свалены в кучу. Вася выбирает наугад из кучи по очереди три кубика.
\begin{enumerate}
\item Какова вероятность, что полученные три числа будут идти в возрастающем порядке?
\item Какова вероятность, что полученные три числа будут идти в возрастающем порядке, если известно, что первое меньше последнего?
\end{enumerate}
 }
\solution{ $\frac{1}{6}$; $\frac{1}{3}$. }


\problem{ (дописать)

Наследование группы крови контролируется аутосомным геном. Три его аллеля обозначаются буквами А, В и 0. Аллели А и В доминантны в одинаковой степени, а аллель 0 рецессивен по отношению к ним обоим. Поэтому существует четыре группы крови. Им соответствуют следующие генотипы:
\begin{itemize}
\item Первая (I) "--- 00;
\item Вторая (II) "--- АА, А0;
\item Третья (III) "--- ВВ, В0;
\item Четвёртая (IV) "--- АВ.
\end{itemize}

Наследование резус-фактора кодируется тремя парами генов и происходит независимо от наследования группы крови. Наиболее значимый ген имеет два аллеля, аллель D доминантный, аллель d рецессивный. Таким образом, получаем следующие генотипы:
\begin{itemize}
\item Резус-положительный "--- DD, Dd;
\item Резус-отрицательный "--- dd.
\end{itemize}

Если у беременной женщины резус"=отрицательная кровь, а у плода резус"=положительная, то есть риск возникновения гемолитической болезни (у матери образуются антитела к резус фактору, безвредные для неё, но вызывающие разрушение эритроцитов плода).

Перед нами два семейства: Монтекки и Капулетти. \\
...}
\solution{ }


\problem{ \label{rekordnaia volna}
Пусть $X_{i}$ "--- НОРСВ, такие, что $\PP(X_{i}=X_{j})=0$. Обозначим за
$E_{k}$ событие, состоящее в том, что $X_{k}$ оказалась
<<рекордом>>, т.\,е. больше, чем все предыдущие $X_{i}$ ($i<k$). Для
определённости будем считать, что $E_{1}=\Omega$.
\begin{enumerate}
\item Найдите $\PP(E_{k})$.
\item  Верно ли, что $E_{k}$ независимы?
\item  Какова вероятность того, что второй рекорд будет зафиксирован в $n$-й момент времени?
\item  Сколько в среднем времени пройдёт до второго рекорда?
\end{enumerate}

\begin{ist}
Williams, 4.3.
\end{ist}
 }
\solution{ Какая-то из первых $k$ величин будет наибольшей. В силу \iid{}
получаем, что $\PP(E_{k})=\frac{1}{k}$. $E_k$ независимы: например, если известно,
что 10-е наблюдение было рекордом, это ничего не говорит о рекордах в первых 9-ти
наблюдениях. Вероятность второго рекорда в $n$-й момент равна $\frac{1}{n(n-1)}$,
а в среднем времени до второго рекорда пройдёт $\infty$. }

\problem{Известно, что $\PP(A \mid B)=\PP(A \mid B^{c})$. Верно ли, что $A$ и $B$ независимы?}
\solution{Да.}


\problem{ \zdt{Randomized response technique}

В анкету для чиновников включён скользкий вопрос: <<Берёте ли Вы
взятки?>>. Чтобы стимулировать чиновников отвечать правдиво,
используется следующий прием. Перед ответом на вопрос чиновник втайне от анкетирующего подкидывает специальную монетку, на гранях
которой написано <<правда>>, <<ложь>>. Если монетка выпадает
<<правдой>>, то предлагается отвечать на вопрос правдиво, если
монетка выпадает на <<ложь>>, то предлагается солгать. Таким
образом, ответ <<да>> не обязательно означает, что чиновник берёт
взятки.

Допустим, что треть чиновников берёт взятки, а монетка
неправильная и выпадает <<правдой>> с вероятностью 0{,}2.
\begin{enumerate}
\item Какова вероятность того, что чиновник ответит <<да>>?
\item  Какова вероятность того, что чиновник берёт взятки, если он
ответил <<да>>? Если ответил <<нет>>?
\end{enumerate}
\todo[inline]{Вставить построение несмещённой оценки?}
}
\solution{ }

\problem{
Пусть события  $A$  и  $B$  независимы и $\PP(B)>0$.
Чему равна  $\PP(A \mid B)$? }
\solution{ $ \PP(A \mid B)=\PP(A)$. }

\problem{
Из колоды в 52 карты извлекается одна карта наугад. Верно ли, что
события <<извлечён туз>> и <<извлечена пика>> являются
независимыми? }
\solution{ Да. }

\problem{
Из колоды в 52 карты извлекаются по очереди две карты наугад.
Верно ли, что события <<первая карта "--- туз>> и <<вторая карта "---
туз>> являются независимыми? }
\solution{ Нет. }

\problem{
Известно, что $\PP(A)=0{,}3$, $\PP(B)=0,{4}$, $\PP(C)=0{,}5$. События
$A$ и $B$ несовместны, события $A$ и $C$ независимы и
$\PP(B\mid C)=0{,}1$.
Найдите $\PP(A\cup B\cup C)$. }
\solution{ }

\problem{
Имеется три монетки. Две <<правильных>> и одна "--- с орлами по
обеим сторонам. Петя выбирает одну монетку наугад и подкидывает её
два раза. Оба раза выпадает орёл. Какова вероятность того, что
монетка <<неправильная>>? }
\solution{ }

\problem{
Самолёт упал либо в горах, либо на равнине. Вероятность того, что самолёт упал в горах, равна 0{,}75. Для поиска пропавшего самолёта выделено 10 вертолётов. Каждый вертолёт можно использовать только в одном месте. Как распределить имеющиеся вертолёты, если вероятность обнаружения пропавшего самолёта отдельно взятым вертолётом равна: $0{,}95$? $0,6$ (пасмурно)? $0{,}1$ (туман)? }
\begin{ist}
Айвазян, экзамен РЭШ.
\end{ist}
\solution{ }

\problem{
Предположим, что социологическим опросам доверяют 70\,\% жителей. Те, кто доверяет опросам, всегда отвечают искренне; те, кто не доверяет, отвечают наугад, равновероятно выбирая <<да>> или <<нет>>. Социолог Петя  в анкету очередного опроса включил вопрос: <<Доверяете ли Вы социологическим опросам?>>
\begin{enumerate}
\item Какова вероятность, что случайно выбранный респондент ответит <<Да>>?
\item  Какова вероятность того, что он действительно доверяет, если известно, что он ответил <<Да>>?
\end{enumerate}
 }
\solution{ }

\problem{
Два охотника выстрелили в одну утку. Первый попадает с
вероятностью 0{,}4, второй "--- с вероятностью 0{,}6. В утку попала ровно
одна пуля. Какова вероятность того, что утка была убита первым
охотником? }
\solution{
$p=\frac{0{,}4\cdot 0{,}4}{0{,}4\cdot 0{,}4+0{,}6\cdot 0{,}6}=\frac{4}{13}$.}

\problem{
С вероятностью 0{,}3 Вася оставил конспект в одной из 10
посещённых им сегодня аудиторий. Вася осмотрел 7 из 10 аудиторий и
конспекта в них не нашёл.
\begin{enumerate}
\item  Какова вероятность того, что конспект будет найден в следующей
осматриваемой им аудитории?
\item  Какова (условная) вероятность того, что конспект оставлен
где-то в другом месте?
\end{enumerate}
 }
\solution{ }

\problem{
Вася гоняет на мотоцикле по единичной окружности с центром в
начале координат. В случайный момент времени он останавливается.
Пусть случайные величины  $X$  и  $Y$  "--- это Васины абсцисса и
ордината в момент остановки. Найдите  $\PP\left(X>\frac{1}{2} \right)$,
$\PP\left(X>\frac{1}{2} \bigm| Y<\frac{1}{2} \right)$. Являются ли события
$A=\left\{X>\frac{1}{2} \right\}$  и
$B=\left\{Y<\frac{1}{2} \right\}$  независимыми?
\begin{hint}
$\cos\left(\frac{\pi }{3} \right)=\frac{1}{2}$, длина окружности $l=2\pi r$.
\end{hint}
}
\solution{ }

\problem{
Пусть  $\PP(A)=1/4$,  $\PP(A \mid B)=\frac{1}{2}$  и $\PP(B\mid A)=\frac{1}{3}$. Найдите $\PP(A\cap
B)$, $\PP(B)$  и  $\PP(A\cup B)$.}
\solution{ }

\problem{
Примерно\footnote{Цифры условные. Celui qui ne mange pas de
bifsteak au cause de la vache folle --- il est fou! Jolivaldt.} 4\,\%
коров заражены <<коровьим бешенством>>. Имеется тест, позволяющий
с определённой степенью достоверности установить, заражено ли мясо
прионом или нет. С вероятностью $0{,}9$ заражённое мясо будет признано
заражённым. <<Чистое>> мясо будет признано заражённым с
вероятностью 0{,}1. Судя по тесту, эта партия мяса заражена. Какова
вероятность того, что она действительно заражена?}
\solution{ }

\problem{
\emph{Роме Протасевичу, искавшему со мной у Мутновского
вулкана в
тумане серую палатку...}

Есть две тёмные комнаты, $A$ и $B$. В одной из них сидит чёрная кошка.
Первоначально предполагается, что вероятность нахождения кошки в
комнате $A$ равна $\alpha$. Вероятность найти чёрную кошку в темной
комнате (если она там есть) с одной попытки равна $p$.  Допустим,
что вы сделали $a$ неудачных попыток поиска кошки в комнате $A$ и
$b$ неудачных попыток в комнате $B$.
\begin{enumerate}
\item Чему равна условная вероятность нахождения кошки в комнате $A$?
\item  При каком условии на $(a-b)$ эта вероятность будет больше
$0{,}5$?
\end{enumerate}
 }
\solution{ }

\problem{
Кубик подбрасывается два раза. Найдите вероятность
получить сумму, равную 8, если на первом кубике выпало 3.}
\solution{ $\frac{1}{6}$. }

\problem{
В коробке 10 пронумерованных монеток, $i$-я монетка выпадает
орлом с вероятностью $\frac{i}{10}$. Из коробки была вытащена одна
монетка наугад. Она выпала орлом. Какова вероятность того, что это
была пятая монетка? }
\solution{
$\frac{1}{11}$.  }

\problem{ Вы играете две партии в шахматы против незнакомца. Равновероятно
незнакомец может оказаться новичком, любителем или профессионалом.
Вероятности вашего выигрыша в отдельной партии, соответственно,
будут равны 0{,}9; 0{,}5; 0{,}3.
\begin{enumerate}
\item Какова вероятность выиграть первую партию?
\item Какова вероятность выиграть вторую партию, если вы выиграли
первую?
\end{enumerate}
 }

\solution{ $p_{a}=\frac{1}{3}(0{,}9+0{,}5+0{,}3)=\frac{17}{30}$, $p_{b}=\frac{1}{3}(0{,}9^{2}+0{,}5^{2}+0{,}3^{2})/p_{a}=\frac{115}{170}$. }

\problem{
В каких из перечисленных случаев вероятность наличия флэша (см. \hyperref[combo]{карточные комбинации} на стр.~\pageref{combo}) больше, чем при полном отсутствии информации:
\begin{enumerate}
\item Первая карта из имеющихся "--- это туз;
\item Первая карта из имеющихся "--- это туз бубей;
\item На руках имеется хотя бы один туз;
\item На руках имеется туз бубей.
\end{enumerate}
 }

\solution{ \ENGs Unverified, but no calculation:

An arbitrary hand can have two aces but a flush hand can't.  The
average number of aces that appear in flush hands is the same as the
average number of aces in arbitrary hands, but the aces are spread out
more evenly for the flush hands, so set (3) contains a higher fraction
of flushes.

Aces of spades, on the other hand, are spread out the same way over
possible hands as they are over flush hands, since there is only one of
them in the deck.  Whether or not a hand is flush is based solely on a
comparison between different cards in the hand, so looking at just one
card is necessarily uninformative.  So the other sets contain the same
fraction of flushes as the set of all possible hands. \RUSs }


\problem{\ENGs A man has 3 equally favorite seats to fish at. The probability with which the man can succeed at catching at each seat is 0.6, 0.7, 0.8 respectively. It is known that the man dropped the hint at one seat three times and just caught one fish. Find the probability that the fish was caught at the first seat. \RUSs}
\solution{$\approx 0{,}503$. }


\subsection{Условное среднее}


\problem{ \label{dve shkatulki} \zdt{Две шкатулки}

Васе предлагают две шкатулки и обещают, что в одной из них денег
в два раз больше, чем в другой. Вася открывает наугад одну из них
"--- в ней $a$ рублей. Вася может взять либо деньги, либо
оставшуюся шкатулку.
\begin{enumerate}
\item Правильно ли Вася считает, что ожидаемое количество денег в
неоткрытой шкатулке равно $\frac{1}{2}\left( {\frac{1} {2}a}
\right)+\frac{1}{2}( {2a} ) = 1\frac{1} {4}a$ и что
поэтому нужно изменить свой выбор?
\item Пусть известно, что в пару шкатулок кладут $3^k$ и $3^{k+1}$
рублей с вероятностью $p_k  = ( {\frac{1} {2}} )^k $. Стоит ли
Васе изменить свой выбор после того, как он открыл
первую шкатулку? Почему?
\end{enumerate}
 }
\solution{Вася считает неправильно: условное распределение суммы можно определить, только зная безусловное.

Концепция условного ожидания неприменима? Вставить это в иллюстрацию условного ожидания? При заданном безусловном распределении Васе следует сменить
свой выбор вне зависимости от того, что он увидел в первой шкатулке. Вторая открытая лучше первой открытой. Это возможно
из-за того, что безусловная ожидаемая сумма равна бесконечности
для обеих шкатулок.  }


\problem{В кабинет бюрократа скопилась очередь ещё до его открытия. Пусть время обслуживания страждущих "--- независимые экспоненциальные случайные величины. Посетитель, пришедший через $t$ минут после открытия, узнал, что первый посетитель уже ушёл, а второй ещё сидит в кабинете. Найдите ожидаемое время обслуживания первого посетителя, $\E(X_{1} \mid X_{1}\le t < X_{1}+X_{2}) $.}
\solution{$\frac{t}{2}$.}
\cat{poisson} \cat{exp} % может перекинуть в Пуассоновский процесс?


\problem{
Пете и Васе предложили одну и ту же задачу. Они могут правильно решить её с вероятностями 0{,}7 и 0{,}8 соответственно. К задаче предлагается 5 ответов на выбор, поэтому будем считать, что выбор каждого из пяти ответов равновероятен, если задача решена неправильно.
\begin{enumerate}
\item Какова вероятность несовпадения ответов Пети и Васи?
\item  Какова вероятность того, что Петя ошибся, если ответы совпали?
\item  Каково ожидаемое количество правильных решений, если ответы совпали?
\end{enumerate}
 }
\solution{ }

\problem{Автобусы ходят регулярно с интервалом в 10~минут. Вася приходит на остановку в случайный момент времени и ждёт автобуса не больше $a$ минут. Величина $a$ "--- константа из интервала $(0;10)$. Если автобус приходит меньше чем за $a$ минут, то Вася уезжает на нём. Если автобуса нет в течение $a$ минут, то Вася заходит в ближайшую кафешку перекусить и через случайное время возвращается на остановку. На второй раз он ждёт до прихода автобуса.
\begin{enumerate}
\item Какое время Вася в среднем проводит в ожидании автобуса?
\item  Постройте график получившейся функции от $a$.
\end{enumerate}
}
\solution{$f(a)=5+\frac{a(10-a)}{20}$.}


