%%%% версия с ОКРУЖЕНИЯМИ вместо комманд

% специальная штука под задачник
% создает команды:
% \problem{ текст задачи }
% \solution{ текст решения }
% \problemonly  - после этой команды будут печататься только \problem{} и \problemtext{}
% \solutiononly - после этой команды будут печататься только \solution{} и \solutiontext{}
% \problemandsolution - после этой команды печатается все
% \secsolution - задает новую (виртуальную) секцию для решений

% может потребоваться %\addtocounter{secsolution}{число глав без задач решений, не прогнанных через problemonly}

% как работать
% файл с решениями отдельной главы должен выглядеть так:
% \problem{ dddd} \solution{ddddddd}
% \problem{df sldk} \solution{ dfssd}

% главный файл может выглядеть двумя способами:

% Способ 1. (для решений контрольной, рядом задачи и ответы)
% \problemandsolution
% \input{file with problems}

% Способ 2. (для задачника, сначала все задачи, затем все ответы)
% \problemonly
% \input{file with problems}

% \solutiononly
% \input{file with problems}

% ААААААААААААААААААААААА надо делать!!!!
% Способ 3. - основной (вариант способа 2)

% \problemonly2
% \input{file with problems}

%\solutiononly - эта команда сама сделает все!




% файл с задачами:
% \section{Первая}
% \problem{ dddd} \solution{ddddddd}
% \problem{df sldk} \solution{ dfssd}
% \problemtext{Этот текст не будет напечатан после solutiononly}
% \section{Вторая}
% \problem{ ааа} \solution{dыва}
% \problem{ыавыв} \solution{ ыва}
% \solutiontext{Этот текст не будет напечатан после problemonly}



% начало кода:

\let\oldchapter\chapter % сохраняем команду \chapter, т.к. мы ее переопределим
\let\oldsection\section % сохраняем команду \section, т.к. мы ее переопределим

\newcommand{\restorechapter}{ % команда для восстановления \section \subsection
\renewcommand{\section}[1]{\oldsection{##1}}
\renewcommand{\chapter}[1]{\oldchapter{##1}}
}

\newcounter{problem}[chapter]
%создаем новый счетчик "problem", 
% будет автоматом сбрасываться на 0 при старте нового раздела
% при создании счетчик сам встанет на 0

\newcounter{secsolution}
% - это номер секции решаемой задачи (поскольку решения идут в одной секции, то номер секции надо менять в ручную)
\newcounter{solution}[secsolution]
% - это номер решаемой задачи, сам сбрасывается при увеличении secsolution на 1



\renewcommand{\thesecsolution}{\arabic{secsolution}}
% команда \thesecsolution просто выводит номер secsolution

\newcommand{\newsecsolution}{
\stepcounter{secsolution} % без создания ссылки увеличит secsolution на 1 со сбросом подчиненного счетчика
}
% команда \newsecsolution увеличит номер секции на 1 и установит номер решения внутри секции равным 0

\renewcommand{\theproblem}{\thechapter.\arabic{problem}.}
\renewcommand{\thesolution}{\thesecsolution.\arabic{solution}.}
% обновляем команду \theproblem - она должна выводить номер секции и номер задачи внутри секции
% почему обновляем? - потому, что она создалась при создании счетчика problem

% \newcommand{\problem}[1]{}
\newenvironment{problem}{%
  \catcode`\{=12
  \catcode`\}=12
  \DoNullproblem
}{%
  \ignorespacesafterend
}


\begingroup
  \catcode`\(=1
  \catcode`\)=2
  \catcode`\{=12
  \catcode`\}=12
  \def\x(%
    \endgroup
    \newcommand\DoNullproblem()
    \long\def\DoNullproblem##1\end{problem}(\end(problem))
  )
\x




%\newcommand{\solution}[1]{}
\newenvironment{solution}{%
  \catcode`\{=12
  \catcode`\}=12
  \DoNullsolution
}{%
  \ignorespacesafterend
}


\begingroup
  \catcode`\(=1
  \catcode`\)=2
  \catcode`\{=12
  \catcode`\}=12
  \def\x(%
    \endgroup
    \newcommand\DoNullsolution()
    \long\def\DoNullsolution##1\end{solution}(\end(solution))
  )
\x

% создаем команды \problem, \solution с одним аргументом, которые ничего не делает
% ниже они будут переопределены



\newenvironment{problemtext}{%
  \catcode`\{=12
  \catcode`\}=12
  \DoNullproblemtext
}{%
  \ignorespacesafterend
}


\begingroup
  \catcode`\(=1
  \catcode`\)=2
  \catcode`\{=12
  \catcode`\}=12
  \def\x(%
    \endgroup
    \newcommand\DoNullproblemtext()
    \long\def\DoNullproblemtext##1\end{problemtext}(\end(problemtext))
  )
\x


\newenvironment{solutiontext}{%
  \catcode`\{=12
  \catcode`\}=12
  \DoNullsolutiontext
}{%
  \ignorespacesafterend
}


\begingroup
  \catcode`\(=1
  \catcode`\)=2
  \catcode`\{=12
  \catcode`\}=12
  \def\x(%
    \endgroup
    \newcommand\DoNullsolutiontext()
    \long\def\DoNullsolutiontext##1\end{solutiontext}(\end(solutiontext))
  )
\x

% эти две команды будут выводить текст, заложенный внутри них только внутри соответствующей секции, в другой - ничего не будет делать
% в отличие от этой команды \problem \solution делают ссылки, слово "задача" и пр.


\newcommand{\problemonly}{
% эта команда переопределяет команду \problem и другие. Нам нужно печатать только условия.

% ставим на 0 все счетчики
\setcounter{problem}{0}
\setcounter{solution}{0}
\setcounter{secsolution}{0}


% команда \problemtext печатает то, что ей передают в качестве аргумента
% команда \solutiontext игнорируется
\renewenvironment{problemtext}{}{}

\renewenvironment{solutiontext}{%
  \catcode`\{=12
  \catcode`\}=12
  \DoNullsolutiontext
}{%
  \ignorespacesafterend
}

% теперь команда \section помимо своих стандартных функций еще и увеличит secsolution на 1.
\renewcommand{\chapter}[1]{\oldchapter{##1}\newsecsolution}
\renewcommand{\section}[1]{\oldsection{##1}}
% \subsection - стандартная


\renewenvironment{problem}{%
\refstepcounter{problem}
% \phantomsection % создаем точку привязки для команды \label % не нужна, т.к. есть refstepcounter
\textbf{Задача} 
\hyperref[s\theproblem]{\theproblem} % гиперссылка на метку "s1.1."
\label{p\theproblem} % метка "p1.1." 
\par
}{}


%\renewcommand{\solution}[1]{}
\renewenvironment{solution}{%
  \catcode`\{=12
  \catcode`\}=12
  \DoNullsolution
}{%
  \ignorespacesafterend
}



}

\newcommand{\solutiononly}{
% эта команда переопределяет команду \solution

\setcounter{problem}{0}
\setcounter{solution}{0}
\setcounter{secsolution}{0}

\renewenvironment{problemtext}{%
  \catcode`\{=12
  \catcode`\}=12
  \DoNullproblemtext
}{%
  \ignorespacesafterend
}

\renewenvironment{solution}{}{}

\renewcommand{\chapter}[1]{\newsecsolution} % можно сюда чего-то добавить, чтобы решения отделялись как-то по секциям
\renewcommand{\section}[1]{}

\renewenvironment{problem}{%
  \catcode`\{=12
  \catcode`\}=12
  \DoNullproblem
}{%
  \ignorespacesafterend
}

\renewenvironment{solution}{%
\refstepcounter{solution}
% \phantomsection
\hyperref[p\thesolution]{\thesolution} \label{s\thesolution}
}{}

} % end of definition for \solutiononly

\newcommand{\problemandsolution}{
% эта команда переопределяет команды \solution, \problem

\setcounter{problem}{0}
\setcounter{solution}{0}
\setcounter{secsolution}{0}

\renewcommand{\problemtext}[1]{##1}
\renewcommand{\solutiontext}[1]{##1}

\renewcommand{\chapter}[1]{\oldchapter{##1}\newsecsolution}
\renewcommand{\section}[1]{\oldsection{##1}}

\renewcommand{\problem}[1]{
\refstepcounter{problem}
% \phantomsection % создаем точку привязки для команды \label
\textbf{Задача} 
\hyperref[s\theproblem]{\theproblem} % гиперссылка на метку "s1.1."
\label{p\theproblem} % метка "p1.1." 
\par ##1}
\renewcommand{\solution}[1]{
\refstepcounter{solution}
% \phantomsection
\hyperref[p\thesolution]{\thesolution} \label{s\thesolution}
##1}
}


