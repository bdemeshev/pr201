









Листок 6 по ТВ и МС 2013--2014 [08.03.2014]







1

Кафедра математической экономики и эконометрики НИУ ВШЭ. Борзых Д. А.

Листок 6

Нормальное распределение



\textbf{Определение.} Случайная величина $X:\Omega \to {\mathbb R}$ имеет \textit{нормальное распределение с параметрами $\mu \in {\mathbb R}$ и $\sigma ^{2} >0$}, пишут $X\sim N(\mu ,\sigma ^{2} )$, если плотность распределения случайной величины $X$ имеет вид 

\[f_{X} (x)={\tfrac{1}{\sqrt{2\pi \sigma ^{2} } }} e^{-{\tfrac{(x-\mu )^{2} }{2\sigma ^{2} }} } .\] 

\textbf{Определение.} Если $X\sim N(0,1)$, то говорят, что случайная величина $X$ имеет \textit{стандартное нормальное распределение}.



\textbf{Задача 1.} Пусть случайная величина $X$ имеет стандартное нормальное распределение. 

Найдите

\begin{enumerate}
\item  ${\rm {\mathbb P}}\left(\left\{-1<X<1\right\}\right)$;

\item  ${\rm {\mathbb P}}\left(\left\{0<X<1\right\}\right)$;

\item  ${\rm {\mathbb P}}\left(\left\{-2<X<2\right\}\right)$;

\item  ${\rm {\mathbb P}}\left(\left\{-2<X\le 0\right\}\right)$;

\item  ${\rm {\mathbb P}}\left(\left\{-2<-X+1\le 0\right\}\right)$;

\item  ${\rm {\mathbb P}}\left(\left\{0<1-X<1\right\}\right)$;

\item  ${\rm {\mathbb P}}\left(\left\{0<1-{\tfrac{1}{2}} X<1\right\}\right)$;

\item  ${\rm {\mathbb P}}\left(\left\{0<1-{\tfrac{1}{2}} X<{\tfrac{1}{2}} \right\}\right)$;

\item  ${\rm {\mathbb P}}\left(\left\{X\in \left[0;1\right]\bigcup \left[2;3\right]\right\}\right)$;

\item  ${\rm {\mathbb P}}\left(\left\{X\in \left[-3;-2\right]\bigcup \left[2;3\right]\right\}\right)$;

\item  ${\rm {\mathbb P}}\left(\left\{X\in \left(-3;0\right)\bigcup \left(0;-3\right)\right\}\right)$;

\item  ${\rm {\mathbb P}}\left(\left\{X\in \left[-2;-1\right]\bigcup \left[2;3\right]\right\}\right)$;

\item  ${\rm {\mathbb P}}\left(\left\{X\in \left[-3;-2\right]\bigcup \left[-{\tfrac{1}{2}} ;-{\tfrac{1}{3}} \right]\right\}\right)$.
\end{enumerate}

\textbf{Задача 2.} Пусть случайная величина $X\sim N\left(1,4\right)$. 

Найдите

\begin{enumerate}
\item  ${\rm {\mathbb P}}\left(\left\{1<X<4\right\}\right)$;

\item  ${\rm {\mathbb P}}\left(\left\{2<X<4\right\}\right)$;

\item  ${\rm {\mathbb P}}\left(\left\{3<X<4\right\}\right)$.
\end{enumerate}

\textbf{Задача 3.} Пусть случайная величина $X$ имеет плотность $f_{X} \left(x\right)={\tfrac{1}{\sqrt{4\pi } }} \cdot e^{-{\tfrac{x^{2} }{4}} } $. 

Найдите

\begin{enumerate}
\item  ${\rm {\mathbb P}}\left(\left\{1<X<2\right\}\right)$;

\item  ${\rm {\mathbb P}}\left(\left\{2<X<3\right\}\right)$;

\item  ${\rm {\mathbb P}}\left(\left\{-2<X<1\right\}\right)$;

\item  ${\rm {\mathbb P}}\left(\left\{-2<-X<1\right\}\right)$.
\end{enumerate}

Ответы:

\begin{enumerate}
\item \begin{enumerate}
\item  ${\rm {\mathbb P}}\left(\left\{1<X<2\right\}\right)={\rm 0.1611}$;

\item  ${\rm {\mathbb P}}\left(\left\{2<X<3\right\}\right)={\rm 0.0617}$;

\item  ${\rm {\mathbb P}}\left(\left\{-2<X<1\right\}\right)={\rm 0.6816}$;

\item  ${\rm {\mathbb P}}\left(\left\{-2<-X<1\right\}\right)={\rm 0.6816}$.
\end{enumerate}
\end{enumerate}

\textbf{Задача 4.} Пусть случайная величина $X$ имеет плотность $f_{X} \left(x\right)={\tfrac{1}{\sqrt{4\pi } }} \cdot e^{-{\tfrac{\left(x-1\right)^{2} }{4}} } $. 

Найдите

\begin{enumerate}
\item  ${\rm {\mathbb P}}\left(\left\{2<X<3\right\}\right)$;

\item  ${\rm {\mathbb P}}\left(\left\{3<X<4\right\}\right)$;

\item  ${\rm {\mathbb P}}\left(\left\{-1<X<2\right\}\right)$.
\end{enumerate}

\textbf{Задача 5.} Пусть случайная величина$X$ имеет функцию распределения $F_{X} \left(x\right)=\int _{-\infty }^{x}{\tfrac{1}{\sqrt{6\pi } }} \cdot e^{-{\tfrac{t^{2} }{6}} } dt $. 

Найдите

\begin{enumerate}
\item  ${\rm {\mathbb P}}\left(\left\{1<X<2\right\}\right)$;

\item  ${\rm {\mathbb P}}\left(\left\{2<X<3\right\}\right)$;

\item  ${\rm {\mathbb P}}\left(\left\{-2<X<1\right\}\right)$.
\end{enumerate}

\textbf{Задача 6.} При помощи таблиц стандартного нормального распределения найдите следующие интегралы.

\begin{enumerate}
\item  $\int _{-1}^{1}{\tfrac{1}{\sqrt{2\pi } }} \cdot e^{-{\tfrac{x^{2} }{2}} } dx $;

\item  $\int _{-2}^{2}{\tfrac{1}{\sqrt{2\pi } }} \cdot e^{-{\tfrac{x^{2} }{2}} } dx $;

\item  $\int _{-1}^{2}{\tfrac{1}{\sqrt{2\pi } }} \cdot e^{-{\tfrac{x^{2} }{2}} } dx $;

\item  $\int _{-2}^{1}{\tfrac{1}{\sqrt{2\pi } }} \cdot e^{-{\tfrac{x^{2} }{2}} } dx $;

\item  $\int _{0}^{1}{\tfrac{1}{\sqrt{2\pi } }} \cdot e^{-{\tfrac{x^{2} }{2}} } dx $;

\item  $\int _{-2}^{0}{\tfrac{1}{\sqrt{2\pi } }} \cdot e^{-{\tfrac{x^{2} }{2}} } dx $;

\item  $\int _{1}^{2}{\tfrac{1}{\sqrt{2\pi } }} \cdot e^{-{\tfrac{x^{2} }{2}} } dx $;

\item  $\int _{-3}^{-2}{\tfrac{1}{\sqrt{2\pi } }} \cdot e^{-{\tfrac{x^{2} }{2}} } dx $;

\item  $\int _{1}^{2}{\tfrac{1}{\sqrt{8\pi } }} \cdot e^{-{\tfrac{x^{2} }{8}} } dx $;

\item  $\int _{-1}^{2}{\tfrac{1}{\sqrt{8\pi } }} \cdot e^{-{\tfrac{x^{2} }{8}} } dx $;

\item  $\int _{-2}^{2}{\tfrac{1}{\sqrt{8\pi } }} \cdot e^{-{\tfrac{x^{2} }{8}} } dx $;

\item  $\int _{1}^{2}{\tfrac{1}{\sqrt{18\pi } }} \cdot e^{-{\tfrac{x^{2} }{18}} } dx $;

\item  $\int _{-1}^{2}{\tfrac{1}{\sqrt{18\pi } }} \cdot e^{-{\tfrac{x^{2} }{18}} } dx $;

\item  $\int _{-2}^{2}{\tfrac{1}{\sqrt{18\pi } }} \cdot e^{-{\tfrac{x^{2} }{18}} } dx $;

\item  $\int _{-1}^{1}e^{-x^{2} } dx $.
\end{enumerate}

\textbf{Задача 7.} Случайная величина $X$ имеет нормальное распределение с параметрами $\mu =1$, $\sigma =2$. Найти ${\rm {\mathbb P}}\left(\left\{1.5<X<3.0\right\}\right)$.

Ответ: ${\rm {\mathbb P}}\left(\left\{1.5<X<3.0\right\}\right)=0.2426$.

\textbf{Задача 8.} Случайная величина $X$ имеет нормальное распределение с параметрами $\mu =-1$, $\sigma =5$. Найти 

\begin{enumerate}
\item  ${\rm {\mathbb P}}\left(\left\{-6.0<X<6.0\right\}\right)$;

\item  ${\rm {\mathbb P}}\left(\left\{-6.0<X<-1.0\right\}\right)$;

\item  ${\rm {\mathbb P}}\left(\left\{X>-1.0\right\}\right)$.
\end{enumerate}

Ответы:

\begin{enumerate}
\item  ${\rm {\mathbb P}}\left(\left\{-6.0<X<6.0\right\}\right)=0.7605$;

\item  ${\rm {\mathbb P}}\left(\left\{-6.0<X<-1.0\right\}\right)=0.3413$;

\item  ${\rm {\mathbb P}}\left(\left\{X>-1.0\right\}\right)=0.5000$.
\end{enumerate}

\textbf{Задача 9.} Случайные величины $X$ и $Y$ независимы и имеют нормальное распределение, ${\rm {\mathbb E}}\left(X\right)=0$, $D\left(X\right)=1$, ${\rm {\mathbb E}}\left(Y\right)=2$, $D\left(Y\right)=6$. 

Найдите

\begin{enumerate}
\item  ${\rm {\mathbb P}}\left(\left\{-1.5<X<0.5\right\}\right)$;

\item  ${\rm {\mathbb P}}\left(\left\{1<X+2Y<7\right\}\right)$.
\end{enumerate}

\textbf{Задача 10.} Случайные величины $X$ и $Y$ независимы и имеют нормальное распределение, ${\rm {\mathbb E}}\left(X\right)=0$, $D\left(X\right)=1$, ${\rm {\mathbb E}}\left(Y\right)=3$, $D\left(Y\right)=7$.  

Найдите

\begin{enumerate}
\item  ${\rm {\mathbb P}}\left(\left\{0.6<X<1.8\right\}\right)$;

\item  ${\rm {\mathbb P}}\left(\left\{1<3X+Y<5\right\}\right)$.
\end{enumerate}

\textbf{Задача 11.} Вычислить квантили стандартного нормального распределения уровней $\alpha =0.01$, $\alpha =0.05$ и $\alpha =0.1$.

Решение.

Способ №1 (при помощи программы MS Excel)



\[z_{0.01} ={\rm \; !"\; }\left({\rm 0,01}\right)=-{\rm 2},{\rm 32635}\] 

\[z_{0.05} ={\rm \; !"\; }\left({\rm 0,05}\right)=-{\rm 1},{\rm 64485}\] 

\[z_{0.1} ={\rm \; !"\; }\left({\rm 0,1}\right)=-{\rm 1},{\rm 28155}\] 



Способ №2 (при помощи пакета MATLAB)



\[z_{0.01} =norminv\left(0.01,\; 0,\; 1\right)=-{\rm 2.3263}\] 

\[z_{0.05} =norminv\left(0.05,\; 0,\; 1\right)=-{\rm 1.6449}\] 

\[z_{0.1} =norminv\left(0.1,\; 0,\; 1\right)=-{\rm 1.2816}\] 



Способ №3 (при помощи таблиц для нормального распределения)\footnote{ См. А.С. Шведов ``Теория вероятностей и математическая статистика'', 2-е изд., стр. 229-230.}

$\alpha =0.01$: ищем в таблице для нормального распределения такую точку $x$, чтобы функция $S\left(x\right)=1-2\cdot \alpha =1-2\cdot 0.01=0.98$, где$S\left(x\right)=\frac{1}{\sqrt{2\pi } } \int _{-x}^{x}e^{-{\tfrac{y^{2} }{2}} } dy $. Такую точку можно найти в строке $``{\rm 2}.{\rm 3}{"} $ и столбце $``{\rm 3}{"} $, следовательно, $x=2.33$. Значит, при помощи таблиц нормального распределения получаем $z_{0.01} =-2.33$.



$\alpha =0.05$: ищем в таблице для нормального распределения такую точку $x$, чтобы функция $S\left(x\right)=1-2\cdot \alpha =1-2\cdot 0.05=0.90$, где$S\left(x\right)=\frac{1}{\sqrt{2\pi } } \int _{-x}^{x}e^{-{\tfrac{y^{2} }{2}} } dy $. Такую точку можно найти в строке $``1.6{"} $ и столбце $``4{"} $, следовательно, $x=1.64$. Значит, при помощи таблиц нормального распределения получаем $z_{0.01} =-1.64$.



$\alpha =0.1$: ищем в таблице для нормального распределения такую точку $x$, чтобы функция $S\left(x\right)=1-2\cdot \alpha =1-2\cdot 0.1=0.80$, где$S\left(x\right)=\frac{1}{\sqrt{2\pi } } \int _{-x}^{x}e^{-{\tfrac{y^{2} }{2}} } dy $. Такую точку можно найти в строке $``1.2{"} $ и столбце $``8{"} $, следовательно, $x=1.28$. Значит, при помощи таблиц нормального распределения получаем $z_{0.01} =-1.28$.

\textbf{Задача 12.} Вычислить квантили стандартного нормального распределения уровней 

\begin{enumerate}
\item  $\alpha =0.15$; 

\item  $\alpha =0.2$; 

\item  $\alpha =0.25$; 

\item  $\alpha =0.35$;

\item  $\alpha =0.85$;

\item  $\alpha =0.80$;

\item  $\alpha =0.75$;

\item  $\alpha =0.65$.
\end{enumerate}

Ответ:

\begin{enumerate}
\item  $z_{0.15} =-{\rm 1.0364}$;

\item  $z_{0.2} =-{\rm 0.8416}$;

\item  $z_{0.25} =-{\rm 0.6745}$;

\item  $z_{0.35} =-{\rm 0.3853}$;

\item  $z_{0.85} ={\rm 1.0364}$;

\item  $z_{0.8} ={\rm 0.8416}$;

\item  $z_{0.75} ={\rm 0.6745}$;

\item  $z_{0.35} ={\rm 0.3853}$.
\end{enumerate}

\textbf{Задача 13.} Найдите такое число $x\in {\mathbb R}$, что

\begin{enumerate}
\item \begin{enumerate}
\item  $\int _{x}^{+\infty }{\tfrac{1}{\sqrt{2\pi } }} e^{-{\tfrac{t^{2} }{2}} } dt =0.01$;

\item  $\int _{x}^{+\infty }{\tfrac{1}{\sqrt{2\pi } }} e^{-{\tfrac{t^{2} }{2}} } dt =0.05$;

\item  $\int _{x}^{+\infty }{\tfrac{1}{\sqrt{2\pi } }} e^{-{\tfrac{t^{2} }{2}} } dt =0.1$;

\item  $\int _{x}^{+\infty }{\tfrac{1}{\sqrt{8\pi } }} e^{-{\tfrac{t^{2} }{8}} } dt =0.1$;

\item  $\int _{x}^{+\infty }{\tfrac{1}{\sqrt{8\pi } }} e^{-{\tfrac{\left(t-8\right)^{2} }{8}} } dt =0.1$;

\item  $\int _{x}^{+\infty }{\tfrac{1}{\sqrt{50\pi } }} e^{-{\tfrac{t^{2} }{50}} } dt =0.1$;

\item  $\int _{x}^{+\infty }{\tfrac{1}{\sqrt{50\pi } }} e^{-{\tfrac{\left(t-50\right)^{2} }{50}} } dt =0.1$.
\end{enumerate}
\end{enumerate}

Ответы:

\begin{enumerate}
\item \begin{enumerate}
\item \begin{enumerate}
\item  $x={\rm 2.3263}$;

\item  $x={\rm 1.6449}$;

\item  $x={\rm 1.2816}$;

\item  $x={\rm 2.5631}$;

\item  $x={\rm 10.5631}$;

\item  $x={\rm 6.4078}$;

\item  $x={\rm 56.4078}$.
\end{enumerate}
\end{enumerate}
\end{enumerate}

\textbf{Задача 14.} Пусть случайная величина $X$ имеет нормальное распределение с нулевым математическим ожиданием. Найдите значение параметра $\sigma $, при котором вероятность попадания случайной величины $X$ в интервал $\left(5;10\right)$ была бы наибольшей.

Ответ: $\sigma =\sqrt{75/\left(2\ln 2\right)} $.

\textbf{Задача 15.} Пусть $X\sim N\left(\mu ,\sigma ^{2} \right)$. Вычислите вероятность попадания случайной величины в интервал $\left(\mu -\sigma ;\mu \right)$.

Ответ:

\textbf{Задача 16.} Пусть $X\sim N\left(\mu ,\sigma ^{2} \right)$. Вычислите вероятность попадания случайной величины в интервал $\left(\mu -2\sigma ;\mu \right)$.

Ответ:

\textbf{Задача 17.} Пусть $X\sim N\left(\mu ,\sigma ^{2} \right)$. Вычислите вероятность попадания случайной величины в интервал $\left(\mu -3\sigma ;\mu \right)$.

Ответ:

\textbf{Задача 18.} Пусть случайная величина $X$ имеет плотность вида

\begin{enumerate}
\item  $f_{X} \left(x\right)=C\cdot e^{-x^{2} } $;

\item  $f_{X} \left(x\right)=C\cdot e^{-x^{2} +1} $;

\item  $f_{X} \left(x\right)=C\cdot e^{-x^{2} +2x} $;

\item  $f_{X} \left(x\right)=C\cdot e^{-x^{2} -2x} $;

\item  $f_{X} \left(x\right)=C\cdot e^{-2x^{2} +4x} $;

\item  $f_{X} \left(x\right)=C\cdot e^{-2x^{2} +8x} $;

\item  $f_{X} \left(x\right)=C\cdot e^{-8x^{2} -16x} $.
\end{enumerate}

Найдите константу $C$, и покажите, что случайная величина $X$ имеет нормальное распределение.

Ответы: 

\begin{tabular}{|p{0.6in}|p{0.6in}|p{0.6in}|p{0.6in}|p{0.6in}|p{0.6in}|p{0.6in}|} \hline 
A. ${\tfrac{1}{\sqrt{\pi } }} $ & B. ${\tfrac{1}{e\sqrt{\pi } }} $, & C. ${\tfrac{1}{e\sqrt{\pi } }} $, & D. ${\tfrac{1}{e\sqrt{\pi } }} $, & E. ${\tfrac{1}{e^{2} \sqrt{{\tfrac{\pi }{2}} } }} $, & F. ${\tfrac{1}{e^{8} \sqrt{{\tfrac{\pi }{2}} } }} $, & G. ${\tfrac{1}{e^{8} \sqrt{{\tfrac{\pi }{8}} } }} $. \\ \hline 
\end{tabular}

\textbf{Задача 19.} Пусть случайная величина $X$ имеет плотность вида $f_{X} \left(x\right)=C\cdot e^{-ax^{2} +bx} $, где $a>0$ и $b\in {\mathbb R}$. Найдите константу $C$, и покажите, что случайная величина $X$ имеет нормальное распределение.

Ответ: $C=\sqrt{{\tfrac{a}{\pi }} } \cdot e^{-{\tfrac{b^{2} }{4a}} } $.

\textbf{Задача 20.} Случайная величина $X$ имеет нормальное распределение с параметрами $\mu $ и $\sigma ^{2} $, где $\mu \in {\mathbb R}$, а $\sigma ^{2} >0$. Найдите ${\rm {\mathbb P}}\left(\left\{X\in \left[\mu -3\sigma ;\mu +3\sigma \right]\right\}\right)$.

\textbf{Задача 21.} Пусть случайная величина $X$ имеет нормальное распределение; $a$ и $b$ - произвольные вещественные числа, причем $a\ne 0$. Докажите, что случайная величина $Y=aX+b$ также имеет нормальное распределение.

\textbf{Задача 22.} Известна плотность случайной величины $X$: $f_{X} \left(x\right)={\tfrac{1}{\sqrt{8\pi } }} e^{-{\tfrac{\left(x-8\right)^{2} }{8}} } $.

Найдите плотность случайной величины $Y=\frac{X-8}{2} $.

\textbf{Задача 23.} Известна плотность случайной величины $X$: $f_{X} \left(x\right)={\tfrac{1}{\sqrt{18\pi } }} e^{-{\tfrac{\left(x-18\right)^{2} }{18}} } $.

Найдите плотность случайной величины $Y=\frac{X-18}{3} $.

\textbf{Задача 24.} Случайная величина $X$ имеет нормальное распределение с параметрами $\mu =1$, $\sigma =1$. Найти плотность распределения случайной величины $Y=2X+1$.

Ответ: $f_{Y} (x)={\tfrac{1}{\sqrt{8\pi } }} e^{-{\tfrac{(x+1)^{2} }{8}} } $.

\textbf{Задача 25.} Случайная величина $X$ имеет нормальное распределение с параметрами $\mu =1$, $\sigma =1$. Найти плотность распределения случайной величины $Y=-2X+1$.

\textbf{Задача 26.} Случайная величина $X$ имеет нормальное распределение с параметрами $\mu =1$, $\sigma =2$. Найти плотность распределения случайной величины $Y=-2X+1$.

\textbf{Задача 27.} Случайная величина $X$ имеет нормальное распределение с параметрами $\mu =0$, $\sigma =1$. Найти функцию распределения случайной величины $Y=X+\left|X\right|$.

\textbf{Задача 28.} Случайная величина $X$ имеет нормальное распределение с параметрами $\mu =0$, $\sigma =1$. Найти функцию распределения случайной величины $Y=\frac{X+\left|X\right|}{2} $.

\textbf{Задача 29.} Случайная величина $X$ имеет нормальное распределение с параметрами $\mu =0$, $\sigma =1$. Найти функцию распределения случайной величины $Y=\frac{X-\left|X\right|}{2} $.

\textbf{Задача 30.} Случайная величина $X$ имеет нормальное распределение с параметрами $\mu =0$, $\sigma =1$. Найти плотность распределения случайной величины $Y=\left|X\right|$.

\textbf{Задача 31.} Случайная величина $X$ имеет нормальное распределение с параметрами $\mu =0$, $\sigma =1$. Найти плотность распределения случайной величины $Y=X^{2} $.

\textbf{Задача 32.} Случайная величина $X$ имеет нормальное распределение с параметрами $\mu =0$, $\sigma =1$. Найти функцию распределения случайной величины $Y=\exp \left(X\right)$.

\textbf{Задача 33.} Пусть случайные величины $X$ и $Y$ независимы. Случайная величина $X$ имеет стандартное нормальное распределение, а случайная величина $Y$ имеет распределение Бернулли с параметром $p$. Найдите плотность случайной величины $Z=X+Y$. 

\textbf{Задача 34.} Случайная величина $X$ имеет нормальное распределение с параметрами $\mu $ и $\sigma ^{2} $, где $\mu \in {\mathbb R}$, а $\sigma ^{2} >0$. Найти плотность распределения случайной величины $Y=\exp \left(X\right)$.

\textbf{Задача 35.} Случайная величина $X$ имеет нормальное распределение с параметрами $\mu =0$, $\sigma =1$. Найти ${\rm {\mathbb E}}\left(X\right)$ и ${\rm {\mathbb E}}\left(X^{2} \right)$.

Ответы: ${\rm {\mathbb E}}\left(X\right)=0$; ${\rm {\mathbb E}}\left(X^{2} \right)=1$.

\textbf{Задача 36.} Случайная величина $X$ имеет нормальное распределение с параметрами $\mu =0$, $\sigma =1$. Найти ${\rm {\mathbb E}}\left|X\right|$.

Ответ: ${\rm {\mathbb E}}\left|X\right|=\sqrt{{\tfrac{2}{\pi }} } $.

\textbf{Задача 37.} Случайная величина $X$ имеет нормальное распределение с параметрами $\mu =0$, $\sigma =1$. Найти ${\rm {\mathbb E}}\left(X^{3} \right)$ и ${\rm {\mathbb E}}\left(X^{4} \right)$.

Ответы: ${\rm {\mathbb E}}\left(X^{3} \right)=0$; ${\rm {\mathbb E}}\left(X^{4} \right)=3$.

\textbf{Задача 38.} Случайная величина $X$ имеет нормальное распределение с параметрами $\mu =0$, $\sigma =1$. Найти ${\rm {\mathbb E}}\exp \left(X\right)$.

\textbf{Задача 39.} Случайная величина $X$ имеет нормальное распределение с параметрами $\mu $ и $\sigma ^{2} $, где $\mu \in {\mathbb R}$, а $\sigma ^{2} >0$. Найти ${\rm {\mathbb E}}\exp \left(X\right)$ и $D\left(\exp \left(X\right)\right)$.

\textbf{Задача 40.} Для каждого натурального числа $k$ вычислите интеграл $I_{k} =\int _{0}^{+\infty }x^{k} e^{-{\tfrac{x^{2} }{2}} } dx $.





