

Листок 10 по ТВ и МС 2013--2014 [08.03.2014]







1

Кафедра математической экономики и эконометрики НИУ ВШЭ. Борзых Д. А.

Листок 10

Несмещенность оценок



\textbf{Задача 1.} Пусть $X=(X_{1} ,...,X_{n} )$ --- случайная выборка, $X_{i} \sim N(\mu ,\sigma ^{2} )$. Является ли оценки $\hat{\mu }=\bar{X}$, $\widehat{\mu ^{2} }=\overline{X^{2} }$ и $\widehat{\sigma ^{2} }={\tfrac{1}{n}} \sum _{i=1}^{n}(X_{i} -\bar{X})^{2}  $ несмещенными оценками параметров $\mu $, $\mu ^{2} $ и $\sigma ^{2} $ соответственно? 

\textbf{Ответ:} да, нет, нет.

 

\textbf{Задача 2.} Пусть $X=(X_{1} ,...,X_{n} )$ --- случайная выборка из равномерного распределения на отрезке $[0;\theta ]$, $\theta >0$. Является ли следующие оценки несмещенными

\begin{enumerate}
\item  $\hat{\theta }=2\bar{X}$,

\item  $\hat{\theta }=X_{(n)} $?
\end{enumerate}

\textbf{Ответ:} (a) да, (b) нет.

 

\textbf{Задача 3.} Пусть $X=(X_{1} ,...,X_{n} )$ --- случайная выборка из равномерного распределения на отрезке $[\theta ;0]$, $\theta <0$. Является ли $\hat{\theta }=X_{(1)} $ несмещенной оценкой для неизвестного параметра $\theta $?

\textbf{Ответ:} нет.



\textbf{Задача 4. }Пусть $X=\left(X_{1} ,...,X_{n} \right)$ --- случайная выборка. Случайные величины $X_{1} ,...,X_{n} $ имеют дискретное распределение, которое задано при помощи таблицы 

\begin{tabular}{|p{0.5in}|p{0.6in}|p{0.6in}|p{0.6in}|} \hline 
$X_{i} $ & $-2$ & $0$ & $1$ \\ \hline 
${\rm {\mathbb P}}_{X_{i} } $ & $1/2-\theta $ & $1/2$ & $\theta $ \\ \hline 
\end{tabular}

Является ли $\hat{\theta }=(\bar{X}+1)/3$ несмещенной оценкой неизвестного параметра $\theta $?

\textbf{Ответ:} да.



\textbf{Задача 5.} Пусть $X=(X_{1} ,...,X_{n} )$ --- случайная выборка. Случайные величины $X_{1} ,...,X_{n} $ имеют дискретное распределение, которое задано при помощи таблицы

\begin{tabular}{|p{0.6in}|p{0.6in}|p{0.6in}|p{0.6in}|} \hline 
$X_{i} $ & $-3$ & $0$ & $2$ \\ \hline 
${\rm {\mathbb P}}_{X_{i} } $ & $2/3-\theta $ & $1/3$ & $\theta $ \\ \hline 
\end{tabular}

Является ли $\hat{\theta }=(\bar{X}+2)/5$ несмещенной оценкой неизвестного параметра $\theta $?

\textbf{Ответ:} да.



\textbf{Задача 6.} Пусть $X=(X_{1} ,...,X_{n} )$ --- случайная выборка. Случайные величины $X_{1} ,...,X_{n} $ имеют дискретное распределение, которое задано при помощи таблицы

\begin{tabular}{|p{0.6in}|p{0.6in}|p{0.6in}|p{0.6in}|} \hline 
$X_{i} $ & $-4$ & $0$ & $3$ \\ \hline 
${\rm {\mathbb P}}_{X_{i} } $ & $3/4-\theta $ & $1/4$ & $\theta $ \\ \hline 
\end{tabular}

Является ли $\hat{\theta }=(\bar{X}+3)/6$ несмещенной оценкой неизвестного параметра $\theta $?

\textbf{Ответ:} да.



\textbf{Задача 7.} Пусть $X=\left(X_{1} ,...,X_{n} \right)$ --- случайная выборка из распределения с функцией распределения $F\left(x;\theta \right)=\left\{\begin{array}{l} {1-e^{-{\tfrac{x^{2} }{\theta }} } {\rm \; \; \; ?@8\; }x\ge 0,} \\ {0{\rm \; \; \; \; \; \; \; \; \; \; \; \; ?@8\; }x<0,} \end{array}\right. $

где $\theta >0$. Является ли $\hat{\theta }={\tfrac{4}{\pi }} (\bar{X})^{2} $ несмещенной оценкой неизвестного параметра $\theta $?

\textbf{Ответ:} нет.



\textbf{Задача 8.} Пусть $X=\left(X_{1} ,...,X_{n} \right)$ --- случайная выборка из распределения с плотностью $f\left(x;\theta \right)=\left\{\begin{array}{l} {{\tfrac{1}{\theta }} \cdot e^{-{\tfrac{x}{\theta }} } {\rm \; \; \; ?@8\; }x\ge 0,} \\ {0{\rm \; \; \; \; \; \; \; \; \; \; ?@8\; }x<0,} \end{array}\right. $

где $\theta >0$. Является ли $\hat{\theta }=\bar{X}$ несмещенной оценкой неизвестного параметра $\theta $?

\textbf{Ответ:} да.



