1. Сначала на примере двумерного.
2. Потом для произвольного n

Если сразу начать с n, то мы потеряем 2-ой курс.
Лучше давать больше примеров!
Указывать и скалярную и векторную форму записи!

Теоремы, доказательства которых для двумерного и n-мерного случаев не
отличается доказывать только в двумерном случае.
Если доказательство отличается существенно - доказывать в обоих случаях.

По содержанию:

1. определение - через ф. плотности

2. пример 1. X- норм, Y-норм, но вектор (X,Y) - не норм ok

3. теорема если (X,Y) - норм, и corr(X,Y)=0, то (X,Y) - независимы

4. пример 2. (не совпадающий с примером 1): X - норм, Y-норм,

Corr(X,Y)=0, но X и Y зависимы

5. кратко свойства характеристический функции (без док-ва, т.к. это
другой сюжет)

6. характеристическая функция для нормального вектора

7. теорема что нормальность вектора = нормальность любой линейной комбинации
(с док-вом)

8. Вычисление E(X|Y), Var(X|Y)
Тут надо подчеркнуть для студентов мысль: хочешь упростить вычисления
- выражай через N(0;1) где возможно!
После выведения формулы обсуждаеть ее крайние случаи, монотонность и
прочие свойства, которые можно понять интуитивно.

9. Доказательство (как минимум одномерной) ЦПТ (любопытствующих много,
а свойства характ. функции - в кармане)
Тут еще хочу дать границу погрешности ЦПТ (без док-ва, скорее всего)

10. Примеры задач с решениями!

11. Доказательство того, что нормирующий коэффициент = 2\pi

12. Хи-квадрат - через квадратичную форму (c док-вом)

13. Для одномерного нормального распределения - границы на хвостовые вероятности
