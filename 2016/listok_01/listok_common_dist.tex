\documentclass[10pt,a4paper]{article}

\usepackage[top=3cm, left=2cm, right=2cm]{geometry} % размер текста на странице

\usepackage{tikz} % картинки в tikz
\usepackage{microtype} % свешивание пунктуации

\usepackage{array} % для столбцов фиксированной ширины

\usepackage{indentfirst} % отступ в первом параграфе

\usepackage{sectsty} % для центрирования названий частей
\allsectionsfont{\centering}

\usepackage{amsmath} % куча стандартных математических плюшек

\usepackage{multicol} % текст в несколько колонок

\usepackage{lastpage} % чтобы узнать номер последней страницы

\usepackage{enumitem} % дополнительные плюшки для списков
%  например \begin{enumerate}[resume] позволяет продолжить нумерацию в новом списке

\usepackage{amsmath}
\usepackage{amssymb}


\usepackage{fontspec}
\usepackage{polyglossia}

\setmainlanguage{russian}
\setotherlanguages{english}

% download "Linux Libertine" fonts:
% http://www.linuxlibertine.org/index.php?id=91&L=1
\setmainfont{Linux Libertine O} % or Helvetica, Arial, Cambria
% why do we need \newfontfamily:
% http://tex.stackexchange.com/questions/91507/
\newfontfamily{\cyrillicfonttt}{Linux Libertine O}

\AddEnumerateCounter{\asbuk}{\russian@alph}{щ} % для списков с русскими буквами




\AddEnumerateCounter{\asbuk}{\russian@alph}{щ}
%\renewcommand{\theenumi}{\asbuk{enumi}}
\renewcommand{\theenumii}{\asbuk{enumii}}


% \usepackage[left=1cm,right=1cm,top=1cm,bottom=1cm]{geometry}

\usepackage{fancyhdr} % весёлые колонтитулы
\pagestyle{fancy}
\lhead{Теория вероятностей, листок 1!}
\chead{}
\rhead{10.10.2016}
\lfoot{}
\cfoot{}
\rfoot{\thepage/\pageref{LastPage}}
\renewcommand{\headrulewidth}{0.4pt}
\renewcommand{\footrulewidth}{0.4pt}

\DeclareMathOperator{\tr}{tr}
\DeclareMathOperator{\E}{\mathbb{E}}
\let\P\relax
\DeclareMathOperator{\P}{\mathbb{P}}
\DeclareMathOperator{\Var}{\mathbb{V}ar}
\DeclareMathOperator{\Cov}{\mathbb{C}ov}

\begin{document}


В 1918 году 10 октября в России официально введена новая орфография. Из алфавита были исключены буквы Ѣ (ять), Ѳ (фита), І («и десятеричное»); вместо них стали употребляться: е, ф, и.


\begin{enumerate}

\item Екатерина II подкидывает монетку два раза. Если монетка выпадает орлом, то Екатерина кладет в мешок черный шар, если решкой — белый. Григорий Потёмкин не~знает, как выпадала монетка, и~достает шары из мешка наугад по очереди.
  \begin{enumerate}
    \item Нарисуйте дерево судеб Екатерины II и Григория Потёмкина
    \item Какова вероятность того, что первый шар Григория окажется чёрного цвета?
    \item Первый шар оказался чёрного цвета. Какова вероятность того, что второй шар Григория будет белым?
  \end{enumerate}

\item Лев собрал 100 зверей. Сколькими способами их можно расставить в очередь ко Льву?
\item Лев собрал 100 зверей и решил их раскрасить, каждого целиком в один цвет. Лев хочет 20 красных, 30 желтых и 50 зелёных зверей.
\begin{enumerate}
  \item Сколько существует вариантов раскрасок?
  \item После покраски 10 зверей сбежали. Какова вероятность того, что сбежал хотя бы один красный зверь? Чему равно математическое ожидание числа сбежавших красных зверей?
\end{enumerate}

\item Поручик Ржевский играет в преферанс. Он взял прикуп, снёс две карты и выбрал козыря. У поручика на руках четыре козыря. Какова вероятность, что оставшиеся четыре козыря разделились между двумя остальными игроками как 4:0, 3:1, 2:2?

\item Первопечатник Иван Фёдоров напечатал 10 страниц. Каждая страница содержит опечатки с вероятностью $p=0.25$. Обозначим $X$ – количество страниц с опечатками.
\begin{enumerate}
  \item Найдите $\P(X=8)$, $\P(X>7)$
  \item Сколько в среднем страниц будут с опечатками?
  \item Чему равно наиболее вероятное количество страниц с опечатками?
\end{enumerate}

\item Первопечаткин Иван Фёдоров печатает страницы до первой страницы с опечаткой. Каждая страница содержит опечатки с вероятностью $p=0.25$.

\begin{enumerate}
  \item Какова вероятность того, что Иван Фёдоров напечатает ровно 5 страниц? Больше 10 страниц?
  \item Сколько в среднем страниц напечатает Иван Фёдоров?
\end{enumerate}

\item Предположим, что количество опечаток, сделанных Иваном Фёдоровым во всех книгах, имеет пуассоновское распределение с параметром $\lambda = 50$.
\begin{enumerate}
  \item Какова вероятность того, что Иван Фёдоров сделал ровно 20 опечаток? Меньше трёх опечаток?
  \item Чему равно математическое ожидание количества опечаток?
\end{enumerate}


\item Функция плотности случайной величины $X$ имеет вид
\[
f(t)=
\begin{cases}
  t/8, t\in [0;4] \\
  0, t\notin [0;4]
\end{cases}
\]


\begin{enumerate}
  \item Найдите $\P(X>2)$, $\P(X=3)$
  \item Найдите $\E(X)$, $\E(X^2)$, $\Var(X)$
  \item Найдите $\P(X>2 | X>1)$, $\E(X|X>1)$
  \item Найдите функцию плотности и функцию распределения величины $Y=\ln X$
\end{enumerate}


\item На столе стоят 4 отличающихся друг от друга чашки, 4 одинаковых граненых стакана, 10 одинаковых кусков сахара, 7 соломинок разных цветов. Сколькими способами можно полностью разложить:

а) сахар по чашкам; б) сахар по стаканам; в) соломинки по чашкам; г) соломинки по стаканам;
д) Как изменятся ответы, если требуется, чтобы пустых емкостей не оставалось?




\end{enumerate}





\end{document}
