\documentclass[12pt]{article}

\usepackage{tikz} % картинки в tikz
\usepackage{microtype} % свешивание пунктуации

\usepackage{array} % для столбцов фиксированной ширины

\usepackage{indentfirst} % отступ в первом параграфе

\usepackage{sectsty} % для центрирования названий частей
\allsectionsfont{\centering}

\usepackage{verbatim}
\usepackage{amsmath} % куча стандартных математических плюшек

\usepackage[top=2cm, left=1.5cm, right=1.5cm, bottom=2cm]{geometry} % размер текста на странице

\usepackage{lastpage} % чтобы узнать номер последней страницы

\usepackage{enumitem} % дополнительные плюшки для списков
%  например \begin{enumerate}[resume] позволяет продолжить нумерацию в новом списке
\usepackage{caption}


\usepackage{fancyhdr} % весёлые колонтитулы
\pagestyle{fancy}
\lhead{Теория вероятностей}
\chead{}
\rhead{2017-04-01, праздник номер три :)}
\lfoot{}
\cfoot{}
\rfoot{\thepage/\pageref{LastPage}}
\renewcommand{\headrulewidth}{0.4pt}
\renewcommand{\footrulewidth}{0.4pt}



\usepackage{todonotes} % для вставки в документ заметок о том, что осталось сделать
% \todo{Здесь надо коэффициенты исправить}
% \missingfigure{Здесь будет Последний день Помпеи}
% \listoftodos --- печатает все поставленные \todo'шки


% более красивые таблицы
\usepackage{booktabs}
% заповеди из докупентации:
% 1. Не используйте вертикальные линни
% 2. Не используйте двойные линии
% 3. Единицы измерения - в шапку таблицы
% 4. Не сокращайте .1 вместо 0.1
% 5. Повторяющееся значение повторяйте, а не говорите "то же"



\usepackage{fontspec}
\usepackage{polyglossia}

\setmainlanguage{russian}
\setotherlanguages{english}

% download "Linux Libertine" fonts:
% http://www.linuxlibertine.org/index.php?id=91&L=1
\setmainfont{Linux Libertine O} % or Helvetica, Arial, Cambria
% why do we need \newfontfamily:
% http://tex.stackexchange.com/questions/91507/
\newfontfamily{\cyrillicfonttt}{Linux Libertine O}

\AddEnumerateCounter{\asbuk}{\russian@alph}{щ} % для списков с русскими буквами
\setlist[enumerate, 2]{label=\asbuk*),ref=\asbuk*}

%% эконометрические сокращения
\DeclareMathOperator{\Cov}{Cov}
\DeclareMathOperator{\Corr}{Corr}
\DeclareMathOperator{\Var}{Var}
\DeclareMathOperator{\E}{E}
\def \hb{\hat{\beta}}
\def \hs{\hat{\sigma}}
\def \htheta{\hat{\theta}}
\def \s{\sigma}
\def \hy{\hat{y}}
\def \hY{\hat{Y}}
\def \v1{\vec{1}}
\def \e{\varepsilon}
\def \he{\hat{\e}}
\def \z{z}
\def \hVar{\widehat{\Var}}
\def \hCorr{\widehat{\Corr}}
\def \hCov{\widehat{\Cov}}
\def \cN{\mathcal{N}}


\begin{document}


Главная мораль: байесовский подход — это всего лишь формула условной вероятности.


\begin{enumerate}

\item Задача о целебных лягушках :)

У одного вида лягушек самки обладают целебными свойствами. Самцы и самки встречаются равновероятно. Неподалёку видны аж две лягушки данного вида, но издалека неясно кто.

Определите вероятность того, что среди этих лягушек есть хотя бы одна целебная, в каждой из ситуаций:
\begin{enumerate}
  \item Самцы квакают, самки — нет, со стороны лягушек слашно кваканье, но не разобрать, одной лягушки или двух.
  \item Самцы и самки квакают по разному, но одинаково часто. Только что послышался отдельный квак одной из лягушек и это квак самца.
\end{enumerate}

\item Яичный бой

Саша и Маша играют в «яичный бой». Перед ними корзина яиц. В начале боя они берут по одному яйцу и бьют их острыми концами. Каждое яйцо в корзине обладает своей «силой», все силы — разные. Более сильное яйцо разбивает более слабое. Внешне яйца не отличимы. Сила яйца не убывает при ударах. Разбитое яйцо выбрасывают, побеждённый берёт новое, а победитель продолжает играть прежним.

Какова вероятность того, что Маша победит в 11-ом раунде, если она уже победила 10 раундов подряд?

\item Классика жанра

Перед нами определение бета-распределения $Beta(\alpha, \beta)$:
\[
f(x) \propto \begin{cases}
x^{\alpha-1}(1-x)^{\beta-1}, \text{ если } x\in[0;1] \\
0, \text{ иначе.}
\end{cases}
\]

Блондинка Анжелика хочет оценить неизвестную вероятность встретить динозавра, $p$. Она предполагает, что динозавры встречаются каждый день независимо от других с постоянной вероятностью. Априорно Анжелика считает, что неизвестное $p$ имеет бета-распределение $Beta(2, 3)$. За 20 дней Анжелика 5 раз видела динозавра. Для краткости обозначим вектором $y$ все имеющиеся наблюдения. Величина $y_i$ — результат $i$-го дня: 1, если динозавр встретился, и 0 иначе.
\begin{enumerate}
\item Чему, по-мнению Анжелики, равны априорные $\E(p)$, мода распределения $p$?
\item Найдите апостериорное распределение $f(p|y)$.
\item Найдите апостериорные ожидание $\E(p|y)$ и моду.
\item Найдите условное распределение $y_{21}$ с учётом имеющихся данных.
\end{enumerate}

\newpage
\item Рассмотрим следующий код в \verb|stan|.

\begin{verbatim}
data {
  int<lower=1> N_x;
  int<lower=1> N_y;
  real y[N_y];
  real x[N_x];
}
parameters {
  real mu_x;
  real mu_y;
  real<lower=0> sigma_x;
  real<lower=0> sigma_y;
}
model {
  for (n_x in 1:N_x) {
    x[n_x] ~ normal(mu_x, sigma_x);
  }
  for (n_y in 1:N_y) {
    y[n_y] ~ normal(mu_y, sigma_y);
  }
  mu_x ~ normal(0, 100);
  mu_y ~ normal(0, 100);
  sigma_x ~ exponential(50);
  sigma_y ~ exponential(50);
}
generated quantities {
  delta = mu_x - mu_y;
  ratio = sigma_x / sigma_y;
}
\end{verbatim}

\begin{enumerate}
\item Выпишите предполагаемую модель для данных.
\item Выпишите априорное распределение.
\item Байесовский интервал для каких величин позволяет построить данный код?
\item Какие предпосылки мешают применить в данном случае классический доверительный интервал для разности математических ожиданий, основанный на $F$-распределении?
\end{enumerate}

\item Просто красивая задача про выборку :)

Есть неизвестное количество чисел. Среди этих чисел одно число встречается строго больше 50\% раз. Ведущий показывает числа исследователю Акану в некотором порядке. Когда все числа закончатся, ведущий скажет «всё». Задача Акана — определить, какое число встречается чаще всех. Проблема в том, что Акан так готовился к контрольной по теории вероятностей, что устал. И больше 10 чисел запомнить не в состоянии.

Предложите алгоритм, которой позволит Акану определить искомое число.

\end{enumerate}

\end{document}
