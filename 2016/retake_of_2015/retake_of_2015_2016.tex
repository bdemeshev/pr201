\documentclass{article}
\usepackage[box, % запрет на перенос вопросов
%nopage,
insidebox, % ставим буквы в квадратики
separateanswersheet, % добавляем бланк ответов
nowatermark, % отсутствие надписи "Черновик"
indivanswers,  % показываем верные ответы
%answers,
lang=RU,
completemulti % добавлять "нет правильного ответа" во всех вопросах множественного выбора
]{automultiplechoice}
\usepackage{multicol}
\usepackage[utf8]{inputenc}
\usepackage[russian]{babel}
\usepackage{comment}
\usepackage{amsmath, amssymb}

\renewcommand{\P}{\mathbb{P}}
\newcommand{\E}{\mathbb{E}}
\newcommand{\Var}{\mathrm{Var}}
\newcommand{\Cov}{\mathrm{Cov}}
\newcommand{\Corr}{\mathrm{Corr}}

\newdimen\AMCinterIrep\AMCinterIrep=0ex

\begin{document}

\input{questions_base.tex}

\onecopy{10}{ % делаем {v} вариантов

\noindent{\bf Теория вероятностей и математическая статистика:  \hfill пересдача 20.09.2016}

\vspace{3ex}

Можно пользоваться простым калькулятором.  В каждом вопросе единственный верный ответ. Ни пуха, ни пера!


\vspace{3ex}

\cleargroup{all} % создадим пустую группу вопросов "all"

\shufflegroup{exam_15} % перетасуем группу вопросов "exam_15"
\copygroup[34]{exam_15}{all} % добавим [n] вопросов из группы "exam_15" в группу "all"

%\copygroup{newpage}{all}


% \copygroup[1]{commontext}{all}


%\copygroup{ruler}{all}



%\shufflegroup{all}
\insertgroup{all} % вставим группу "all" в текст работы

%\AMCcleardoublepage
\clearpage

\AMCformBegin

% добавляем/убираем коммент
%Ура! На этой страничке вопросов уже нет :)

\namefield{\fbox{
  \begin{minipage}{42em}
    Имя, фамилия и номер группы:\vspace*{3ex}\par
    \noindent\dotfill\vspace{2mm}
  \end{minipage}
}}

\vspace{2ex}

Номер в списке (для автоматического распознавания):

\vspace{2ex}

\AMCcodeH{student}{3}

\vspace{2ex}

\begin{multicols}{2}
\AMCform
\end{multicols}

\vspace{2ex}

 Учитываются только ответы, перенесённые на этот листок.

}
\end{document}
