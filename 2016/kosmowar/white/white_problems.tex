\documentclass[12pt, addpoints]{exam} % добавить или удалить answers в скобках, чтобы показать ответы
%\usepackage[T2A, T1]{fontenc}
%\usepackage[utf8x]{inputenc}
%\usepackage[greek, russian]{babel}
%\usepackage[OT1]{fontenc}


\usepackage{fontspec}
\usepackage{polyglossia}

\setdefaultlanguage{russian}
\setotherlanguages{english, greek}

\setmainfont[Ligatures=TeX]{Linux Libertine O}
% http://www.linuxlibertine.org/index.php?id=91&L=1

%\usepackage{mathtools}
\usepackage{comment}
\usepackage{booktabs}
\usepackage{amsmath}
\usepackage{icomma}
\usepackage{tikz}
\usepackage{amsfonts}
\usepackage{amssymb}
\usepackage[left=1cm,right=1cm,top=2cm,bottom=2cm]{geometry}
\DeclareMathOperator{\E}{\mathbb{E}}
\DeclareMathOperator{\Var}{\mathbb{V}\mathrm{ar}}
\DeclareMathOperator{\Cov}{\mathbb{C}\mathrm{ov}}
\DeclareMathOperator{\Corr}{\mathbb{C}\mathrm{orr}}
\let\P\relax
\DeclareMathOperator{\P}{\mathbb{P}}
\newcommand{\cN}{\mathcal{N}}
\newcommand{\hbeta}{\hat{\beta}}
\renewcommand{\le}{\leqslant}
\renewcommand{\ge}{\geqslant}
\renewcommand{\leq}{\leqslant}
\renewcommand{\geq}{\geqslant}

\usepackage{floatrow}
%\newfloatcommand{capbtabbox}{table}[][\FBwidth]

\begin{document}

\pagestyle{headandfoot}
\runningheadrule
\firstpageheader{Теория вероятностей}{КосмоWAR, белые задачи}{17 декабря 2016}
\firstpageheadrule
\runningheader{Теория вероятностей}{КосмоWAR, белые задачи}{17 декабря 2016}
\firstpagefooter{}{}{}
\runningfooter{}{}{}
\runningfootrule

\hqword{Задача}
\hpgword{Страница}
\hpword{Максимум}
\hsword{Баллы}
\htword{Итого}
\pointname{\%}
%\renewcommand{\solutiontitle}{\noindent\textbf{Решение:}\par\noindent}
\renewcommand{\solutiontitle}{}

%Таблица с результатами заполняется проверяющим работу. Пожалуйста, не делайте в ней пометок.

%\begin{center}
%  \gradetable[h][questions]
%\end{center}


%vspace{0.2in}
%%\makebox[\textwidth]{Команда:\enspace\hrulefill}


\begin{center}
ЭРА I
\end{center}

\begin{center}
Инженерное дело
\end{center}

\begin{questions}

\question \textbf{Высокоточная линза для лазера.} Константин трудится над оптикой нового уровня. Исследуя движение мелких частиц в сетке алмаза, он установил, что $X \sim \mathcal{N}(0; 4)$, $Y=X+Z$, где $Z \sim \mathcal{N}(0; 3)$, причем $X$ и $Z$ независимы. Помогите Константину найти $\left(\E{X^4} - \Cov{(X^2, Y^2)}\right)$. {\it Подсказка: все подряд считать необязательно.}
\begin{solution}
16
\end{solution}

\question \textbf{Строительство космопорта.} Правительство Конфедерации объявило тендер на строительство космопорта на планете Мар-Сара. Ожидается, что на нее каждый день будет прилетать в среднем $25$ пассажирских шаттлов с планеты Антига Прайм, $42$~грузовых корабля с Тарсониса и $3$~одиночных звездолета с планеты Виктор~V. Компания <<Метелица>> утверждает, что проектируемый ею космодром способен принять до $80$~кораблей в день. Какова вероятность того, что такой космодром не справится с дневной нагрузкой?
\begin{solution}
$\sum\limits_{81}^{+\infty}e^{-70}\dfrac{70^k}{k!} \approx 0,11$
\end{solution}

\question \textbf{Космоветролет\textsuperscript{TM}.} Компания Whizzlespark недавно изобрела космоветролет --- звездолет с парусом для космического ветра --- и уже готовится к выходу на IGO (initial galactic offering). Аналитики предсказывают, что математическое ожидание доходности акции компании составит $4$~г.е. с дисперсией $8$. Вася хочет вложить некоторую сумму в портфель из акций Whizzlespark и облигаций компании <<Метелица>> с минимальной дисперсией.  Мат. ожидание и дисперсия доходности облигаций <<Метелицы>> равны $2$ и $1$ соответственно. Найдите долю акций Whizzlespark в портфеле Васи.

\begin{solution}
0
\end{solution}


\question  \textbf{Обман.} Размер выплат по страховке каждому инвестору, прогоревшему на акциях Whizzlespark --- случайная величина с математическим ожиданием, равным $1000$~ед. и среднеквадратическим отклонением, равным $5000$~ед. Выплаты отдельным клиентам независимы. Сколько должно быть наличных денег в банке, чтобы с вероятностью $0,95$ денег хватило на обслуживание $100$ клиентов?

\begin{solution}
$ \E X = 100*1000, \Var X = 100* 5000*2,
\P \left(\dfrac{X - \E X}{\sigma} \ge \dfrac{Money - \E X}{\sigma}  \right) = 0,95$
\end{solution}

\question \textbf{Мутантоловка.} У инженера Виталия есть близкий друг, который устал воевать с мутантами. Виталий решил изобрести для него в подарок ко дню рождения супер-мутантоловку, которая будет ловить до $10$ мутантов в минуту. Пусть $X$~--- количество пойманных мутантов, $X \in \{1, 2, 3, 4, 5, 6, 7, 8, 9, 10\}$ и вероятности заданы функцией: $\P(X=x) = \frac{x}{c}$. Чему равна величина $c$?
		%
\begin{solution}
$55$
\end{solution}


\question \textbf{Генератор гениальных идей.} Гениальные идеи, приходящие на ум исследователю Пете в день, образуют Пуассоновский поток. Известно, что вероятности появления одной и двух гениальных идей связаны соотношением:  $3\P(X=1) = \P(X=2)$. Найдите вероятность того, что за день Петя выдаст аж $4$~гениальные идеи.
		%
\begin{solution}
$e^{-6}\cdot\dfrac{6^4}{4!}=0.13$
\end{solution}


\end{questions}
% done


\newpage

\begin{center}
ЭРА I
\end{center}

\begin{center}
Биология
\end{center}

\begin{questions}

\question \textbf{Менделевское тригибридное скрещивание.} Скрещиваются гетерозиготный по всем трём признакам М (один глаз, фиолетовые волосы, две ноги) и гетерозиготная по всем трём признакам Ж (один глаз, фиолетовые волосы, две ноги). Какова вероятность, что их ребенок окажется с двумя глазами, оранжевыми волосами и тремя ногами?\\
Гетерозигота здесь --- АаВвСс. \\
{\it Hint:} при данном скрещивании один ген каждого типа заимствуется у отца, второй --- у матери. Рецессивный признак реализуется только при наличии двух рецессивных аллелей, к примеру, {\it aa}.

\begin{solution}
 $\frac{1}{64}$
\end{solution}

\question \textbf{Случайный котёнок.} Елена В. Прекрасная нашла не менее прекрасного котёнка на планете $N$ и решила, что завтра непременно придумает ему имя, а именно наугад вытянет любое из следующего перечня: {\it Вейбулл, Пуассон, Сигма, Чебышёв}.
Однако с вероятностью $\frac{1}{3}$ завтра будет солнечный шторм, от чего у Елены будет болеть голова, тогда на помощь придет её ассистент, который недолго думая назовет котенка именем из списка так, чтобы первая буква имени совпадала с сегодняшним днем недели. Найдите вероятность того, что котёнка назовут Пуассоном (все дни недели равновероятны).

\begin{solution}
 $\frac{11}{42}$
\end{solution}

\question \textbf{Молекулярный анализ.} Исследователь Майкл Браун Младший Младший привез с далекой планеты  $2007 OR_{10}$ образец атмосферы. Известно, что в контейнере $n$ атомов: $n/2$ белых и столько же чёрных, но неизвестно, как они соединены в молекулы. Какова вероятность вытащить молекулу из $n/2$ атомов, в которой будет одинаковое количество чёрных и белых?

\begin{solution} $\frac{(C^{n/4}_{n/2})^2}{C_{n}^{n/2}}$
\end{solution}

\question \textbf{Как приручить гиппогрифа.} Команда исследователей впервые поймала гиппогрифа и посадила его в вольер класса люкс+. Однако гиппогриф~--- это очень вольное животное. В первый день заточения он точно будет сопротивляться, во второй~--- с вероятностью $1/2$, в третий~--- с $1/4$ и так далее.

С какой вероятностью исследователи не смогут приручить его в течение семи дней?
\begin{solution}
 $\dfrac12\cdot\dfrac14\cdot \ldots \cdot\dfrac{1}{64}$
\end{solution}

\question \textbf{Робокот.} Маша и ее $40$ робокошек любят смотреть «Угадай мелодию».  Маша угадывает мелодию ровно с трёх, четырёх или пяти нот с вероятностями $1/4$, $1/2$ и $1/4$ соответственно. Функция распределения числа нот, необходимых кошке Мусе для отгадывания, такова:
\[
F(x)=\begin{cases}
0, \; x \leq4, \\
1/4, \; x \in[4;5),\\
1/2, \; x \in[5;6),\\
3/4, \; x \in[6;7),\\
1, \; x \geq7. \\
\end{cases}
\]
Найдите вероятность того, что одновременно Муся отгадает мелодию ровно с пяти нот, а Маша с четырёх и более.

\begin{solution}
 3/16
\end{solution}

\question \textbf{Лекарство от вредности.} Два друга-биолога договорились встретиться между $10$ и $15$ часами дня в парке, чтобы обсудить концепцию лекарства от вредности. Пришедший первым ждет $30$ минут и уходит, так же поступает другой. Какова вероятность того, что они встретятся, если каждый из друзей может равновероятно прийти в любой момент времени в указанном промежутке?

\begin{solution}
$\frac{19}{100}$
\end{solution}

\end{questions}
% done


\newpage

\begin{center}
ЭРА I
\end{center}

\begin{center}
Кулинария
\end{center}

\begin{questions}
\question \textbf{Неслипающиеся макароны.} Вася живет на далекой планете Дубки. Он очень любит есть макароны (с сосисками), но у него не всегда получается сварить макароны так, чтобы они не слиплись. Если у Васи хорошее настроение, то в $9$ из $10$ случаев макароны не слипаются и Вася вкусно ужинает.  Если же Вася встал не с той ноги, то приготовить не слипшиеся макароны он может только в $1$ случае из $10$. Вероятность проснуться в хорошем настроении равна $0,3$, в плохом~--- $0,7$. Найдите вероятность того, что Вася в плохом настроении приготовит не слипшиеся макароны.

\begin{solution}

$\dfrac{0,7 \cdot 0,1}{0,7 \cdot 0,1 + 0,3 \cdot 0,9} = \dfrac{7}{34}$
\end{solution}

\question \textbf{Фиолетовая пряность.} Марсианский пудинг готовится из фиолетовой пряности и шоколада. Петя голоден и впопыхах, не смерив, смешивает остатки фиолетовой пряности из килограммового пакета и шоколада из килограммовой банки. Между тем, если пряности слишком много~--- хотя бы на~$20\%$ больше, чем шоколада, пудинг будет невкусным. Какова вероятность того, что Петин пудинг будет вкусным, если до этого ингредиенты использовались равномерно и отдельно друг от друга?

\begin{solution}
$\dfrac{7}{12}$
\end{solution}



\question \textbf{Космически вкусные рецепты.} Баба Аня смотрит кулинарное шоу <<Космический вкус>> с вероятностью $\frac34$, баба Таня --- с вероятностью $\frac12$. Каждая из них всегда сохраняет рецепты блюд, когда смотрит шоу. Баба Галя смотрит программу каждое воскресенье, но записывает рецепты с вероятностью всего лишь $\frac14$. Перед Новым годом баба Галя осознаёт, что ей нужны рецепты блюд из всех выпусков этого телешоу. Она может использовать свои записи или, в случае необходимости, попросить рецепты у бабы Ани и бабы Тани. Найдите вероятность того, что у бабы Гали будут все рецепты из всех выпусков шоу. (В году примерно 52 воскресенья.)

\begin{solution}

$\left(\dfrac14 + \dfrac34\cdot\left(1 - \dfrac{1}{4}\cdot\dfrac12\right)\right)^{52} = \left(\dfrac{29}{32}\right)^{52} $
\end{solution}

\question \textbf{Космокола.} Исследователь Н. очень любит космопепси, но еще больше он любит космоколу. Однако из-за нарушения планетой, на которой живет Н., межгалактического кодекса, поставки космоколы временно прекратились. Н. решил быть предельно настойчивым и отыскать бутылочку своего любимого напитка в даркнете. Однако это и впрямь непросто: вероятность того, что продавец продаст ему космопепси под видом космоколы равна $91\%$ (а вы ведь знаете, что разница есть!). Найдите вероятность того, что Н. в поисках своей мечты придется купить ровно $12$ бутылок напитка.

\begin{solution}
 $0.09\cdot(0.91)^{11}$
\end{solution}

\question \textbf{Вино из одуванчиков.} Дима решил начать производство вина из одуванчиков, для чего ему нужно заказать молекулярный пресс, производство которого ведется только на Марсе. Как ни горестно, но помимо стоимости пресса,  $1125$  марсианских рублей (м.\,р.), придется заплатить еще и $175$ м.\,р. за доставку. Дима планирует потратить всю годовую стипендию, которая составляет примерно $16900$ российских рублей. Какова вероятность, того что покупка удастся, если курс марсианского рубля к российскому распределен равномерно на интервале $[8; 20]$.

\begin{solution}
 $\dfrac{5}{12}$
\end{solution}

\question \textbf{Урановый торт.} Профессор Глюк отмечает свой $500$-й день рождения. Он готовит вкусный (наверное) торт с урановым кремом, но отмеряет все ингредиенты на глаз, поэтому радиус торта $R$ случаен. Функция плотности случайной величины $R$ имеет вид
\[
f(r)=\begin{cases}
0, \; r < 0 \\
\cos r, \; r \in [4\pi;9\pi/2] \\
0, \; r > \pi/2
\end{cases}.
\]
Найдите наиболее вероятный радиус торта.

\begin{solution}
$ \dfrac{9 \pi}{2} - 1 \approx 13 $
\end{solution}

\end{questions}

% done

\newpage
%\vspace{0.2in}
%\makebox[\textwidth]{Команда:\enspace\hrulefill}

\begin{center}
ЭРА II
\end{center}

\begin{center}
Биология
\end{center}

\begin{questions}


\question \textbf{Жизнь на Марсе.} Сумасшедший исследователь Сироб сын Сироба из галактики Колесо телеги думает, что в его галактике есть только три типа обитаемых планет: марсы~--- $A$ штук, земли~--- $B$ штук, глизе~667~--- $C$ штук. Каждый день Сироб равновероятно совершает вылет ровно на одну из этих планет. Пусть $X$~--- количество вылетов на марсы, а $Y$~--- количество вылетов на земли за n дней. Найдите $\Cov (X,Y)$.

\begin{solution}
$ \Cov (X,Y) = - n \dfrac{AB}{(A+B+C)^2} $
\end{solution}

\question  \textbf{Душа рыжих.} Пётр где-то услышал, что у людей с рыжим цветом волос нет души, и забеспокоился. К счастью, это не точно: у каждого человека с рыжими волосами душа есть с вероятностью $\frac13$. Пётр сам шатен, но все $10$ его друзей рыжие ---  неудивительно, что его волнует этот вопрос. Какова вероятность того, что ровно у половины друзей Петра есть душа?

\begin{solution}
 $C_{10}^5\left(\frac{1}{3}\right)^5\left(\frac{2}{3}\right)^5$
\end{solution}

\question \textbf{Сверхчеловеки.} Гроксы~--- гибриды людей и машин~--- живут  в середине галактики. Равновероятно они могут полететь на планеты, которые находятся на парсек (космическую единицу) вверх, вниз, влево или вправо (по диагонали гроксы не летают). Пусть $X$ и $Y$ — это абсцисса и ордината корабля гроксов после первого манёвра. Найдите $\Cov (X,Y) $.

\begin{solution}
0
\end{solution}


\question \textbf{Неизвестная молекула.} Биолог Карл работает с неизвестной молекулой. Её масса~--- случайная величина, распределённая равномерно на отрезке $[0;\pi]$ (в а.\,е.\,м.). Обозначим $X_i$ массу $i$-й молекулы, где $i=1, \ldots, 10^{100}$. Определите следующую вероятность: $\P(\min\{X_1,\ldots, X_{10^{100}}\}>e)$

\begin{solution}
$\left(\dfrac{\pi-e}{\pi}\right)^{10^{100}}$
\end{solution}

\question \textbf{Смышленый енот.} Рассеянная лаборантка Маша  ухаживает за енотиками и забывает закрывать дверь. Еноты, увидев открытую дверь,  придерживаются следующего принципа: выбегать из лаборатории с вероятностью $0,8$. Найдите вероятность того, что, вернувшись в кабинет, Маша найдёт хотя бы одного енотика, если до её ухода их было пять.

\begin{solution}
0.87
\end{solution}

\question \textbf{Титановая рыбка.} Анатолий вместе со своей экспедицией нашел пруд с титановыми рыбками и решил отловить несколько для дальнейшего изучения. Вероятность того, что Анатолий поймает хотя бы одну рыбку, сделав $4$ попытки, равна $0,9984$. Найдите, с какой вероятностью он поймает её с первого раза (ни одна рыбка не пострадала).
\begin{solution}
0.8
\end{solution}


\end{questions}

% done


\newpage
\begin{center}
ЭРА II
\end{center}

\begin{center}
Инженерное дело
\end{center}

\begin{questions}

\question  \textbf{Сингулярность не предел.}  Исследователи черных дыр затеяли эксперимент: они создали два одинаковых множества межгалактических кораблей, каждое состоит из $n$ кораблей, и запустили одних к черной дыре в галактике Млечный путь (да-да, в нашей галактике есть черная дыра!), а других к черной дыре в галактике Messier~87. Пусть каждая черная дыра поглощает корабль с вероятностью~$p$. Найдите математическое ожидание общего количества  межгалактических кораблей, поглощенных обеими дырами.

\begin{solution}
$2np$
\end{solution}

\question \textbf{Шифр Глюка.} Профессор Глюк решил защитить свой смартфон от взлома, но вскоре забыл код. Всплывающая подсказка говорит, что нужно понять чему равно $\P(A|B)\P (B|C)\P (C)$, если $\P (A | B \cap C) = \P (A | B)$. Помогите Глюку.


\begin{solution}
 $P(A \cap B \cap C)$
\end{solution}

\question  \textbf{Скорость звука.} В этой задаче будем проверять ваше умение сделать что-то быстро и правильно. Сформулируйте, пожалуйста, центральную предельную теорему.

\begin{solution}
Если $X_i$ независимы,  одинаково распределены с математическим ожиданием $\mu $ и дисперсией $\sigma^2 $, то $\dfrac{\bar{X} - \mu}{\sqrt{\sigma^2/n}} \rightarrow N(0,1) $
\end{solution}

\question \textbf{Просто Архип.}
На втором курсе учится очень красивый молодой человек по имени Архип. Люди в космос летают, придумывают умные часы, а вот он Архип. А лекция по теории вероятностей безусловно очень полезна, но невероятно сложна, поэтому вместо того чтобы слушать лектора, девушки глазеют по сторонам. Количество девушек, которые влюбляются в Архипа за фиксированный промежуток времени времени --- случайная величина, имеющая распределение Пуассона. В среднем за пару по теории вероятностей, в Архипа влюбляется $8$ девушек. Какова вероятность, что за пару в Архипа влюбятся по крайней мере две девушки?

\begin{solution}
 $1 - 9e^{-8}$
\end{solution}

\question \textbf{Потерянная вероятность.} Функция плотности случайной величины имеет следующий вид: $f(x)=\dfrac{1}{\pi(1+x^2)}$. Найдите $\P(X\in[-1;1])$.
\begin{solution}
$\P(X\in[-1;1])=\dfrac{1}{2}$
\end{solution}

\question \textbf{Штурмовик.} На заводе Империи каждому произведенному штурмовику присваивают восьмизначный номер. Какова вероятность, что случайно выбранный штурмовик будет иметь номер с $8$ не повторяющимися цифрами, если все цифры равновероятно и независимо стоят на $8$ позициях?
		%
\begin{solution}
$\cfrac{10!}{2! \cdot 10^8}$
\end{solution}




\end{questions}

% done

\newpage
\begin{center}
ЭРА II
\end{center}

\begin{center}
Кулинария
\end{center}

\begin{questions}

\question \textbf{Плюшки-пампушки.} В кулинарную лавку было доставлено $30$ плюшек-пампушек от первого хлебобулочного комбината и $70$ от второго. Однако в процессе перевозки часть плюшек-пампушек пострадала и прибыла в лавку с дефектом. Процент брака среди плюшек из первого комбината равен $4$, а из второго~--- $8$. Найдите вероятность, что взятая наугад плюшка-пампушка будет иметь дефект.
\begin{solution}
$ (0.3*00.4 + 0.8*00.8 = 0.76) $
\end{solution}


\question \textbf{Волшебный кексик.} Повар хочет испечь волшебные кексики. Для этого ему нужно определить вид функции плотности времени приготовления одного кексика $X$. Из следующих соображений помогите ему найти $a$ и $\E X$:
		\[
		f(t)=\begin{cases}
		0, \; t< 1, \\
		t-a, \; t \in [1;2), \\
		0, \; t \geq 2.
		\end{cases}
		\]

\begin{solution}
$a=\dfrac{1}{2}, \E(X)=\dfrac{19}{12}.$
\end{solution}

\question \textbf{Пирожок с изюминкой. } Совместное распределение случайных величин: $X$~--- изюминок в пирожке и $Y$~---  орехов в нем, задано следующим образом:

		\begin{tabular}{c|c|c|c|}
			\centering
			& $X=0$ & $X=5$  & $X=10$  \\
			\hline $Y=5$ & 0,1 & 0,1 & 0,1 \\
			\hline  $Y=7$&  0,2& 0,2 & 0,3 \\
			\hline
		\end{tabular} \\
		Определите, сколько изюминок в среднем попадает в пирожок.

\begin{solution}
$\E(X)=5.5$
\end{solution}

\question \textbf{Квадратный пудинг.} Афанасий приготовил свой фирменный квадратный пудинг и выложил его на красивую тарелку так, что пудинг образует вписанный в окружность квадрат. Однако в последний момент он решает украсить блюдо клубникой и наудачу кидает ее на тарелку. Какова вероятность того, что ягода приземлится на пудинг?

\begin{solution}
$\frac{2}{\pi}  $
\end{solution}

\question \textbf{Робот-печенькораспределитель.} Прогресс не стоит на месте! Чтобы облегчить жизнь кулинарам, был изобретен робот-печенькораспределитель. Сколькими способами может этот робот разложить $20$ шоколадных печенек  по $5$ различным коробкам так, чтобы в каждой коробке оказалось не менее двух печенек?

\begin{solution}
В каждую коробку кладем по 2 печеньки. Тогда останется 10 печенек и 5 коробок, число способов равно С(10+5-1;4)
\end{solution}

\question \textbf{Сингулярное мясо.} Очень одаренный Петя решил приготовить отбивные и стучит отбойным молотком по огромному куску мяса. От каждого удара Вася --- сосед Пети по квартире --- проснётся с вероятностью $1/10$. Сколько в среднем нужно ударить Пете, чтобы разбудить Васю?
\begin{solution}
9
\end{solution}




\end{questions}


\newpage
\begin{center}
ЭРА III
\end{center}
\begin{center}
Инженерное дело.
\end{center}

\begin{questions}

\question \textbf{Просто минимум.}
Пусть $\xi$~--- случайная величина.  Найдите $a$, при котором достигается минимум выражения $\E (\xi - a)^2$.

\begin{solution}
 $\E (X) $
\end{solution}



\question  \textbf{Закапыватель в глаза.} Исследователь Рафаэль придумал чудо-средство~--- закапыватель в глаза, и перед тем, как объявить миру свое изобретение, он хочет провести следующее испытание: проверить закапыватель на $10$~друзьях ($20$~глаз), где общее количество <<попаданий>>~--- случайная величина $X$. Известно, что при каждом испытании чудо-изобретения капля попадает в глаз с вероятностью $0,9$. Но Рафаэля почему-то терзают сомнения о надобности его изобретения, поэтому он подбрасывает кубик и записывает кол-во выпавших очков~--- случайную величину $Y$. Если $X+Y$ окажется в интервале от $18,5$ до $24,5$, он расскажет миру об изобретении. Оцените вероятность того, что мир узнает об открытии Рафаэля.

\begin{solution}
$P(\cdot) \ge 0,48$
\end{solution}

\question \textbf{Лихой пылесос. } Время за которое пылесос поглащает крошку с пола --- случайная величина, для которой задана следующая функция распределения:
		\[
		F(x)=\begin{cases}
		C_1, \; x< 0, \\
		0.5 x^2+ 0.5x+ C_3, \; x \in [0;1], \\
		C_2, \; x \geq 1.
		\end{cases}
		\]
		Найдите  $C_3$.
		%
\begin{solution}
$C_3=0$
\end{solution}


\question \textbf{Ковер-улет. } Петя, Вася и Толя изобрели ковер-улет размером $4\times 4$ метра. Чтобы усмирить свою неудержимую радость, ребята решили покурить кальян прямо на ковре. Случайно упавшие из кальяна угли прожгли ковер в $15$ разных местах. Какова вероятность того, что останется целым кусочек ковра размером хотя бы метр на метр?
		%
\begin{solution}
1
\end{solution}

\question \textbf{Четкая посадка.} При посадке космического корабля существует некоторое отклонение от намеченного места посадки, которое имеет экспоненциальное распределение. Среднее отклонение от намеченного места посадки равно $5$ метров. Какова вероятность того, что два космических корабля при посадке отклонятся от намеченного места посадки на $7$ метров?

\begin{solution}
0, так как экспоненциальное распределение абсолютно непрерывно
\end{solution}

\question  \textbf{Скорость света.} В этой задаче будем проверять ваше умение сделать что-то быстро и правильно. Сформулируйте, пожалуйста, закон больших чисел.

\begin{solution}
Если $X_i$ независимы, одинаково распределены и математическое ожидание $\E X_i$ существует, то $ plim \bar{X_n}=\E X_1 $
\end{solution}



\end{questions}

\newpage
\begin{center}
ЭРА III
\end{center}

\begin{center}
Биология
\end{center}

\begin{questions}

\question \textbf{Редкий вид.} Маленький принц ищет редкий вид трав для борьбы с баобабами, путешествуя по другим планетам. Он знает, что с вероятностью $p=0,01$ на планете есть редкий вид трав. Какова вероятность того, что, посетив $300$ планет, Маленький принц обнаружит редкий вид трав на пяти планетах?

\begin{solution}
$e^{-3}\cdot\dfrac{3^5}{5!} \approx 0,10082$
\end{solution}


\question  \textbf{Смурфотрекер.} Юные натуралисты доказали существование смурфиков на голубых звездах галактики UDFj-39546284. Теперь исследователям предстоит оценить динамику роста их населения. Помогите им решить следующую задачу: всего живет $n$ смурфиков, они  рождаются в определенном месяце независимо друг от друга. Какое среднее количество месяцев, в которые родился хотя бы один смурф?

\begin{solution}
$12 \cdot (1-{(\frac{11}{12})}^n)$
\end{solution}

\question \textbf{Инопланетяне.} Время $t$ между высадками инопланетян на Землю имеет экспоненциальное распределение: $t \sim Exp(\lambda)$. Найдите такие целые $\lambda$, при которых $\P(t\leq 3) = 0,9 \cdot \P(t\leq 4)$.
\begin{solution}
$\lambda = 1$
\end{solution}

\question \textbf{Ядро дроида.} Энакин отправился на поиски Гривуса в систему Ревет. У него нет карты системы, поэтому он может равновероятно попасть на каждую из планет, в том числе на уже посещённую. Всего в системе $N$ планет, $K$ из них обитаемы. Если Энакин попадает на обитаемую планету, то местные с вероятностью $1/5$ подскажут Энакину, где спрятался Гривус. Найдите ожидаемое количество планет, которые Энакин посетит \textit{прежде} чем доберётся до Гривуса.
\begin{solution}
E = K/N*1/5 + K/N*4/5(1+E) + (1 - K/N - 1/N)(1 + E) Следовательно, E = (N - 1)/(1/5*K + 1)
\end{solution}


\question \textbf{Генеалогическое древо.} В некотором королевстве нашей галактики у бесконечной династии равновероятно рождаются рыцари и принцессы. Сколько детей в династии должно родиться, чтобы чтобы в среднем не более чем в четверти случаев частота рождения принцессы отличалась от $1/2$ на $0,01$ или сильнее?

\begin{solution}
$10^4 $
\end{solution}


\question \textbf{Молекулы-модницы.} Представим, что у нас есть много-много молекул. Одна из них сломала ноготь, разозлилась и толкнула двух своих соседок, а те, в свою очередь, могут с вероятностью $1/2$ толкнуть еще двух соседок, либо постичь нирвану и никого не толкать. Найдите мат. ожидание числа молекул, которые буду задействованы в перепалке.

\begin{solution}
$E = 1/9(2 + 2Е) + 1/3(1 + E)$. Следовательно, $E = 5/4 + 2 = 3.25$
\end{solution}


\end{questions}

% done

\newpage
\begin{center}
ЭРА III
\end{center}

\begin{center}
Кулинария
\end{center}

\begin{questions}

\question \textbf{Брутальная прожарка.} Чтобы приготовить говяжий стейк средней прожарки, нужно довести его до температуры $52^\circ$, а вот для извлечения мифриловой эссенции (диетической альтернативы поваренной соли) нужно довести мифрил до температуры $1\,000\,000\,000^\circ$! На таких температурах термометры дают огромную погрешность, а именно дисперсия каждого измерения равна $100\,000^\circ$. Оцените по неравенству Чебышёва вероятность того, что при температуре $1\,000\,000\,000^\circ$ среднее по ста ее измерениям будет иметь погрешность выше $100\,000^\circ$.
\begin{solution}
$P(\cdot)\le 0,01$
\end{solution}

\question \textbf{Горячая печка.} Когда кулинар Вася готовит печенья в печке, некоторые из них подгорают. Вероятность того, что печенье подгорит, равна $0,95$. Какое количество печенек нужно испечь, чтобы доля неподгорелых была заключена в границах от $0,93$ до $0,97$ включительно с вероятностью не менее $0,93$? Оцените по неравенству Чебышева.
\begin{solution}
$n \ge 1697$
\end{solution}

\question \textbf{Сверхъяркий краситель.} R2-D2 хочет собрать на планете Gourmet-15 пряные цветы, лепестки которых придают необыкновенный цвет и вкус еде, в которую их добавят. R2-D2 знает, что в каждом цветке содержится $X$~мг красителя, причем $X\sim U[4; 16]$. Он хочет собрать один грамм драгоценной приправы. Какова вероятность того, что ему не хватит $180$ цветков?
\begin{solution}
$\P(\cdot) \approx 0,356$
\end{solution}

\question \textbf{Сахарный удар.} Совместная функция плотности количества корицы и ванильного сахара в новом новогоднем сверхсладком кофе из Старбакса имеет вид

      \[
		f(c,v)=\begin{cases}
		2(c^3 + v^3), \; c \in [0;1], v \in [0,1], \\
		0\; \text{иначе},
		\end{cases}
		\]

где $C$ и $V$  измеряются в граммах. Найдите вероятность того, что общий вес корицы и ванили в напитке превысит $1$ грамм.

\begin{solution}
$\P (C + V > 1) = 4/5 $ интегралом
\end{solution}

\question \textbf{Пряная сладость.} Совместная функция плотности количества корицы и ванильного сахара в новом новогоднем сверхсладком кофе из Старбакса имеет вид

      \[
		f(c,v)=\begin{cases}
		2(c^3 + v^3), \; c \in [0;1], v \in [0,1], \\
		0\; \text{иначе,}
		\end{cases}
		\]

где $C$ и $V$  измеряются в граммах. Найдите математическое ожидание количества корицы в напитке.

\begin{solution}
13/20 = 0,65
\end{solution}


\question  \textbf{Цветастые макароны.} Сестры Маргарита и Мари-Элизабет приготовили $n$ штук разноцветных макарон (всего $n$ цветов) и теперь выкладывают их на прилавок случайным образом слева направо. Найдите вероятность того, что синий, желтый и красный будут расположены именно в этом порядке, не обязательно рядом.

\begin{solution}
1/6
\end{solution}

\end{questions}

% done


\end{document}
