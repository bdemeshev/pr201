\documentclass[pdftex,12pt,a4paper]{article}

% jan 2012

% sudo yum install texlive-bbm texlive-bbm-macros texlive-asymptote texlive-cm-super texlive-cyrillic texlive-pgfplots texlive-subfigure
% yum install texlive-chessboard texlive-skaknew % for \usepackage{chessboard}
% yum install texlive-minted texlive-navigator texlive-yax texlive-texapi

% растягиваем границы страницы
%\emergencystretch=2em \voffset=-2cm \hoffset=-1cm
%\unitlength=0.6mm \textwidth=17cm \textheight=25cm

\usepackage{makeidx} % для создания предметных указателей
\usepackage{verbatim} % для многострочных комментариев
\usepackage{cmap} % для поиска русских слов в pdf
\usepackage[pdftex]{graphicx} % для вставки графики 
% omit pdftex option if not using pdflatex


%\usepackage{dsfont} % шрифт для единички с двойной палочкой (для индикатора события)
\usepackage{bbm} % шрифт - двойные буквы

\usepackage[colorlinks,hyperindex,unicode,breaklinks]{hyperref} % гиперссылки в pdf


\usepackage[utf8]{inputenc} % выбор кодировки файла
\usepackage[T2A]{fontenc} % кодировка шрифта
\usepackage[russian]{babel} % выбор языка

\usepackage{amssymb}
\usepackage{amsmath}
\usepackage{amsthm}
\usepackage{epsfig}
\usepackage{bm}
\usepackage{color}

\usepackage{multicol}


\usepackage{textcomp}  % Чтобы в формулах можно было русские буквы писать через \text{}

\usepackage{embedfile} % Чтобы код LaTeXа включился как приложение в PDF-файл

\usepackage{subfigure} % для создания нескольких рисунков внутри одного

\usepackage{tikz,pgfplots} % язык для рисования графики из latex'a
\usetikzlibrary{trees} % прибамбас в нем для рисовки деревьев
\usetikzlibrary{arrows} % прибамбас в нем для рисовки стрелочек подлиннее
\usepackage{tikz-qtree} % прибамбас в нем для рисовки деревьев


\usepackage{ifpdf} % чтобы проверять, запускаем мы pdflatex или просто latex

\ifpdf
	\usepackage[pdftex]{graphicx} 
	\DeclareGraphicsRule{*}{mps}{*}{} % все неупомянутые ps файлы объявляем упрощенными, т.е. mps типа. Просто ps графику нельзя использовать, но без некоторых спец. команд - можно. Например, результат работы metapost - это ps файлы простого (mps) типа. Собственно ради использования metapost эта строка и введена.
\else
	\usepackage{graphicx}
\fi



% конец добавки

\usepackage{asymptote} % After graphicx!, пакет для рисования графиков и прочего
%\usepackage{sagetex} % i suppose after graphicx also..., для связи с sage



\embedfile[desc={Исходный LaTeX файл}]{\jobname.tex} % Включение кода в выходной файл
\embedfile[desc={Стилевой файл}]{/home/boris/science/tex_general/title_bor_utf8.tex}



% вместо горизонтальной делаем косую черточку в нестрогих неравенствах
\renewcommand{\le}{\leqslant}
\renewcommand{\ge}{\geqslant} 
\renewcommand{\leq}{\leqslant}
\renewcommand{\geq}{\geqslant}

% делаем короче интервал в списках 
\setlength{\itemsep}{0pt} 
\setlength{\parskip}{0pt} 
\setlength{\parsep}{0pt}

% свешиваем пунктуацию (т.е. знаки пунктуации могут вылезать за правую границу текста, при этом текст выглядит ровнее)
\usepackage{microtype}

% более красивые таблицы
\usepackage{booktabs}
% заповеди из докупентации: 
% 1. Не используйте вертикальные линни
% 2. Не используйте двойные линии
% 3. Единицы измерения - в шапку таблицы
% 4. Не сокращайте .1 вместо 0.1
% 5. Повторяющееся значение повторяйте, а не говорите "то же"


% DEFS
\def \mbf{\mathbf}
\def \msf{\mathsf}
\def \mbb{\mathbb}
\def \tbf{\textbf}
\def \tsf{\textsf}
\def \ttt{\texttt}
\def \tbb{\textbb}

\def \wh{\widehat}
\def \wt{\widetilde}
\def \ni{\noindent}
\def \ol{\overline}
\def \cd{\cdot}
\def \bl{\bigl}
\def \br{\bigr}
\def \Bl{\Bigl}
\def \Br{\Bigr}
\def \fr{\frac}
\def \bs{\backslash}
\def \lims{\limits}
\def \arg{{\operatorname{arg}}}
\def \dist{{\operatorname{dist}}}
\def \VC{{\operatorname{VCdim}}}
\def \card{{\operatorname{card}}}
\def \sgn{{\operatorname{sign}\,}}
\def \sign{{\operatorname{sign}\,}}
\def \xfs{(x_1,\ldots,x_{n-1})}
\def \Tr{{\operatorname{\mbf{Tr}}}}
\DeclareMathOperator*{\argmin}{arg\,min}
\DeclareMathOperator*{\argmax}{arg\,max}
\DeclareMathOperator*{\amn}{arg\,min}
\DeclareMathOperator*{\amx}{arg\,max}
\def \cov{{\operatorname{Cov}}}

\def \xfs{(x_1,\ldots,x_{n-1})}
\def \ti{\tilde}
\def \wti{\widetilde}


\def \mL{\mathcal{L}}
\def \mW{\mathcal{W}}
\def \mH{\mathcal{H}}
\def \mC{\mathcal{C}}
\def \mE{\mathcal{E}}
\def \mN{\mathcal{N}}
\def \mA{\mathcal{A}}
\def \mB{\mathcal{B}}
\def \mU{\mathcal{U}}
\def \mV{\mathcal{V}}
\def \mF{\mathcal{F}}

\def \R{\mbb R}
\def \N{\mbb N}
\def \Z{\mbb Z}
\def \P{\mbb{P}}
%\def \p{\mbb{P}}
\def \E{\mbb{E}}
\def \D{\msf{D}}
\def \I{\mbf{I}}

\def \a{\alpha}
\def \b{\beta}
\def \t{\tau}
\def \dt{\delta}
\def \e{\varepsilon}
\def \ga{\gamma}
\def \kp{\varkappa}
\def \la{\lambda}
\def \sg{\sigma}
\def \sgm{\sigma}
\def \tt{\theta}
\def \ve{\varepsilon}
\def \Dt{\Delta}
\def \La{\Lambda}
\def \Sgm{\Sigma}
\def \Sg{\Sigma}
\def \Tt{\Theta}
\def \Om{\Omega}
\def \om{\omega}


\def \ni{\noindent}
\def \lq{\glqq}
\def \rq{\grqq}
\def \lbr{\linebreak}
\def \vsi{\vspace{0.1cm}}
\def \vsii{\vspace{0.2cm}}
\def \vsiii{\vspace{0.3cm}}
\def \vsiv{\vspace{0.4cm}}
\def \vsv{\vspace{0.5cm}}
\def \vsvi{\vspace{0.6cm}}
\def \vsvii{\vspace{0.7cm}}
\def \vsviii{\vspace{0.8cm}}
\def \vsix{\vspace{0.9cm}}
\def \VSI{\vspace{1cm}}
\def \VSII{\vspace{2cm}}
\def \VSIII{\vspace{3cm}}


\newcommand{\grad}{\mathrm{grad}}
\newcommand{\dx}[1]{\,\mathrm{d}#1} % для интеграла: маленький отступ и прямая d
\newcommand{\ind}[1]{\mathbbm{1}_{\{#1\}}} % Индикатор события
%\renewcommand{\to}{\rightarrow}
\newcommand{\eqdef}{\mathrel{\stackrel{\rm def}=}}
\newcommand{\iid}{\mathrel{\stackrel{\rm i.\,i.\,d.}\sim}}
\newcommand{\const}{\mathrm{const}}

%на всякий случай пока есть
%теоремы без нумерации и имени
%\newtheorem*{theor}{Теорема}

%"Определения","Замечания"
%и "Гипотезы" не нумеруются
%\newtheorem*{defin}{Определение}
%\newtheorem*{rem}{Замечание}
%\newtheorem*{conj}{Гипотеза}

%"Теоремы" и "Леммы" нумеруются
%по главам и согласованно м/у собой
%\newtheorem{theorem}{Теорема}
%\newtheorem{lemma}[theorem]{Лемма}

% Утверждения нумеруются по главам
% независимо от Лемм и Теорем
%\newtheorem{prop}{Утверждение}
%\newtheorem{cor}{Следствие}


%\emergencystretch=2em \voffset=-2cm \hoffset=-2cm
%\unitlength=0.6mm \textwidth=18cm \textheight=26cm


\begin{document}
\parindent=0 pt % отступ равен 0

Лекция по байесовским сетям.

\section{Что такое байесовская сеть?}

Здесь картинка.

На картинке:
\begin{enumerate}
\item Кружочками обозначаются случайные величины.
\item Стрелочками --- причинно следственные связи:

\begin{tikzpicture}[grow=right]
\tikzstyle{mycircle} = [circle, draw, minimum width=16pt, inner sep=0pt] % node style
\tikzstyle{edge from parent}=
     [-angle 45,draw, % рисуем стрелочку
     edge from parent path={(\tikzparentnode) -- (\tikzchildnode)}] % это магическое заклинание направляющее ребра к центру узла, а не к точке под узлом:     
\tikzstyle{every node}=[mycircle]
\node {$X$} % создаем узел 
    child { node {$Y$} } ;
\end{tikzpicture}

Значение величины $X$ становится известно раньше значиня $Y$. Закон распределения величины $Y$ зависит от значения величины $X$.
\end{enumerate}

Несколько терминов:
\begin{enumerate}
\item Вилка (fork)
\item Коллайдер, перевернутая вилка (collider, inverted fork)
\item Путь\footnote{В теории графов термином путь называют то, что мы называем направленным путем} (trail, path) от $A$ до $B$ --- последовательность вершин от вершины $A$ до вершины $B$, в которой переходы могут делаться и по стрелочкам и против стрелочек
\item Направленный путь от $A$ до $B$ --- последовательность вершин от вершины $A$ до вершины $B$, в которой переходы делаются только по стрелочкам
\item Потомок. Узел $Y$ называется потомком узла $X$, если существует направленный путь от $X$ до $Y$.

\end{enumerate}


По байесовской сети легко определить зависимость и условную зависимость величин. Сначала разберемся с зависимостью.

Величины $X$ и $Y$ независимы, если выполнены все три условия 
\begin{enumerate}
\item Нет направленного пути от $X$ до $Y$
\item Нет направленного пути от $Y$ до $X$
\item Не существует такой величины $Z$, от которой был бы направленный путь и до $X$ и до $Y$
\end{enumerate}


Упражнение. Найдите все пары независимых величин.

\section{Условная независимость}

Условная независимость. События $A$ и $B$ называются условно независимыми при условии, что событие $C$ произошло, если $\P(AB \mid C)=\P(A \mid C)\P(B \mid C)$

Примеры.
\begin{enumerate}
\item Независимые, но условно зависимые события.
\item Зависимые, но условно независимые события.
\item Независимы при условии $C$, зависимы при отрицании $C$
\end{enumerate}

Дискретные случайные величины $X$ и $Y$ условно независимы при условии $Z$, если для любых $x$, $y$ и $z$:
\begin{equation}
\P(X=x,Y=y \mid Z=z) = \P(X=x \mid Z=z)\cdot \P(Y=y \mid Z=z)
\end{equation}

Условную независимость величин обозначают $X \perp Y \mid Z$


Примечание: некоторые авторы пишут $A \perp B \mid C$ для событий, под этой записью подразумевается на самом деле сразу два условия:

\begin{equation}
A \perp B \mid C \Leftrightarrow 
\begin{cases}
A \mbox{ и } B \mbox{ независимы при условии } C \\
A \mbox{ и } B \mbox{ независимы при условии } C^c \\
\end{cases}
\end{equation}


Путь между $X$ и $Y$ называют $d$-разделенным (d-separated, directionally separated) множеством узлов $Z$ если выполнено хотя бы одно из условий
\begin{enumerate}
\item узел из $Z$ разрывает последовательное соединение на пути
\item узел из $Z$ разрывает <<вилку>> на пути
\item на пути есть <<коллайдер>>, не являющийся узлом из $Z$ и не содержащий узел из $Z$ в качестве одного из потомков
\end{enumerate}


Можно эквивалентно говорить о том, что между $X$ и $Y$ НЕ является $d$-разделенным узлом $Z$, если выполнены оба условия:
\begin{enumerate}
\item любой коллайдер на пути либо сам является узлом из множества $Z$, либо имеет потомка из множества $Z$
\item никакой другой узел на пути не входит в множество $Z$
\end{enumerate}


Случайные величины $X$ и $Y$ условно независимы при условии $Z$, если узлы $X$ и $Y$ являются $d$-разделенными узлом $Z$.


Упражнения
\begin{tikzpicture}[grow=up]
\tikzstyle{mycircle} = [circle, draw, minimum width=16pt, inner sep=0pt] % node style
\tikzstyle{edge from parent}=
     [angle 45-,draw, % рисуем стрелочку
     edge from parent path={(\tikzparentnode) -- (\tikzchildnode)}] % это магическое заклинание направляющее ребра к центру узла, а не к точке под узлом:     
\tikzstyle{every node}=[mycircle]
\node {$X_8$} % создаем узел 
    child { node {$X_6$}
	    child { node {$X_4$}        
		    child { node {$X_1$} } }
	    child { node {$X_5$}        
		    child { node {$X_2$} }
		    child { node {$X_3$} } } }
    child { node {$X_7$} } ;
\end{tikzpicture}

Проверьте независимость $X_1 \perp X_2$, $X_1 \perp X_2 \mid X_8$, $X_1 \perp X_2 \mid X_7$, $X_1 \perp X_2 \mid X_6$


\end{document}