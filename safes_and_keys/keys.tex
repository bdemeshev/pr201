\documentclass[pdftex,12pt,a4paper]{article}


% jan 2012

% sudo yum install texlive-bbm texlive-bbm-macros texlive-asymptote texlive-cm-super texlive-cyrillic texlive-pgfplots texlive-subfigure
% yum install texlive-chessboard texlive-skaknew % for \usepackage{chessboard}
% yum install texlive-minted texlive-navigator texlive-yax texlive-texapi

% растягиваем границы страницы
%\emergencystretch=2em \voffset=-2cm \hoffset=-1cm
%\unitlength=0.6mm \textwidth=17cm \textheight=25cm

\usepackage{makeidx} % для создания предметных указателей
\usepackage{verbatim} % для многострочных комментариев
\usepackage{cmap} % для поиска русских слов в pdf
\usepackage[pdftex]{graphicx} % для вставки графики 
% omit pdftex option if not using pdflatex


%\usepackage{dsfont} % шрифт для единички с двойной палочкой (для индикатора события)
\usepackage{bbm} % шрифт - двойные буквы

\usepackage[colorlinks,hyperindex,unicode,breaklinks]{hyperref} % гиперссылки в pdf


\usepackage[utf8]{inputenc} % выбор кодировки файла
\usepackage[T2A]{fontenc} % кодировка шрифта
\usepackage[russian]{babel} % выбор языка

\usepackage{amssymb}
\usepackage{amsmath}
\usepackage{amsthm}
\usepackage{epsfig}
\usepackage{bm}
\usepackage{color}

\usepackage{multicol}


\usepackage{textcomp}  % Чтобы в формулах можно было русские буквы писать через \text{}

\usepackage{embedfile} % Чтобы код LaTeXа включился как приложение в PDF-файл

\usepackage{subfigure} % для создания нескольких рисунков внутри одного

\usepackage{tikz,pgfplots} % язык для рисования графики из latex'a
\usetikzlibrary{trees} % прибамбас в нем для рисовки деревьев
\usetikzlibrary{arrows} % прибамбас в нем для рисовки стрелочек подлиннее
\usepackage{tikz-qtree} % прибамбас в нем для рисовки деревьев


\usepackage{ifpdf} % чтобы проверять, запускаем мы pdflatex или просто latex

\ifpdf
	\usepackage[pdftex]{graphicx} 
	\DeclareGraphicsRule{*}{mps}{*}{} % все неупомянутые ps файлы объявляем упрощенными, т.е. mps типа. Просто ps графику нельзя использовать, но без некоторых спец. команд - можно. Например, результат работы metapost - это ps файлы простого (mps) типа. Собственно ради использования metapost эта строка и введена.
\else
	\usepackage{graphicx}
\fi



% конец добавки

\usepackage{asymptote} % After graphicx!, пакет для рисования графиков и прочего
%\usepackage{sagetex} % i suppose after graphicx also..., для связи с sage



\embedfile[desc={Исходный LaTeX файл}]{\jobname.tex} % Включение кода в выходной файл
\embedfile[desc={Стилевой файл}]{/home/boris/science/tex_general/title_bor_utf8.tex}



% вместо горизонтальной делаем косую черточку в нестрогих неравенствах
\renewcommand{\le}{\leqslant}
\renewcommand{\ge}{\geqslant} 
\renewcommand{\leq}{\leqslant}
\renewcommand{\geq}{\geqslant}

% делаем короче интервал в списках 
\setlength{\itemsep}{0pt} 
\setlength{\parskip}{0pt} 
\setlength{\parsep}{0pt}

% свешиваем пунктуацию (т.е. знаки пунктуации могут вылезать за правую границу текста, при этом текст выглядит ровнее)
\usepackage{microtype}

% более красивые таблицы
\usepackage{booktabs}
% заповеди из докупентации: 
% 1. Не используйте вертикальные линни
% 2. Не используйте двойные линии
% 3. Единицы измерения - в шапку таблицы
% 4. Не сокращайте .1 вместо 0.1
% 5. Повторяющееся значение повторяйте, а не говорите "то же"


% DEFS
\def \mbf{\mathbf}
\def \msf{\mathsf}
\def \mbb{\mathbb}
\def \tbf{\textbf}
\def \tsf{\textsf}
\def \ttt{\texttt}
\def \tbb{\textbb}

\def \wh{\widehat}
\def \wt{\widetilde}
\def \ni{\noindent}
\def \ol{\overline}
\def \cd{\cdot}
\def \bl{\bigl}
\def \br{\bigr}
\def \Bl{\Bigl}
\def \Br{\Bigr}
\def \fr{\frac}
\def \bs{\backslash}
\def \lims{\limits}
\def \arg{{\operatorname{arg}}}
\def \dist{{\operatorname{dist}}}
\def \VC{{\operatorname{VCdim}}}
\def \card{{\operatorname{card}}}
\def \sgn{{\operatorname{sign}\,}}
\def \sign{{\operatorname{sign}\,}}
\def \xfs{(x_1,\ldots,x_{n-1})}
\def \Tr{{\operatorname{\mbf{Tr}}}}
\DeclareMathOperator*{\argmin}{arg\,min}
\DeclareMathOperator*{\argmax}{arg\,max}
\DeclareMathOperator*{\amn}{arg\,min}
\DeclareMathOperator*{\amx}{arg\,max}
\def \cov{{\operatorname{Cov}}}

\def \xfs{(x_1,\ldots,x_{n-1})}
\def \ti{\tilde}
\def \wti{\widetilde}


\def \mL{\mathcal{L}}
\def \mW{\mathcal{W}}
\def \mH{\mathcal{H}}
\def \mC{\mathcal{C}}
\def \mE{\mathcal{E}}
\def \mN{\mathcal{N}}
\def \mA{\mathcal{A}}
\def \mB{\mathcal{B}}
\def \mU{\mathcal{U}}
\def \mV{\mathcal{V}}
\def \mF{\mathcal{F}}

\def \R{\mbb R}
\def \N{\mbb N}
\def \Z{\mbb Z}
\def \P{\mbb{P}}
%\def \p{\mbb{P}}
\def \E{\mbb{E}}
\def \D{\msf{D}}
\def \I{\mbf{I}}

\def \a{\alpha}
\def \b{\beta}
\def \t{\tau}
\def \dt{\delta}
\def \e{\varepsilon}
\def \ga{\gamma}
\def \kp{\varkappa}
\def \la{\lambda}
\def \sg{\sigma}
\def \sgm{\sigma}
\def \tt{\theta}
\def \ve{\varepsilon}
\def \Dt{\Delta}
\def \La{\Lambda}
\def \Sgm{\Sigma}
\def \Sg{\Sigma}
\def \Tt{\Theta}
\def \Om{\Omega}
\def \om{\omega}


\def \ni{\noindent}
\def \lq{\glqq}
\def \rq{\grqq}
\def \lbr{\linebreak}
\def \vsi{\vspace{0.1cm}}
\def \vsii{\vspace{0.2cm}}
\def \vsiii{\vspace{0.3cm}}
\def \vsiv{\vspace{0.4cm}}
\def \vsv{\vspace{0.5cm}}
\def \vsvi{\vspace{0.6cm}}
\def \vsvii{\vspace{0.7cm}}
\def \vsviii{\vspace{0.8cm}}
\def \vsix{\vspace{0.9cm}}
\def \VSI{\vspace{1cm}}
\def \VSII{\vspace{2cm}}
\def \VSIII{\vspace{3cm}}


\newcommand{\grad}{\mathrm{grad}}
\newcommand{\dx}[1]{\,\mathrm{d}#1} % для интеграла: маленький отступ и прямая d
\newcommand{\ind}[1]{\mathbbm{1}_{\{#1\}}} % Индикатор события
%\renewcommand{\to}{\rightarrow}
\newcommand{\eqdef}{\mathrel{\stackrel{\rm def}=}}
\newcommand{\iid}{\mathrel{\stackrel{\rm i.\,i.\,d.}\sim}}
\newcommand{\const}{\mathrm{const}}

%на всякий случай пока есть
%теоремы без нумерации и имени
%\newtheorem*{theor}{Теорема}

%"Определения","Замечания"
%и "Гипотезы" не нумеруются
%\newtheorem*{defin}{Определение}
%\newtheorem*{rem}{Замечание}
%\newtheorem*{conj}{Гипотеза}

%"Теоремы" и "Леммы" нумеруются
%по главам и согласованно м/у собой
%\newtheorem{theorem}{Теорема}
%\newtheorem{lemma}[theorem]{Лемма}

% Утверждения нумеруются по главам
% независимо от Лемм и Теорем
%\newtheorem{prop}{Утверждение}
%\newtheorem{cor}{Следствие}


%\usepackage{showkeys} % показывать метки

% специальная штука под задачник
% создает команды:
% \problem{ текст задачи }
% \solution{ текст решения }
% \problemonly  - после этой команды будут печататься только \problem{} и \problemtext{}
% \solutiononly - после этой команды будут печататься только \solution{} и \solutiontext{}
% \problemandsolution - после этой команды печатается все
% \secsolution - задает новую (виртуальную) секцию для решений

% может потребоваться %\addtocounter{secsolution}{число глав без задач решений, не прогнанных через problemonly}

% как работать
% файл с решениями отдельной главы должен выглядеть так:
% \problem{ dddd} \solution{ddddddd}
% \problem{df sldk} \solution{ dfssd}

% главный файл может выглядеть двумя способами:

% Способ 1. (для решений контрольной, рядом задачи и ответы)
% \problemandsolution
% \input{file with problems}

% Способ 2. (для задачника, сначала все задачи, затем все ответы)
% \problemonly
% \input{file with problems}

% \solutiononly
% \input{file with problems}

% ААААААААААААААААААААААА надо делать!!!!
% Способ 3. - основной (вариант способа 2)

% \problemonly2
% \input{file with problems}

%\solutiononly - эта команда сама сделает все!




% файл с задачами:
% \section{Первая}
% \problem{ dddd} \solution{ddddddd}
% \problem{df sldk} \solution{ dfssd}
% \problemtext{Этот текст не будет напечатан после solutiononly}
% \section{Вторая}
% \problem{ ааа} \solution{dыва}
% \problem{ыавыв} \solution{ ыва}
% \solutiontext{Этот текст не будет напечатан после problemonly}



% начало кода:

\let\oldsection\section % сохраняем команду \section, т.к. мы ее переопределим
\let\oldsubsection\subsection % сохраняем команду \subsection, т.к. мы ее переопределим

\newcommand{\restoresection}{ % команда для восстановления \section \subsection
\renewcommand{\section}[1]{\oldsection{##1}}
\renewcommand{\subsection}[1]{\oldsubsection{##1}}
}

\newcounter{problem}[section]
%создаем новый счетчик "problem",
% будет автоматом сбрасываться на 0 при старте нового раздела
% при создании счетчик сам встанет на 0

\newcounter{secsolution}
% - это номер секции решаемой задачи (поскольку решения идут в одной секции, то номер секции надо менять в ручную)
\newcounter{solution}[secsolution]
% - это номер решаемой задачи, сам сбрасывается при увеличении secsolution на 1



\renewcommand{\thesecsolution}{\arabic{secsolution}}
% команда \thesecsolution просто выводит номер secsolution

\newcommand{\newsecsolution}{
\stepcounter{secsolution} % без создания ссылки увеличит secsolution на 1 со сбросом подчиненного счетчика
}
% команда \newsecsolution увеличит номер секции на 1 и установит номер решения внутри секции равным 0

\renewcommand{\theproblem}{\thesection.\arabic{problem}.}
\renewcommand{\thesolution}{\thesecsolution.\arabic{solution}.}
% обновляем команду \theproblem - она должна выводить номер секции и номер задачи внутри секции
% почему обновляем? - потому, что она создалась при создании счетчика problem

\newcommand{\problem}[1]{}
\newcommand{\solution}[1]{}
% создаем команды \problem, \solution с одним аргументом, которые ничего не делает
% ниже они будут переопределены


\newcommand{\problemtext}[1]{}
\newcommand{\solutiontext}[1]{}
% эти две команды будут выводить текст, заложенный внутри них только внутри соответствующей секции, в другой - ничего не будет делать
% в отличие от этой команды \problem \solution делают ссылки, слово "задача" и пр.


\newcommand{\problemonly}{
% эта команда переопределяет команду \problem

\setcounter{problem}{0}
\setcounter{solution}{0}
\setcounter{secsolution}{0}

\renewcommand{\problemtext}[1]{##1}
\renewcommand{\solutiontext}[1]{}


\renewcommand{\section}[1]{\oldsection{##1}\newsecsolution}
\renewcommand{\subsection}[1]{\oldsubsection{##1}}


\renewcommand{\problem}[1]{
\refstepcounter{problem}
% \phantomsection % создаем точку привязки для команды \label % не нужна, т.к. есть refstepcounter
\vspace{0.5ex plus 0.2ex minus 0.2ex}
Задача
\hyperref[s\theproblem]{\theproblem} % гиперссылка на метку "s1.1."
\label{p\theproblem} % метка "p1.1."
\par ##1}
\renewcommand{\solution}[1]{}
}

\newcommand{\solutiononly}{
% эта команда переопределяет команду \solution

\setcounter{problem}{0}
\setcounter{solution}{0}
\setcounter{secsolution}{0}

\renewcommand{\problemtext}[1]{}
\renewcommand{\solutiontext}[1]{##1}

\renewcommand{\section}[1]{\newsecsolution} % можно сюда чего-то добавить, чтобы решения отделялись как-то по секциям
\renewcommand{\subsection}[1]{}

\renewcommand{\problem}[1]{}
\renewcommand{\solution}[1]{
\refstepcounter{solution}
% \phantomsection
\hyperref[p\thesolution]{\thesolution} \label{s\thesolution}
##1}}



\newcommand{\problemandsolution}{
% эта команда переопределяет команды \solution, \problem

\setcounter{problem}{0}
\setcounter{solution}{0}
\setcounter{secsolution}{0}

\renewcommand{\problemtext}[1]{##1}
\renewcommand{\solutiontext}[1]{##1}

\renewcommand{\section}[1]{\oldsection{##1}\newsecsolution}
\renewcommand{\subsection}[1]{\oldsubsection{##1}}

\renewcommand{\problem}[1]{
\refstepcounter{problem}
% \phantomsection % создаем точку привязки для команды \label
Задача
\hyperref[s\theproblem]{\theproblem} % гиперссылка на метку "s1.1."
\label{p\theproblem} % метка "p1.1."
\par ##1}
\renewcommand{\solution}[1]{
\refstepcounter{solution}
% \phantomsection
\hyperref[p\thesolution]{\thesolution} \label{s\thesolution}
##1}
}




%\title{Задачи по элементарной теории вероятностей и матстатистике}
%\author{Составитель: Борис Демешев, boris.demeshev@gmail.com}
%\date{\today}
\title{Вокруг задачи про ключи и сейфы}
\author{Борис Демешев}

\date{\today}


\begin{document}

\maketitle

\parindent=0 pt % отступ равен 0


\bibliographystyle{plain} % стиль оформления ссылок

\begin{abstract}
Задача про ключи и сейфы, ее решения и вариации на тему.
\end{abstract}



\section{Постановка задачи и первые попытки}
В ряд один за одним стоят 100 сейфов. Закрыть открытый сейф можно без ключа. Достать содержимое закрытого сейфа можно только открыв его ключом. К каждому сейфу подходит ровно один ключ. 

Мы раскладываем ключи по сейфам наугад, один ключ в один сейф. Затем закрываем все сейфы кроме первых двух. А затем пытаемся открыть все сейфы.

Какова вероятность того, что мы сможем это сделать? \hyperref[lojniiseif]{Туда!}

Вперед! Два открытых сейфа сразу дают нам два ключа. Этими ключами можно открыть два новых сейфа... Хотя нет, необязательно, один или оба ключа могут быть от уже раскрытых сейфов. Значит с начала три варианта: два полезных ключа, один полезный, ни одного полезного. Дальше число вариантов быстро возрастает. С налету задача не решилась, видно, что дерево возможных исходов слишком ветвистое. \cite{aops:keys}

% Здесь нужно посадить дерево!

Что ж! Попробуем решить похожую задачу, но попроще. Что будет если закрывать все сейфы кроме первого? Шансы, что первый ключ будет полезным равны $\frac{99}{100}$. Если первый добытый ключ полезный, то теперь у нас 99 сейфов и один открыт. Значит условная вероятность получить второй полезный ключ (при условии, что первый полезный) равна $\frac{98}{99}$. Продолжаем и получаем, что:
$$ 
p_{1,100}=\frac{99}{100}\cdot\frac{98}{99}\cdot...\cdot\frac{1}{2}=\frac{1}{100}
$$
Попробуем еще одну упрощенную версию. Что будет если закрыть только последний сейф? Единственная ситуация, когда мы не сможем его открыть, если ключ от него оказался в нем самом. Вероятность такого <<запертого>> ключа равна $\frac{1}{100}$. Значит, вероятность успеха равна $\frac{99}{100}$.

Уже неплохо! Попробуем рассмотреть совсем крайние случаи! Если изначально не открыт ни один сейф, то $p_{0,100}=0$, а если открыты все, то $p_{100,100}=1$. 

Возникает смутная догадка, а вдруг $p_{k,n}=\frac{k}{n}$? Пробуем проверить догадку на еще одном простом примере - всего 4 сейфа, из которых закрывают 2. Всего вариантов распределения ключей $4!$. Тупиковыми оказываются три типа расположения ключей. Во-первых, ситуации, где третий сейф содержит ключ от самого себя, таких ситуаций будет $3!$. Во-вторых, ситуации, где четвертый сейф содержит ключ от самого себя, таких ситуаций будет $3!$, из них 2 совпадают с теми, где заперт ключ от третьего сейфа.
И в-третьих, ситуации, где третий содержит ключ от четвертого, а четвертый - от третьего, таких ситуаций две. Значит вероятность открыть все сейфы равна: 
$$
p_{2,4}=\frac{4!-(3!+3!-2!)-2!}{4!}=\frac{1}{2}
$$
Укрепившись в нашей догадке мы готовы бороться за решение. Искать решение задачи проще, если знаешь ответ!

\section{Метод первого шага}
Разобьем одну копилку и проследим за судьбой извлеченного ключа. С
вероятностью $\frac{k}{n}$ ключ не дает ничего нового, так как
предназначен для копилки, которую мы и так разобьем, а с
вероятностью $\frac{n-k}{n}$ <<разбивает>>
еще одну копилку. \\
Получаем уравнение: $p(k,n)=\frac{k}{n}p(k-1,n-1)+\frac{n-k}{n}p(k,n-1)$. \\
С очевидными краевыми условиями $p(n,n)=1$, $p(0,n)=0$. \\
Угадываем решение, $p(k,n)=\frac{k}{n}$. \\

Данный метод позволяет также получить ответ на вопрос:
Сколько в среднем можно добыть ключей? Или: сколько сейфов в среднем окажутся в итоге открытыми?

По аналогии с нахождением вероятности получаем уравнение: \\
$e(k,n)=1+\frac{k}{n}e(k-1,n-1)+\frac{n-k}{n}e(k,n-1)$. \\
Граничные условия на этот раз имеют вид: $e(n,n)=n$, $e(0,n)=0$ \\
После экспериментов с мелкими $k$ и $n$ можно попробовать $\frac{k(n+1)}{k+1}$. \\

\section{Метод разложения в сумму}

Для нахождения математического ожидания в каком-нибудь непростом случае часто помогает очень простое свойство $E(X+Y)=E(X)+E(Y)$. 

?????

\phantomsection
Добавим в наши ряды <<ложный сейф>>. \label{lojniiseif} Расположим их в ??? круг. Открытыми сейфами круг разбит на $k+1$ часть. Средняя длина каждой $\frac{n+1}{k+1}$. Из них открывается $k$. Бред полный, но додумать наверняка можно!!!



????

Какова ожидаемая доля разбитых копилок? \\
Какова вероятность получить конкретный ключ, если разбивать $k$ копилок наугад? \\
Какова вероятность получить содержимое конкретной копилки, если разбивать $k$ копилок наугад? \\

Ожидаемая доля вскрытых копилок легко получается из среднего
количества вскрытых копилок, $\frac{k(n+1)}{n(k+1)}$. \\
Вероятность, также равна этому числу. Если $X_{i}$ - индикатор
того, взят ли $i$-ый ключ, то вероятность получить отдельный ключ
равна $P(X_{i}=1)=E(X_{i})$, а cреднее количество добытых ключей
равно $E(X_{1}+X_{2}+...+X_{n})=nP(X_{i}=1)$.
Следовательно, искомая вероятность равна $\frac{k(n+1)}{n(k+1)}$. \\





\section{Комбинаторика и группа перестановок}
Что происходит, если начать с одного открытого сейфа? Достаем новый ключ. Открываем новый сейф. Снова достаем новый ключ... Рано или поздно добытый ключ окажется бесполезным и будет вести в уже открытый сейф. Точнее (и это важно!) в изначальный! Почему? Иначе оказалось бы, что к некоторому сейфу подходит два ключа: тот, которым он был открыт в первый раз, и первый бесполезный ключ. Получается, что ключи добываются <<циклами>>.

Есть еще одна формулировка этого факта. События <<получен ключ от $i$-ого сейфа>> и <<получено содержимое $i$-ого сейфа>> совпадают. Казалось бы, только из первого следует второе. Ведь первые $k$ сейфов изначально открыты. Однако каждый цикл ключей получается
целиком, поэтому если $i$-ый сейф копилка была открыт изначально, то
получение ключей окончится только если будет получен ключ от
$i$-ого сейфа. \\


Имеется $n!$ способов разложить ключи по копилкам. \\
Есть $(n-1)!$ способов разложить ключи по копилкам так, чтобы был
ровно один цикл. \\
Зафиксируем открываемые $k$ копилок. Оказывается, что число
способов разложить ключи так, чтобы образовывалось $s$ вскрываемых
циклов, также равно $(n-1)!$. Можно обойтись без громоздких сумм с
биномиальными коэффициентами. Заметим, что $s\le k$. \\
Среди $k$ открываемых копилок выберем $s$ <<граничных>> копилок.
Рассмотрим произвольную раскладку ключей, в которой имеется ровно
один цикл. Граничные копилки <<разбивают>> цикл на $s$ циклов.
Получается взаимно однозначное соответствие между раскладками
ключей, в которых имеется $s$ циклов и раскладками ключей с одним
циклом. Однозначность получается при фиксированных граничных
копилках. \\
Нам подходят те раскладки ключей, в которых образуется от 1 до $k$
вскрываемых циклов. Получаем вероятность, равную
$\frac{k(n-1)!}{n!}$ \\

\section{Перепостановка эксперимента}
Бывает, что вероятность или математическое ожидание легче посчитать, если представить себе другой (но равносильный) случайный эксперимент.

Например, есть такая классическая задача. Тоже, кстати, про ключи! Есть десять ключей. К замку подходит лишь один из них. Мы их пробуем в случайном порядке, не подошедшие откладываем в сторону. Какова вероятность того, что замок откроется десятым ключом?

Людям впервые столкнувшимся с теорией вероятностей ответ $1/10$ часто кажется неправдоподобным. Они говорят: <<Это вероятность для первого ключа! А для десятого она другая, ведь возможно, что мы уже откроем дверь каким-нибудь предыдущим ключом!>>.

Попробуем однако поставить другой эксперимент. Представим, что ключи - это не ключи, а шарики. Представим также, что на земле есть узкая канавка, шириной в один шарик, в которую все шарики скатяться. У канавки также есть направление, скажем, стрелочка рядом нарисована. Один шарик помечен крестиком. Потрясем мешок с шариками, чтобы они все хорошо перемешались, а затем раскроем мешок над нашей канавкой. Шарики покатаются и остановятся, упорядочившись в нашей канавке. Какова вероятность, что самый последний шарик (по стрелочке) - меченый? Очевидно, что она равна $1/10$. Осталось только сказать, что задачи абсолютно идентичны, так как есть взаимно однозначное соответствие между порядком шариков в канавке и порядком опробования ключей. Представьте! Перед тем как мы начали доставать ключи, богиня случая провела эксперимент с шариками, и <<выдает>> нам ключи в том порядке, в каком шарики выстроились в линию.


\section{Теорема об остановке мартингала}
Если $X_{n}$ - мартингал и $T$ - ограниченный сверху момент остановки (т.е. существует число $m$, такое что $T<m$), то $E(X_{T})=E(X_{1})$. 




\section{ <<Золотой ключик>> и новая задача}
Задача решена, и ответ: $p_{k,n}=\frac{k}{n}$. Поставим вопрос наооборот: придумать задачу в которой ответ равен $\frac{k}{n}$!

Это, конечно же, несложно даже не отходя от наших сейфов! Например: положим наугад в один из сейфов <<золотой ключик>>, который открывает все сейфы. Затем, закроем все, кроме $k$ сейфов и попытаемся открыть все сейфы. Удача нас ждет только если нам доступен <<золотой ключик>>! А шансы раздобыть его равны $\frac{k}{n}$.

Значит, если все ключи заменить одним универсальным, то ответ не меняется! Возникает идея еще одного доказательства: а вдруг легко доказать, что при замене нескольких ключей на одну отмычку, открывающую те же сейфы, вероятность не меняется? Если это так, тогда мы по-тихоньку заменеям все ключи на отмычки, до тех пор, пока не останется <<золотой ключик>>. Вероятность в ходе процесса не меняется, а в конце равна $\frac{k}{n}$! 

У меня не получилось простого доказательства этого факта. Он тем не менее верен, как  видно из решения новой задачи: 

На столе стоят 100 копилок.
Достать содержимое копилки можно двумя способами: либо разбить
копилку, либо открыть дно специальным ключиком. К каждой копилке
подходит ровно один ключик. Сначала определенным образом ключи
связываются в несколько связок. Затем мы раскладываем связки
ключей по копилкам наугад, не более одной связки в копилку. Часть
копилок остается пустыми.
Затем разбиваем 2 копилки. \\
Какова вероятность того, что мы сможем достать все ключи? \\

Сведем эту задачу к уже решенной! Вероятность, как ни странно, не зависит от
того, какие конкретно
связки сформированы. И она снова равна $\frac{2}{100}$! \\

Решение получается путем перехода к исходной задаче. \\
Зафиксируем открываемые копилки. Можно обнаружить, что
последовательное удаление ключей, открывающих пустые копилки, не
изменяет ситуации. Если все ключи добывались до удаления
бесполезных ключей, то все ключи добываются и после удаления
бесполезных ключей. И наооборот. \\
После удаления бесполезных ключей остаются пустые копилки и
копилки с одним ключом. А такая ситуация равносильна размещению
<<золотого ключика>>. В результате вероятность снова равна
$\frac{k}{n}$. \\



\section{Напутствие...}
При решении задачи возникали разные гипотезы. Все тупиковые ветви не упомянуты, хотя хотелось бы найти хорошее описание процесса решения задачи. От начала и до конца, со всеми тупиками и ошибками. 
Гипотезы я часто провереля с помощью простой компьютерной программы, случайным образом раскладывавшей ключи. Ее не сложно написать, но я несколько раз попадался на том, что  $n!$ растет слишком быстро! Перебор всех вариантов невозможен. И при рассмотрении случайных перестановок очень важным оказывается способ, которым выбирается <<случайная>> перестановка. \\

Как выглядит ожидаемое количество взятых ключей или вероятность
получить отдельный ключ, когда речь идет о связках ключей мне не
известно. Если также просто, то это будет просто фантастика! \\


Автор задачи мне не известен. Знаю лишь, что задача была
предложена на Тайваньской национальной математической олимпиаде в
начале 2006 года. Позже она активно обсуждалась на разных студенческих форумах в интернете, например миэф, ру-мат, 

<<Задача на пиратскую вероятность>> на форуме факультета математики Волжского Государственного университета 

Активнее всего она обсуждалась на форуме artofproblemsolving. По этому поводу дельный совет, если знаешь английский, то обсуждай любую задачу на самом активном математическом форуме - на artofproblemsolving!


И напоследоак - Вариация на тему...
В здании есть $n$ эвакуационных выходов, из которых открыты $k$
штук. Началась учебная тревога. По учебной тревоге каждый из $n$
сотрудников бежит к своему выходу. Если выход открыт, то сотрудник
выбегает из здания. Если выход закрыт и сотрудник прибежал к этому
выходу первым, то он в панике возвращается вглубь здания, блуждает
по коридорам и выбегает к наугад выбираемому выходу, возможно к
своему же. Если до сотрудника у закрытого выхода кто-то уже был,
то сотрудник бежит по следам предшественника, возможно по своим же. \\
Какова вероятность того, что все сотрудники будут спасены? \\










%На мой взгляд оба решения не объясняют простоту ответа. При таком
%простом ответе хочется чего-то более наглядного. Возникла идея
%<<золотого ключа>>, от которой хотелось еще одного решения, а
%получилась еще одна задача. \\









%Links: \\
%http://www.artofproblemsolving.com/Forum/viewtopic.php?t=80325 \\
%http://mf.volsu.ru/forum/viewtopic.php?t=1679 \\

\bibliography{D:/documents/tex_general/opit} 
% название файла с коллекцией названий статей/книг

\end{document}