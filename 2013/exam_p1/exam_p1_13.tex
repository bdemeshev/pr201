\documentclass[12pt,a4paper]{article}
\usepackage[utf8]{inputenc}
\usepackage[russian]{babel}

\usepackage{amsmath}
\usepackage{amsfonts}
\usepackage{amssymb}
\usepackage{graphicx}
\usepackage[left=1cm,right=2cm,top=2cm,bottom=2cm]{geometry}


\newcommand{\E}{\mathrm{E}}
\newcommand{\Var}{\mathrm{Var}}
\newcommand{\Cov}{\mathrm{Cov}}
\newcommand{\Corr}{\mathrm{Corr}}
\renewcommand{\P}{\mathrm{P}}



\newcommand{\otvet}[5] 
{ \begin{tabular}{|p{2.5cm}|p{2.5cm}|p{2.5cm}|p{2.5cm}|p{2.5cm}|p{2cm}|}
\hline 
1) #1 & 2) #2 & 3) #3 & 4) #4 & 5) #5 & Ответ: \\ 
\hline 
\end{tabular} }

\newcommand{\lotvet}[5] 
{ \begin{tabular}{|p{11.6cm}|p{2cm}|}
\hline 
1) #1 \par
2) #2 \par
3) #3 \par
4) #4 \par
5) #5 & Ответ: \\ 
\hline 
\end{tabular} }



\begin{document}

\section*{A posse ad esse non valet cosequentia}
\begin{enumerate}
%в комментариях предполагаемые ответы

%1
\item Условная вероятность $\P(A\mid B)$ для независимых событий равна

\otvet{$\frac{\P(A)}{\P(B)}$}{$\P(A)\cdot \P(B)$}{$\frac{\P(A\cup B)}{\P(B)}$}{$\frac{\P(B)}{\P(A\cap B)}$}{$\P(A)$}

%2
\item События $A$ и $B$ называются независимыми, если

\lotvet{$\P(A\cup B)=\P(A)+\P(B)$}
{$\P(A)\cdot\P(B)=\P(A\cap B)$}
{$\P(A\cup B)=\P(A)+\P(B)-\P(A\cap B)$}
{$\P(A\cap B)=0$}
{нет верного} \\ 

%3
\item Вероятность опечатки в одном символе равна 0.01. Событие $A$ --- в слове из 5 букв будет 2 опечатки. Вероятность $P(A)$ примерно равняется 

\otvet{0.0001}{0.001}{0.0004}{0.004}{0.04}

%4
\item В урне 3 белых и 2 черных шара. Случайным образом вынимается один шар, пусть $X$ --- число вынутых черных шаров. Величина $\E(X)$ равняется

\otvet{1}{0.5}{2/3}{2/5}{1/5}

\item В урне 3 белых и 2 черных шара. Случайным образом вынимается один шар, пусть $X$ --- число вынутых черных шаров. Величина $\Var(X)$ равняется

\otvet{6/25}{1/25}{2/5}{2/3}{2/25}

%6
\item В урне 3 белых и 2 черных шара. Случайным образом вынимается один шар и откладывается в сторону, затем вынимается еще один шар. Событие $A$ --- второй шар --- черный. Вероятность $\P(A)$ равняется

\otvet{6/25}{1/25}{2/5}{2/3}{2/25}

%7
\item Если $f(x)$ --- функция плотности, то $\int_{-\infty}^{+\infty}f(u)\,du$ равен

\otvet{0}{1}{$\E(X)$}{$\Var(X)$}{$F(x)$}

% 8
\item Если $f(x)$ --- функция плотности, то $\int_{-\infty}^{x}f(u)\,du$ равен

\otvet{0}{1}{$\E(X)$}{$\Var(X)$}{$F(x)$}

%9
\item Если случайная величина $X$ нормальна $N(0,1)$ и $F(x)$ --- это ее функция распределения, то $F(4)$ примерно равняется

\otvet{0}{0.25}{0.5}{0.75}{1}

\newpage


\item Дисперсия $\Var(X)$ считается по формуле

\lotvet{$\E^2(X)$}{$\E(X^2)$}{$\E(X^2)+\E^2(X)$}{$\E(X^2)-\E^2(X)$}{$\E^2(X)-\E(X^2)$}

%11
\item Дисперсия разности случайных величин $X$ и $Y$ вычисляется по формуле

\lotvet{$\Var(X-Y)=\Var(X)-\Var(Y)$}
{$\Var(X-Y)=\Var(X)+\Var(Y)$}
{$\Var(X-Y)=\Var(X)+\Var(Y)-2\Cov(X,Y)$}
{$\Var(X-Y)=\Var(X)-\Var(Y)+2\Cov(X,Y)$}
{$\Var(X-Y)=\Var(X)-\Var(Y)-2\Cov(X,Y)$}

%12
\item Известно, что $\E(X)=1$, $\E(Y)=2$, $\Var(X)=4$, $\Var(Y)=9$, $\Corr(X,Y)=0.5$. Дисперсия $\Var(2X+3)$  равняется

\otvet{16}{8}{11}{4}{19}


%13
\item Известно, что $\E(X)=1$, $\E(Y)=2$, $\Var(X)=4$, $\Var(Y)=9$, $\Corr(X,Y)=0.5$. Дисперсия $\Cov(X,Y)$  равняется

\otvet{0.5}{18}{3}{12}{0}


%14
\item Известно, что $\E(X)=1$, $\E(Y)=2$, $\Var(X)=4$, $\Var(Y)=9$, $\Corr(X,Y)=0.5$. Дисперсия $\Corr(2X+3,1-Y)$  равняется

\otvet{1}{-1}{-0.5}{0.5}{0}

%15
\item Совместная функция распределения $F(x,y)$ двух случайных величин $X$ и $Y$ это

\lotvet{$\P(X\leq x)/ \P(Y\leq y)$}{$\P(X\leq x)\cdot \P(Y\leq y)$}
{$\P(X\leq x\mid Y\leq y)$}{$\P(X\leq x,Y\leq y)$}{$\P(X\leq x)+\P(Y\leq y)$}

%16
\item Если случайная величина $X$, имеющая функцию плотности $a(x)$, и случайная величина $Y$, имеющая функцию плотности $b(y)$, независимы, то для их совместной функции плотности  $f(x,y)$ справедливо

\lotvet{$f(x,y)=a(x)+b(y)$}{$f(x,y)=a(x)/b(y)$}{$f(x,y)=a(x)b(y)/(a(x)+b(y))$}
{$f(x,y)=a(x)\cdot b(y)$}{$f(x,y)=\E(a(X)b(Y))$}


%17
\item Случайные величины $X$ и $Y$ независимы и стандартно нормально распределены. Тогда $Z=X-2Y$ имеет распределение

\otvet{N(0,1)}{$t_2$}{N(0,1)}{N(0,2)}{U[0;2]}

%18
\item $Z_1,Z_2,...,Z_n\sim N(0,1)$. Тогда величина $\frac{Z_1}{\sqrt{\sum_{i=3}^n Z_i^2/n}}$ имеет распределение

\otvet {$N(0,1)$}{$t_n$}{$F_{1,n-2}$}{$\chi^2_n$}{$t_{n-2}$}

%19
\item Если случайная величина $X$ стандартно нормально распределенa, то случайная величина $Z=X^2$ имеет распределение   

\otvet{$N(1;0)$}{$N(0;1)$}{$F_{1,1}$}{$t_2$}{$\chi_1^{2}$}

%20
\item Если $X_1$, $X_2$, \ldots, $X_n$ независимы и равномерно распределены $U[-\sqrt{3},\sqrt{3}]$  то при $n\to\infty$ величина $\bar{X}_n$ стремится по распределению к 


\lotvet{вырожденному с $\P(X=0)=1$}
{$U[-\sqrt{3},\sqrt{3}]$}
{$U[0;1]$}
{$N(0,1)$}
{$N(0,3)$}

%21
\item Если $X_i$ независимы и имеют нормальное распределение $N(\mu;\sigma^2)$, то $\sqrt{n}(\bar{X}-\mu)/\hat{\sigma}$ имеет распределение

\otvet{$N(0;1)$}{$t_{n-1}$}{$\chi^2_{n-1}$}{$N(\mu;\sigma^2)$}{нет верного ответа}

%22
\item Последовательность оценок $\hat{\theta}_1$, $\hat{\theta}_2$, \ldots называется состоятельной, если 

\lotvet{$\E(\hat{\theta}_n)=\theta$}{$\Var(\hat{\theta}_n)\to 0$}{$\P(|\hat{\theta}_n - \theta |>t)\to 0$ для всех $t$}{$\E(\hat{\theta}_n)\to \theta$}
{$\Var(\hat{\theta}_n)\geq \Var(\hat{\theta}_{n+1})$} \\ \\

%23
\item Величины $X_1$, \ldots, $X_5$ равномерны на отрезке $[0;2a]$. Известно, что $\sum_{i=1}^5 x_i=25$. При использовании первого момента оценка методом моментов неизвестного $a$ равна

\otvet{1}{5}{10}{20}{нет верного ответа}





%24
\item При построении доверительного интервала для дисперсии по выборке из $n$ наблюдений при неизвестном ожидании используется статистика, имеющая распределение

\otvet{$N(0;1)$}{$t_{n-1}$}{$\chi^2_{n-1}$}{$\chi^2_{n}$}{$t_n$}

%25
\item Из 100 случайно выбранных человек ровно 50 ответили, что предпочитают молочный шоколад темному. Реализация 90\% доверительного интервала для предпочтения молочного шоколада равна:

\otvet{[0.4;0.6]}{[0.45;0.55]}{[0.3;0.7]}{[0.49;0.51]}{[0.48;0.52]}



%26
\item При построении доверительного интервала для отношения дисперсий по двум независимым нормальным выборкам из $n$ наблюдений каждая, используется статистика, имеющая распределение

\otvet{$F_{n-1,n-1}$}{$t_{n-1}$}{$\chi^2_{n-1}$}{$\chi^2_{n}$}{$t_n$}



%27
\item Функция правдоподобия, построенная по случайной выборке $X_1$, \ldots, $X_n$ из распределения с функцией плотности $f(x)=(\theta+1)x^{\theta}$ при $x\in [0;1]$ имеет вид

\otvet{$(\theta+1)x^{n\theta}$}{$\sum (\theta+1)x_i^{\theta}$}
{$(\theta+1)^{\sum x_i}$}{$(\sum x_i)^{\theta}$}{$(\theta+1)^n\prod x_i^{\theta}$}




%28
\item Если $P$-значение меньше уровня значимости $\alpha$, то гипотеза $H_0$: $\mu=\mu_0$

\lotvet{отвергается}{не отвергается}{отвергается только если $H_a$: $\mu \neq \mu_0$}{отвергается только если $H_a$: $\mu<\mu_0$}{недостаточно информации} \\

%29
\item \emph{Смещенной} оценкой математического ожидания по выборке независимых, одинаково распределенных случайных величин $X_1$, $X_2$, $X_3$ является оценка

\lotvet{$(X_1+X_2)/2$}{$(X_1+X_2+X_3)/3$}{$0.7X_1+0.2X_2+0.1X_3$}{$0.3X_1+0.3X_2+0.3X_3$}{$X_1+X_2-X_3$} \\ \\

%30
\item Ошибкой первого рода является 

\lotvet{Принятие неверной гипотезы}
{Отвержение основной гипотезы, когда она верна}
{Отвержение альтернативной гипотезы, когда она верна}
{Отказ от принятия любого решения}
{Необходимость пересдачи ТВ и МС}




\end{enumerate}

\newpage
\section*{Accesio cedit principali}

\begin{enumerate}
%в комментариях предполагаемые ответы

%11
\item Дисперсия разности случайных величин $X$ и $Y$ вычисляется по формуле

\lotvet{$\Var(X-Y)=\Var(X)-\Var(Y)+2\Cov(X,Y)$}
{$\Var(X-Y)=\Var(X)-\Var(Y)-2\Cov(X,Y)$}{$\Var(X-Y)=\Var(X)-\Var(Y)$}
{$\Var(X-Y)=\Var(X)+\Var(Y)$}
{$\Var(X-Y)=\Var(X)+\Var(Y)-2\Cov(X,Y)$}

%16
\item Если случайная величина $X$, имеющая функцию плотности $a(x)$, и случайная величина $Y$, имеющая функцию плотности $b(y)$, независимы, то для их совместной функции плотности  $f(x,y)$ справедливо

\lotvet{$f(x,y)=a(x)+b(y)$}{$f(x,y)=a(x)/b(y)$}{$f(x,y)=a(x)\cdot b(y)$}{$f(x,y)=\E(a(X)b(Y))$}{$f(x,y)=a(x)b(y)/(a(x)+b(y))$}



%23
\item Величины $X_1$, \ldots, $X_5$ равномерны на отрезке $[0;2a]$. Известно, что $\sum_{i=1}^5 x_i=25$. При использовании первого момента оценка методом моментов неизвестного $a$ равна

\otvet{1}{5}{10}{20}{нет верного ответа}

%9
\item Если случайная величина $X$ нормальна $N(0,1)$ и $F(x)$ --- это ее функция распределения, то $F(4)$ примерно равняется

\otvet{0}{0.25}{0.5}{0.75}{1}



%3
\item Вероятность опечатки в одном символе равна 0.01. Событие $A$ --- в слове из 5 букв будет 2 опечатки. Вероятность $P(A)$ примерно равняется 

\otvet{0.0001}{0.001}{0.0004}{0.004}{0.04}



\item В урне 3 белых и 2 черных шара. Случайным образом вынимается один шар, пусть $X$ --- число вынутых черных шаров. Величина $\Var(X)$ равняется

\otvet{6/25}{1/25}{2/5}{2/3}{2/25}

%6
\item В урне 3 белых и 2 черных шара. Случайным образом вынимается один шар и откладывается в сторону, затем вынимается еще один шар. Событие $A$ --- второй шар --- черный. Вероятность $\P(A)$ равняется

\otvet{6/25}{1/25}{2/5}{2/3}{2/25}


%12
\item Известно, что $\E(X)=1$, $\E(Y)=2$, $\Var(X)=4$, $\Var(Y)=9$, $\Corr(X,Y)=0.5$. Дисперсия $\Var(2X+3)$  равняется

\otvet{16}{8}{11}{4}{19}

%1
\item Условная вероятность $\P(A\mid B)$ для независимых событий равна

\otvet{$\frac{\P(A)}{\P(B)}$}{$\P(A)\cdot \P(B)$}{$\frac{\P(A\cup B)}{\P(B)}$}{$\frac{\P(B)}{\P(A\cap B)}$}{$\P(A)$}


%7
\item Если $f(x)$ --- функция плотности, то $\int_{-\infty}^{+\infty}f(u)\,du$ равен

\otvet{0}{1}{$\E(X)$}{$\Var(X)$}{$F(x)$}



\item Дисперсия $\Var(X)$ считается по формуле

\lotvet{$\E^2(X)$}{$\E(X^2)$}{$\E(X^2)+\E^2(X)$}{$\E(X^2)-\E^2(X)$}{$\E^2(X)-\E(X^2)$}






%13
\item Известно, что $\E(X)=1$, $\E(Y)=2$, $\Var(X)=4$, $\Var(Y)=9$, $\Corr(X,Y)=0.5$. Дисперсия $\Cov(X,Y)$  равняется

\otvet{0.5}{18}{3}{12}{0}

%26
\item При построении доверительного интервала для отношения дисперсий по двум независимым нормальным выборкам из $n$ наблюдений каждая, используется статистика, имеющая распределение

\otvet{$F_{n-1,n-1}$}{$t_{n-1}$}{$\chi^2_{n-1}$}{$\chi^2_{n}$}{$t_n$}

%14
\item Известно, что $\E(X)=1$, $\E(Y)=2$, $\Var(X)=4$, $\Var(Y)=9$, $\Corr(X,Y)=0.5$. Дисперсия $\Corr(2X+3,1-Y)$  равняется

\otvet{1}{-1}{-0.5}{0.5}{0}

% 8
\item Если $f(x)$ --- функция плотности, то $\int_{-\infty}^{x}f(u)\,du$ равен

\otvet{0}{1}{$\E(X)$}{$\Var(X)$}{$F(x)$}

%15
\item Совместная функция распределения $F(x,y)$ двух случайных величин $X$ и $Y$ это

\lotvet{$\P(X\leq x)/ \P(Y\leq y)$}{$\P(X\leq x)\cdot \P(Y\leq y)$}
{$\P(X\leq x\mid Y\leq y)$}{$\P(X\leq x,Y\leq y)$}{$\P(X\leq x)+\P(Y\leq y)$}

%24
\item При построении доверительного интервала для дисперсии по выборке из $n$ наблюдений при неизвестном ожидании используется статистика, имеющая распределение

\otvet{$N(0;1)$}{$t_{n-1}$}{$\chi^2_{n-1}$}{$\chi^2_{n}$}{$t_n$}


%17
\item Случайные величины $X$ и $Y$ независимы и стандартно нормально распределены. Тогда $Z=2X-Y$ имеет распределение

\otvet{$t_2$}{N(0,5)}{N(0,1)}{N(0,3)}{U[0;3]}



%19
\item Если случайная величина $X$ стандартно нормально распределенa, то случайная величина $Z=X^2$ имеет распределение   

\otvet{$N(1;0)$}{$N(0;1)$}{$F_{1,1}$}{$t_2$}{$\chi_1^{2}$}



%28
\item Если $P$-значение меньше уровня значимости $\alpha$, то гипотеза $H_0$: $\mu=\mu_0$

\lotvet{отвергается}{не отвергается}{отвергается только если $H_a$: $\mu \neq \mu_0$}{отвергается только если $H_a$: $\mu<\mu_0$}{недостаточно информации} \\



%21
\item Если $X_i$ независимы и имеют нормальное распределение $N(\mu;\sigma^2)$, то $\sqrt{n}(\bar{X}-\mu)/\hat{\sigma}$ имеет распределение

\otvet{$N(0;1)$}{$t_{n-1}$}{$\chi^2_{n-1}$}{$N(\mu;\sigma^2)$}{нет верного ответа}

%29
\item \emph{Смещенной} оценкой математического ожидания по выборке независимых, одинаково распределенных случайных величин $X_1$, $X_2$, $X_3$ является оценка

\lotvet{$(X_1+X_2)/2$}{$(X_1+X_2+X_3)/3$}{$0.7X_1+0.2X_2+0.1X_3$}{$0.3X_1+0.3X_2+0.3X_3$}{$X_1+X_2-X_3$} \\ 

%22
\item Последовательность оценок $\hat{\theta}_1$, $\hat{\theta}_2$, \ldots называется состоятельной, если 

\lotvet{$\E(\hat{\theta}_n)=\theta$}{$\Var(\hat{\theta}_n)\to 0$}{$\P(|\hat{\theta}_n - \theta |>t)\to 0$ для всех $t$}{$\E(\hat{\theta}_n)\to \theta$}
{$\Var(\hat{\theta}_n)\geq \Var(\hat{\theta}_{n+1})$} \\ \\









%18
\item $Z_1,Z_2,...,Z_n\sim N(0,1)$. Тогда величина $\frac{Z_1}{\sqrt{\sum_{i=3}^n Z_i^2/n}}$ имеет распределение

\otvet {$N(0,1)$}{$t_n$}{$F_{1,n-2}$}{$\chi^2_n$}{$t_{n-2}$}

%25
\item Из 100 случайно выбранных человек ровно 50 ответили, что предпочитают молочный шоколад темному. Реализация 90\% доверительного интервала для предпочтения молочного шоколада равна:

\otvet{[0.4;0.6]}{[0.45;0.55]}{[0.3;0.7]}{[0.49;0.51]}{[0.48;0.52]}







%27
\item Функция правдоподобия, построенная по случайной выборке $X_1$, \ldots, $X_n$ из распределения с функцией плотности $f(x)=(\theta+1)x^{\theta}$ при $x\in [0;1]$ имеет вид

\otvet{$(\theta+1)x^{n\theta}$}{$\sum (\theta+1)x_i^{\theta}$}
{$(\theta+1)^{\sum x_i}$}{$(\sum x_i)^{\theta}$}{$(\theta+1)^n\prod x_i^{\theta}$}



%20
\item Если $X_1$, $X_2$, \ldots, $X_n$ независимы и равномерно распределены $U[-\sqrt{3},\sqrt{3}]$  то при $n\to\infty$ величина $\bar{X}_n$ стремится по распределению к 


\lotvet{$N(0,1)$}
{$N(0,3)$}{вырожденному с $\P(X=0)=1$}
{$U[-\sqrt{3},\sqrt{3}]$}
{$U[0;1]$}

%4
\item В урне 3 белых и 2 черных шара. Случайным образом вынимается один шар, пусть $X$ --- число вынутых черных шаров. Величина $\E(X)$ равняется

\otvet{2/5}{1/5}{1}{0.5}{2/3}


%2
\item События $A$ и $B$ называются независимыми, если

\lotvet{$\P(A\cup B)=\P(A)+\P(B)$}
{$\P(A)\cdot\P(B)=\P(A\cap B)$}
{$\P(A\cup B)=\P(A)+\P(B)-\P(A\cap B)$}
{$\P(A\cap B)=0$}
{нет верного} \\ 

%30
\item Ошибкой первого рода является 

\lotvet{Принятие неверной гипотезы}
{Отвержение основной гипотезы, когда она верна}
{Отвержение альтернативной гипотезы, когда она верна}
{Отказ от принятия любого решения}
{Необходимость пересдачи ТВ и МС}




\end{enumerate}


\newpage
\section*{Ad cogitandum et agendum homo natus est}

\begin{enumerate}
%в комментариях предполагаемые ответы

%22
\item Последовательность оценок $\hat{\theta}_1$, $\hat{\theta}_2$, \ldots называется состоятельной, если 

\lotvet{$\Var(\hat{\theta}_n)\geq \Var(\hat{\theta}_{n+1})$}{$\E(\hat{\theta}_n)=\theta$}{$\Var(\hat{\theta}_n)\to 0$}{$\P(|\hat{\theta}_n - \theta |>t)\to 0$ для всех $t$}{$\E(\hat{\theta}_n)\to \theta$}
 \\ 

%6
\item В урне 3 белых и 2 черных шара. Случайным образом вынимается один шар и откладывается в сторону, затем вынимается еще один шар. Событие $A$ --- второй шар --- черный. Вероятность $\P(A)$ равняется

\otvet{2/5}{2/3}{2/25}{6/25}{1/25}

%13
\item Известно, что $\E(X)=1$, $\E(Y)=2$, $\Var(X)=4$, $\Var(Y)=9$, $\Corr(X,Y)=0.5$. Дисперсия $\Cov(X,Y)$  равняется

\otvet{0.5}{18}{3}{12}{0}

%3
\item Вероятность опечатки в одном символе равна 0.01. Событие $A$ --- в слове из 5 букв будет 2 опечатки. Вероятность $P(A)$ примерно равняется 

\otvet{0.0001}{0.001}{0.0004}{0.004}{0.04}


\item В урне 3 белых и 2 черных шара. Случайным образом вынимается один шар, пусть $X$ --- число вынутых черных шаров. Величина $\Var(X)$ равняется

\otvet{6/25}{1/25}{2/5}{2/3}{2/25}





% 8
\item Если $f(x)$ --- функция плотности, то $\int_{-\infty}^{x}f(u)\,du$ равен

\otvet{0}{1}{$\E(X)$}{$\Var(X)$}{$F(x)$}

%9
\item Если случайная величина $X$ нормальна $N(0,1)$ и $F(x)$ --- это ее функция распределения, то $F(4)$ примерно равняется

\otvet{0}{0.25}{0.5}{0.75}{1}

\newpage
\item Дисперсия $\Var(X)$ считается по формуле

\lotvet{$\E^2(X)$}{$\E(X^2)$}{$\E(X^2)+\E^2(X)$}{$\E(X^2)-\E^2(X)$}{$\E^2(X)-\E(X^2)$}

%11
\item Дисперсия разности случайных величин $X$ и $Y$ вычисляется по формуле

\lotvet{$\Var(X-Y)=\Var(X)-\Var(Y)$}
{$\Var(X-Y)=\Var(X)+\Var(Y)$}
{$\Var(X-Y)=\Var(X)+\Var(Y)-2\Cov(X,Y)$}
{$\Var(X-Y)=\Var(X)-\Var(Y)+2\Cov(X,Y)$}
{$\Var(X-Y)=\Var(X)-\Var(Y)-2\Cov(X,Y)$}


%4
\item В урне 3 белых и 2 черных шара. Случайным образом вынимается один шар, пусть $X$ --- число вынутых черных шаров. Величина $\E(X)$ равняется

\otvet{1}{0.5}{2/3}{2/5}{1/5}

%12
\item Известно, что $\E(X)=1$, $\E(Y)=2$, $\Var(X)=4$, $\Var(Y)=9$, $\Corr(X,Y)=0.5$. Дисперсия $\Var(2X+3)$  равняется

\otvet{16}{8}{11}{4}{19}


%28
\item Если $P$-значение меньше уровня значимости $\alpha$, то гипотеза $H_0$: $\mu=\mu_0$

\lotvet{отвергается}{не отвергается}{отвергается только если $H_a$: $\mu \neq \mu_0$}{отвергается только если $H_a$: $\mu<\mu_0$}{недостаточно информации} \\


%14
\item Известно, что $\E(X)=1$, $\E(Y)=2$, $\Var(X)=4$, $\Var(Y)=9$, $\Corr(X,Y)=0.5$. Дисперсия $\Corr(2X+3,1-Y)$  равняется

\otvet{1}{-1}{-0.5}{0.5}{0}

%24
\item При построении доверительного интервала для дисперсии по выборке из $n$ наблюдений при неизвестном ожидании используется статистика, имеющая распределение

\otvet{$N(0;1)$}{$t_{n-1}$}{$\chi^2_{n-1}$}{$\chi^2_{n}$}{$t_n$}

%15
\item Совместная функция распределения $F(x,y)$ двух случайных величин $X$ и $Y$ это

\lotvet{$\P(X\leq x)/ \P(Y\leq y)$}{$\P(X\leq x)\cdot \P(Y\leq y)$}
{$\P(X\leq x\mid Y\leq y)$}{$\P(X\leq x,Y\leq y)$}{$\P(X\leq x)+\P(Y\leq y)$}


%26
\item При построении доверительного интервала для отношения дисперсий по двум независимым нормальным выборкам из $n$ наблюдений каждая, используется статистика, имеющая распределение

\otvet{$F_{n-1,n-1}$}{$t_{n-1}$}{$\chi^2_{n-1}$}{$\chi^2_{n}$}{$t_n$}

%16
\item Если случайная величина $X$, имеющая функцию плотности $a(x)$, и случайная величина $Y$, имеющая функцию плотности $b(y)$, независимы, то для их совместной функции плотности  $f(x,y)$ справедливо

\lotvet{$f(x,y)=a(x)+b(y)$}{$f(x,y)=a(x)/b(y)$}{$f(x,y)=a(x)b(y)/(a(x)+b(y))$}
{$f(x,y)=a(x)\cdot b(y)$}{$f(x,y)=\E(a(X)b(Y))$}

%29
\item \emph{Смещенной} оценкой математического ожидания по выборке независимых, одинаково распределенных случайных величин $X_1$, $X_2$, $X_3$ является оценка

\lotvet{$(X_1+X_2)/2$}{$(X_1+X_2+X_3)/3$}{$0.7X_1+0.2X_2+0.1X_3$}{$0.3X_1+0.3X_2+0.3X_3$}{$X_1+X_2-X_3$} \\ 

%17
\item Случайные величины $X$ и $Y$ независимы и стандартно нормально распределены. Тогда $Z=X-2Y$ имеет распределение

\otvet{N(0,5)}{$t_2$}{N(0,1)}{N(0,3)}{U[0;2]}


%20
\item Если $X_1$, $X_2$, \ldots, $X_n$ независимы и равномерно распределены $U[-\sqrt{3},\sqrt{3}]$  то при $n\to\infty$ величина $\bar{X}_n$ стремится по распределению к 


\lotvet{вырожденному с $\P(X=0)=1$}
{$U[-\sqrt{3},\sqrt{3}]$}
{$U[0;1]$}
{$N(0,1)$}
{$N(0,3)$}

%1
\item Условная вероятность $\P(A\mid B)$ для независимых событий равна

\otvet{$\frac{\P(A)}{\P(B)}$}{$\P(A)\cdot \P(B)$}{$\frac{\P(A\cup B)}{\P(B)}$}{$\frac{\P(B)}{\P(A\cap B)}$}{$\P(A)$}

%2
\item События $A$ и $B$ называются независимыми, если

\lotvet{$\P(A\cup B)=\P(A)+\P(B)$}
{$\P(A)\cdot\P(B)=\P(A\cap B)$}
{$\P(A\cup B)=\P(A)+\P(B)-\P(A\cap B)$}
{$\P(A\cap B)=0$}
{нет верного} \\ 

\newpage
%21
\item Если $X_i$ независимы и имеют нормальное распределение $N(\mu;\sigma^2)$, то $\sqrt{n}(\bar{X}-\mu)/\hat{\sigma}$ имеет распределение

\otvet{$N(0;1)$}{$t_{n-1}$}{$\chi^2_{n-1}$}{$N(\mu;\sigma^2)$}{нет верного ответа}



%23
\item Величины $X_1$, \ldots, $X_5$ равномерны на отрезке $[0;2a]$. Известно, что $\sum_{i=1}^5 x_i=25$. При использовании первого момента оценка методом моментов неизвестного $a$ равна

\otvet{1}{5}{10}{20}{нет верного ответа}







%25
\item Из 100 случайно выбранных человек ровно 50 ответили, что предпочитают молочный шоколад темному. Реализация 90\% доверительного интервала для предпочтения молочного шоколада равна:

\otvet{[0.4;0.6]}{[0.45;0.55]}{[0.3;0.7]}{[0.49;0.51]}{[0.48;0.52]}






%27
\item Функция правдоподобия, построенная по случайной выборке $X_1$, \ldots, $X_n$ из распределения с функцией плотности $f(x)=(\theta+1)x^{\theta}$ при $x\in [0;1]$ имеет вид

\otvet{$(\theta+1)x^{n\theta}$}{$\sum (\theta+1)x_i^{\theta}$}
{$(\theta+1)^{\sum x_i}$}{$(\sum x_i)^{\theta}$}{$(\theta+1)^n\prod x_i^{\theta}$}


%18
\item $Z_1,Z_2,...,Z_n\sim N(0,1)$. Тогда величина $\frac{Z_1}{\sqrt{\sum_{i=3}^n Z_i^2/n}}$ имеет распределение

\otvet {$N(0,1)$}{$t_n$}{$F_{1,n-2}$}{$\chi^2_n$}{$t_{n-2}$}

%7
\item Если $f(x)$ --- функция плотности, то $\int_{-\infty}^{+\infty}f(u)\,du$ равен

\otvet{0}{1}{$\E(X)$}{$\Var(X)$}{$F(x)$}

%19
\item Если случайная величина $X$ стандартно нормально распределенa, то случайная величина $Z=X^2$ имеет распределение   

\otvet{$N(1;0)$}{$N(0;1)$}{$F_{1,1}$}{$t_2$}{$\chi_1^{2}$}



%30
\item Ошибкой первого рода является 

\lotvet{Принятие неверной гипотезы}
{Отвержение основной гипотезы, когда она верна}
{Отвержение альтернативной гипотезы, когда она верна}
{Отказ от принятия любого решения}
{Необходимость пересдачи ТВ и МС}




\end{enumerate}


\newpage
\section*{Fortes fortuna adjuvat}

\begin{enumerate}
%в комментариях предполагаемые ответы

\item Совместная функция распределения $F(x,y)$ двух случайных величин $X$ и $Y$ это

\lotvet{$\P(X\leq x,Y\leq y)$}{$\P(X\leq x)+\P(Y\leq y)$}{$\P(X\leq x)/ \P(Y\leq y)$}{$\P(X\leq x)\cdot \P(Y\leq y)$}
{$\P(X\leq x\mid Y\leq y)$}

%16
\item Если случайная величина $X$, имеющая функцию плотности $a(x)$, и случайная величина $Y$, имеющая функцию плотности $b(y)$, независимы, то для их совместной функции плотности  $f(x,y)$ справедливо

\lotvet{$f(x,y)=a(x)\cdot b(y)$}{$f(x,y)=\E(a(X)b(Y))$}{$f(x,y)=a(x)+b(y)$}{$f(x,y)=a(x)/b(y)$}{$f(x,y)=a(x)b(y)/(a(x)+b(y))$}




%28
\item Если $P$-значение меньше уровня значимости $\alpha$, то гипотеза $H_0$: $\mu=\mu_0$

\lotvet{отвергается}{не отвергается}{отвергается только если $H_a$: $\mu \neq \mu_0$}{отвергается только если $H_a$: $\mu<\mu_0$}{недостаточно информации} \\

%19
\item Если случайная величина $X$ стандартно нормально распределенa, то случайная величина $Z=X^2$ имеет распределение   

\otvet{$F_{1,1}$}{$t_2$}{$\chi_1^{2}$}{$N(1;0)$}{$N(0;1)$}

%20
\item Если $X_1$, $X_2$, \ldots, $X_n$ независимы и равномерно распределены $U[-\sqrt{3},\sqrt{3}]$  то при $n\to\infty$ величина $\bar{X}_n$ стремится по распределению к 


\lotvet{$U[0;1]$}
{$N(0,1)$}
{$N(0,3)$}{вырожденному с $\P(X=0)=1$}
{$U[-\sqrt{3},\sqrt{3}]$}



%29
\item \emph{Смещенной} оценкой математического ожидания по выборке независимых, одинаково распределенных случайных величин $X_1$, $X_2$, $X_3$ является оценка

\lotvet{$(X_1+X_2)/2$}{$(X_1+X_2+X_3)/3$}{$0.7X_1+0.2X_2+0.1X_3$}{$0.3X_1+0.3X_2+0.3X_3$}{$X_1+X_2-X_3$} \\ 


%4
\item В урне 3 белых и 2 черных шара. Случайным образом вынимается один шар, пусть $X$ --- число вынутых черных шаров. Величина $\E(X)$ равняется

\otvet{1}{0.5}{2/3}{2/5}{1/5}

\item В урне 3 белых и 2 черных шара. Случайным образом вынимается один шар, пусть $X$ --- число вынутых черных шаров. Величина $\Var(X)$ равняется

\otvet{2/5}{2/3}{2/25}{6/25}{1/25}

%6
\item В урне 3 белых и 2 черных шара. Случайным образом вынимается один шар и откладывается в сторону, затем вынимается еще один шар. Событие $A$ --- второй шар --- черный. Вероятность $\P(A)$ равняется

\otvet{6/25}{1/25}{2/5}{2/3}{2/25}

%3
\item Вероятность опечатки в одном символе равна 0.01. Событие $A$ --- в слове из 5 букв будет 2 опечатки. Вероятность $\P(A)$ примерно равняется 

\otvet{0.0004}{0.004}{0.04}{0.0001}{0.001}

%26
\item При построении доверительного интервала для отношения дисперсий по двум независимым нормальным выборкам из $n$ наблюдений каждая, используется статистика, имеющая распределение

\otvet{$F_{n-1,n-1}$}{$t_{n-1}$}{$\chi^2_{n-1}$}{$\chi^2_{n}$}{$t_n$}


%7
\item Если $f(x)$ --- функция плотности, то $\int_{-\infty}^{+\infty}f(u)\,du$ равен

\otvet{0}{1}{$\E(X)$}{$\Var(X)$}{$F(x)$}


%2
\item События $A$ и $B$ называются независимыми, если

\lotvet{$\P(A\cup B)=\P(A)+\P(B)-\P(A\cap B)$}
{$\P(A\cap B)=0$}
{нет верного}{$\P(A\cup B)=\P(A)+\P(B)$}
{$\P(A)\cdot\P(B)=\P(A\cap B)$}
 \\ 



%9
\item Если случайная величина $X$ нормальна $N(0,1)$ и $F(x)$ --- это ее функция распределения, то $F(4)$ примерно равняется

\otvet{0}{0.25}{0.5}{0.75}{1}


\item Дисперсия $\Var(X)$ считается по формуле

\lotvet{$\E^2(X)$}{$\E(X^2)$}{$\E(X^2)+\E^2(X)$}{$\E(X^2)-\E^2(X)$}{$\E^2(X)-\E(X^2)$}
\newpage

%11
\item Дисперсия разности случайных величин $X$ и $Y$ вычисляется по формуле

\lotvet{$\Var(X-Y)=\Var(X)+\Var(Y)-2\Cov(X,Y)$}
{$\Var(X-Y)=\Var(X)-\Var(Y)+2\Cov(X,Y)$}
{$\Var(X-Y)=\Var(X)-\Var(Y)-2\Cov(X,Y)$}{$\Var(X-Y)=\Var(X)-\Var(Y)$}
{$\Var(X-Y)=\Var(X)+\Var(Y)$}


%25
\item Из 100 случайно выбранных человек ровно 50 ответили, что предпочитают молочный шоколад темному. Реализация 90\% доверительного интервала для предпочтения молочного шоколада равна:

\otvet{[0.4;0.6]}{[0.45;0.55]}{[0.3;0.7]}{[0.49;0.51]}{[0.48;0.52]}


%12
\item Известно, что $\E(X)=1$, $\E(Y)=2$, $\Var(X)=4$, $\Var(Y)=9$, $\Corr(X,Y)=0.5$. Дисперсия $\Var(2X+3)$  равняется

\otvet{16}{8}{11}{4}{19}

%27
\item Функция правдоподобия, построенная по случайной выборке $X_1$, \ldots, $X_n$ из распределения с функцией плотности $f(x)=(\theta+1)x^{\theta}$ при $x\in [0;1]$ имеет вид

\otvet{$(\sum x_i)^{\theta}$}{$(\theta+1)^n\prod x_i^{\theta}$}{$(\theta+1)x^{n\theta}$}{$\sum (\theta+1)x_i^{\theta}$}
{$(\theta+1)^{\sum x_i}$}




%15

%17
\item Случайные величины $X$ и $Y$ независимы и стандартно нормально распределены. Тогда $Z=2X-Y$ имеет распределение

\otvet{N(0,5)}{$t_2$}{N(0,1)}{N(0,3)}{U[0;2]}

%18
\item $Z_1,Z_2,...,Z_n\sim N(0,1)$. Тогда величина $\frac{Z_1}{\sqrt{\sum_{i=3}^n Z_i^2/n}}$ имеет распределение

\otvet {$N(0,1)$}{$t_n$}{$F_{1,n-2}$}{$\chi^2_n$}{$t_{n-2}$}


%21
\item Если $X_i$ независимы и имеют нормальное распределение $N(\mu;\sigma^2)$, то $\sqrt{n}(\bar{X}-\mu)/\hat{\sigma}$ имеет распределение

\otvet{$N(0;1)$}{$t_{n-1}$}{$\chi^2_{n-1}$}{$N(\mu;\sigma^2)$}{нет верного ответа}

% 8
\item Если $f(x)$ --- функция плотности, то $\int_{-\infty}^{x}f(u)\,du$ равен

\otvet{0}{1}{$\E(X)$}{$\Var(X)$}{$F(x)$}



%22
\item Последовательность оценок $\hat{\theta}_1$, $\hat{\theta}_2$, \ldots называется состоятельной, если 

\lotvet{$\E(\hat{\theta}_n)=\theta$}{$\Var(\hat{\theta}_n)\to 0$}{$\P(|\hat{\theta}_n - \theta |>t)\to 0$ для всех $t$}{$\E(\hat{\theta}_n)\to \theta$}
{$\Var(\hat{\theta}_n)\geq \Var(\hat{\theta}_{n+1})$} \\ \\

%23
\item Величины $X_1$, \ldots, $X_5$ равномерны на отрезке $[0;2a]$. Известно, что $\sum_{i=1}^5 x_i=25$. При использовании первого момента оценка методом моментов неизвестного $a$ равна

\otvet{10}{20}{1}{5}{нет верного ответа}


%13
\item Известно, что $\E(X)=1$, $\E(Y)=2$, $\Var(X)=4$, $\Var(Y)=9$, $\Corr(X,Y)=0.5$. Дисперсия $\Cov(X,Y)$  равняется

\otvet{3}{12}{0}{0.5}{18}



%24
\item При построении доверительного интервала для дисперсии по выборке из $n$ наблюдений при неизвестном ожидании используется статистика, имеющая распределение

\otvet{$t_{n-1}$}{$\chi^2_{n-1}$}{$\chi^2_{n}$}{$t_n$}
{$N(0;1)$}







%1
\item Условная вероятность $\P(A\mid B)$ для независимых событий равна

\otvet{$\frac{\P(A)}{\P(B)}$}{$\P(A)\cdot \P(B)$}{$\frac{\P(A\cup B)}{\P(B)}$}{$\frac{\P(B)}{\P(A\cap B)}$}{$\P(A)$}




%14
\item Известно, что $\E(X)=1$, $\E(Y)=2$, $\Var(X)=4$, $\Var(Y)=9$, $\Corr(X,Y)=0.5$. Дисперсия $\Corr(2X+3,1-Y)$  равняется

\otvet{1}{-1}{-0.5}{0.5}{0}



%30
\item Ошибкой первого рода является 

\lotvet{Принятие неверной гипотезы}
{Отвержение основной гипотезы, когда она верна}
{Отвержение альтернативной гипотезы, когда она верна}
{Отказ от принятия любого решения}
{Необходимость пересдачи ТВ и МС}




\end{enumerate}







\end{document}