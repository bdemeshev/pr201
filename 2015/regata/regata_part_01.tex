\documentclass[12pt, addpoints, answers]{exam} % добавить или удалить answers в скобках, чтобы показать ответы
%\usepackage[T2A, T1]{fontenc}
%\usepackage[utf8x]{inputenc}
%\usepackage[greek, russian]{babel}
%\usepackage[OT1]{fontenc}


\usepackage{fontspec}
\usepackage{polyglossia}

\setdefaultlanguage{russian}
\setotherlanguages{english, greek}

\setmainfont[Ligatures=TeX]{Linux Libertine O}
% http://www.linuxlibertine.org/index.php?id=91&L=1

%\usepackage{mathtools}
\usepackage{comment}
\usepackage{booktabs}
\usepackage{amsmath}
\usepackage{tikz}
\usepackage{amsfonts}
\usepackage{amssymb}
\usepackage[left=2cm,right=2cm,top=2cm,bottom=2cm]{geometry}
\DeclareMathOperator{\E}{\mathbb{E}}
\DeclareMathOperator{\Var}{\mathbb{V}\mathrm{ar}}
\DeclareMathOperator{\Cov}{\mathbb{C}\mathrm{ov}}
\let\P\relax
\DeclareMathOperator{\P}{\mathbb{P}}
\newcommand{\cN}{\mathcal{N}}
\newcommand{\hbeta}{\hat{\beta}}

\usepackage{floatrow}
%\newfloatcommand{capbtabbox}{table}[][\FBwidth]

\begin{document}

\pagestyle{headandfoot}
\runningheadrule
\firstpageheader{Теория вероятностей}{Невероятная регата, индивидуальная часть}{26 сентября 2015}
\firstpageheadrule
\runningheader{Теория вероятностей}{Невероятная регата, индивидуальная часть}{26 сентября 2015}
\firstpagefooter{}{}{}
\runningfooter{}{}{}
\runningfootrule




\hqword{Задача}
\hpgword{Страница}
\hpword{Максимум}
\hsword{Баллы}
\htword{Итого}
\pointname{\%}
%\renewcommand{\solutiontitle}{\noindent\textbf{Решение:}\par\noindent}
\renewcommand{\solutiontitle}{}

%Таблица с результатами заполняется проверяющим работу. Пожалуйста, не делайте в ней пометок.

%\begin{center}
%  \gradetable[h][questions]
%\end{center}

\vspace{0.2in}

\makebox[\textwidth]{Группа, имя и фамилия:\enspace\hrulefill}

\vspace{0.2in}

\makebox[\textwidth]{Настроение:\enspace\hrulefill}


\begin{center}
\textbf{Часть 1. Каждый сам за себя!!!}
\end{center}

\begin{questions}

\question Для разминки вспомним греческий алфавит!

\begin{parts}
\part По-гречески --- Σωκρατης, а по-русски --- \fillin[Сократ].
\part Изобразите прописные и строчные буквы: эта \fillin[Η, η], дзета \fillin[Ζ, ζ], вега \fillin[нет], шо \fillin[ϸ]. Если такой буквы в греческом нет, то поставьте прочерк. 
\part Назовите буквы: τ \fillin[тау], θ \fillin[тета], ξ \fillin[кси].
%\part Если пересчитать все буквы в греческом алфавите, то их окажется ровно \fillin[24].
\end{parts}

\begin{solution}
Греческая буква шо, ϸ, была введена Александром Македонским и ныне вышла из употребления. По крайней мере, в греческом :) Заглавная примерно такая же, только её utf-код 03f7 не поддерживается шрифтом Linux Libertine.
\end{solution}

\question Подбрасываются 2 симметричные монеты. Событие $A$ --- на первой монете выпал герб, событие $B$ --- на второй монете выпал герб, событие $C$ --- монеты выпали разными сторонами.
\begin{parts}
\part Будут ли эти события попарно независимы? \fillin[да]
\part Сформулируйте определение независимости в совокупности для трех событий: 

\vspace{1in}

\begin{solution}
События независимы в совокупности, если для любого поднабора событий $A_1$, \ldots, $A_k$ выполняется равенство $\P(A_1 \cap A_2 \cap \ldots \cap A_k) = \P(A_1) \cdot \ldots \cdot \P(A_k)$
\end{solution}


\part Являются ли события $A$, $B$, $C$ независимыми в совокупности? \fillin[нет]
\end{parts}


\question Имеются два игральных кубика: \textbf{красный} со смещенным центром тяжести, так что вероятность выпадения «6» равняется 1/3, а оставшиеся грани имеют равные шансы на появление и 
правильный \textbf{белый} кубик.  Петя случайным образом выбирает кубик и подбрасывает его.
\begin{parts}
\part Вероятность того, что выпадет «6», равна \fillin[1/4].
\part Вероятность того, что Петя взял красный кубик, если известно, что выпала шестерка, равна \fillin[2/3].
\part Если бы в эксперименте Петя подбрасывал  бы кубик не один раз, а 60 раз, то безусловное математическое ожидание количества выпавших шестёрок равнялось бы \fillin[15].
\end{parts}


\begin{comment}
\question Неразменный пятак всегда выпадает «орлом». У Александра Привалова в кармане один неразменный пятак и два обычных, равновероятно выпадающих «орлом» и «решкой». Привалов достаёт одну из монет наугад не глядя.
\begin{parts}
\part Вероятность того, что он достанет неразменный пятак равна \fillin[1/3].
\part Не глядя на монету, Привалов подкидывает её. Вероятность того, что она выпадет  «орлом», равна \fillin[2/3]. 
%\part Если бы эту случайную монету подкинуть не один раз, а 10, то математическое ожидание числа «орлов» равнялось бы \fillin[$20/3$].
\part Наконец Привалов глядит на упавшую монету и видит, что выпал «орёл». Вероятность того, что монета --- неразменный пятак, равна \fillin[...].
\end{parts}

\end{comment}

\question Винни-Пуху снится сон, будто он спустился в погреб, а там бесконечное количество горшков. Каждый из них независимо от других может оказаться либо пустым с вероятностью $0.8$, либо с мёдом с вероятностью $0.2$. Винни-Пух начинает перебирать горшки по очереди в поисках полного. Хотя у него в голове и опилки, Винни-Пух два раза в один и тот же горшок заглядывать не будет. 
\begin{parts}
\part Вероятность того, что все горшки окажутся пустыми равна \fillin[0]. 

\part Вероятность того, что полный горшок будет найден ровно с шестой попытки, равна \fillin[$0.8^5\cdot 0.2$].
\part Вероятность того, что полный горшок будет найден на шестой попытке или ранее, равна \fillin[$1-0.8^6$].
%\part Математическое ожидание числа перебранных горшков равняется \fillin[5].
\end{parts}

\question На самом деле у Винни-Пуха в погребе стоит 10 горшков. Каждый из них независимо от других может оказаться либо пустым с вероятностью $0.8$, либо с мёдом с вероятностью $0.2$. 
\begin{parts}
\part Все десять горшков окажутся пустыми с вероятностью \fillin[$0.8^{10}$].
\part Ровно $7$ горшков из десяти окажутся пустыми с вероятностью \fillin[$C_{10}^3 0.2^3 0.8^7$]
\part Математическое ожидание числа горшков с мёдом равно \fillin[2].
\end{parts}


\begin{comment}
\question Внутри треугольника с вершинами $(0,0)$, $(2,5)$ и $(8,0)$ случайно равномерно по площади выбирается точка. Пусть $X$ и $Y$ --- абсцисса и ордината этой случаной точки.
\begin{parts}
\part Вероятность того, что $X>5$ равна \fillin[...].
\part Вероятность того, что $X>5$ и одновременно $Y<3$ равна \fillin[..].
\part Вероятность того, что $X>5$ если известно, что $Y<3$ равна \fillin[...].
\part События $X>5$ и $Y<3$ являются \fillin[...]висимыми.
\part Функция плотности величины $X$ равна 
\begin{flalign*}
f(x) &= &&
\end{flalign*}

\end{parts}
\end{comment}

\question В галактике Флатландии все объекты двумерные. На планету Тау-Слона (окружность) в случайных точках независимо друг от друга садятся три корабля. Любые два корабля могут поддерживать прямую связь между собой, если центральный угол между ними меньше прямого. 

\begin{parts}
\part Вероятность того, что первый и второй корабли могут поддерживать прямую связь равна \fillin[1/2]
\part Вероятность того, что все корабли смогут поддерживать прямую связь друг с другом равна \fillin[3/16]
\part Вероятность того, что все корабли смогут поддерживать прямую связь друг с другом, если первый и второй корабль могут поддерживать прямую связь, равна \fillin[3/8]
\end{parts}
Подсказка: во Флатландии хватит рисунка на плоскости, ведь координату третьего корабля можно принять за\ldots



\question Время (в часах), за которое студенты выполняют экзаменационное задание является случайной величиной $X$ с функцией плотности
\[
f(x)=\begin{cases}
3x^2, \, \text{ если } x \in [0;1] \\
0, \, \text{ иначе }
\end{cases}
\]

\begin{parts}
\part Функция распределения случайной величины $X$ равна

\begin{solution}
\begin{flalign*}
F_X(x) &= \begin{cases}
0, \, x<0 \\
x^3, \, x \in [0;1] \\
1, \, x>1 
\end{cases}&&
\end{flalign*}
\end{solution}

\part Вероятность того, что случайно выбранный студент закончит работу менее чем за полчаса равна \fillin[1/8].
\part Медиана распределения равна \fillin[один на корень кубический из двух] 
\part Вероятность того, что студент, которому требуется по меньшей мере 15 минут для выполнения задания, справится с ним более, чем за 30 минут, равна \fillin[$56/63$]. 
\part Функция распределения случайной величины $Y=1/X$ равна

\vspace{1in}
\begin{solution}
\begin{flalign*}
F_Y(y) &= \begin{cases}
0, \, y<0 \\
1-1/y^3, \, y>0
\end{cases}&&
\end{flalign*}
\end{solution}

\part Функция плотности случайной величины $Y=1/X$ равна

\vspace{1in}
\begin{solution}
\begin{flalign*}
f_Y(y) &= \begin{cases}
0, \, y<0 \\
3y^{-4}, \, y>0
\end{cases}&&
\end{flalign*}
\end{solution}
\end{parts}


\end{questions}

\end{document}