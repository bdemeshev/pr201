\documentclass[12pt,a4paper]{article}
\usepackage[utf8]{inputenc}
\usepackage[russian]{babel}

\usepackage{amsmath}
\usepackage{amsfonts}
\usepackage{amssymb}
\usepackage[left=2cm,right=2cm,top=2cm,bottom=2cm]{geometry}
\begin{document}

\section*{Дельта-метод}

Нормальное распределение возникает, если суммируется большое количество независимых одинаково распределенных случайных величин. Однако оно возникает и в других ситуациях! Дельта-метод основан на том факте, что даже нелинейная функция от нормально распределенной случайной величины  иногда имеет распределение близкое к нормальному.


\subsubsection*{Дельта-метод на практике}

Если функция $f$ дифференциируема, то в окрестности точки $\mu$ функция $f$ похожа на прямую, то есть $f(x)\approx f(\mu) + f'(mu)(x-\mu)$. Линейное преобразование нормально распределенной случайной величины оставляет её нормально распределенной, если угловой коэффициент отличен от нуля, т.е. $f'(\mu) \neq 0$. Если $X\approx N(\mu,\sigma^2)$ и  дисперсия $X$ мала, то $X$ практически всегда попадает в небольшую окрестность $\mu$, а в ней $f$ похожа на линейную функцию и $f(X) \approx N(\mu, \sigma^2 (f'(\mu))^2$. 

Получаем практическую версию дельта-метода.

Если: $f$ --- дифференциируема, $f'(\mu)\neq 0$, $X\approx N(\mu,\sigma^2)$ и дисперсия $\sigma^2$ мала, то $f(X) \approx N(f(\mu),\sigma^2 (f'(\mu))^2)$.


\subsubsection*{Дельта-метод в теории}

Естественно, строгая формулировка идеи <<дисперсия $\sigma^2$ мала>> использует понятие предела и последовательностей случайных величин.

Если: $f$ --- $f$ --- дифференциируема, $f'(\mu)\neq 0$, и последовательность случайных величин $\{X_n\}$ удовлетворяет условию:
\[
\sqrt{n} (X_n - \mu) \to N(0,\sigma^2)
\]
То: последовательность $f(X_n)$ удовлетворяет условию:
\[
\sqrt{n} (f(X_n) - f(\mu)) \to N(0,\sigma^2 (f'(\mu))^2 )
\]



\subsubsection*{Примеры задач}

\begin{enumerate}
\item Известно, что $X_i$ независимы и одинаково распределены со средним $5$ и дисперсией $4^2$. Найдите примерный закон распределения величины $(1+\bar{X})/(\bar{X}^2+5)$ при большом $n$.

\item Величина $X$ имеет биномиальное распределение $Bin(n,p)$ и $n$ велико. Какое распределение примерно имеют величины $\ln(X/n)$? $X/(n-X)$?
\end{enumerate}





\end{document}