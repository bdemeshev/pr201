
\section{Два способа задать геометрию}

Множество случайных величин - это векторное (линейное) пространство. Если сложить две случайных величины, то получится случайная величина, если умножить случайную величину на число получится случайная величина. Поэтому случайные величины - это векторы.

Чтобы задать геометрию достаточно определить скалярное произведение, то есть действие $<X,Y>$, которое любой паре случайных величин ставит в соответствие число. При этом должны выполняться свойства:

1. $<X,Y>=<Y,X>$

2. $<X+Y,Z>=<X,Z>+<Y,Z>$

3. $<X,X>\geq 0$

4. $<X,X>=0$ только если $X=0$.

Почему скалярное произведение определяет геометрию? Геометрия - это же про длины, углы и расстояния! Все это восстанавливается из скалярного произведения по формулам 9-го класса. Длину любого вектора теперь можно найти по формуле $||X||=\sqrt{<X,X>}$.
Для того, чтобы найти угол между векторами достаточно знать косинус этого угла. А косинус определяется как: $cos(X,Y)=\frac{<X,Y>}{||X||||Y||}$. Расстояние определяется как длина разности: $d(X,Y)=||X-Y||$.

Мы используем двойные палочки для длины чтобы отличать ее от модуля: модуль случайной величины - это случайная величина, а длина - это константа.

Можно предложить много разных геометрий (или, что то же самое, скалярных произведений) в пространстве случайных величин, но интересными, пожалуй, являются две: $<X,Y>=\E(XY)$ и $<X,Y>=\Cov(X,Y)$.

Геометрия позволит <<увидеть>> некоторые понятия и теоремы. Чаще всего мы будет сталкиваться с теоремой Пифагора, настолько часто, что можно быть уверенным: если что-то неотрицательное равно сумме двух неотрицательных частей, то это теорема Пифагора и можно ее проиллюстрировать.

\section{Геометрия ожидаемого произведения}
Пусть скалярное произведение задано $<X,Y>=\E(XY)$.

Легко убедиться, что первые три требования к скалярному произведению выполнены:

1. $\E(XY)=\E(YX)$

2. $\E((X+Y)Z)=\E(XZ)+\E(YZ)$

3. $\E(X^2)\geq 0$

Четвертое требование выполнено не совсем полностью. Оказывается $\E(X^2)$ может быть равно нулю, даже если $X$ не всегда ноль. Например, пусть $Y$ равномерно на $[0;1]$, а $X$ равен 1, если $Y=0.5$ и 0 иначе. В данном примере $X$ может равняться одному, но  $\E(X^{2})=0$. Вызвано это тем, что $P(X=0)=1$. То есть в этой геометрии нулевой считается случайная величина, которая равна нулю с вероятностью один. 

Если вероятность события $A$ равна 1, то говорят, что $A$ происходит почти наверное. В геометрии ожидаемого произведения случайные величины почти наверное равные нулю не отличимы от настоящего нуля. И, следовательно, если $X=Y$ почти наверное, то в этой геометрии $X$ не отличим от $Y$. Действительно, в этом случае $X-Y=0$ почти наверное, то есть $X-Y$ не отличима от нуля.

Что нам дает введение геометрии?

Теперь вполне серьезно можно говорить о длине случайное величины $||X||=\sqrt{\E(X^{2})}$ или об угле между двумя случайными величинами $\angle(X,Y)=arccos(\frac{\E(XY)}{\sqrt{\E(X^{2})\E(Y^{2})}})$. А что из этого?

Автоматически возникает понятие перпендикулярных (ортогональных) векторов: случайные величины перпендикулярны, если угол между ними равен $\frac{\pi}{2}$. Или, если косинус угла между ними равен нулю. Или, $\E(XY)=0$.

Теорему Пифагора никто не отменял: если $X\bot Y$, то $||X-Y||^{2}=||X||^{2}+||Y||^{2}$.

Доказательство: $\E((X-Y)^2)=\E(X^{2})+\E(Y^{2})-2\E(XY)=\E(X^{2})+\E(Y^{2})$.

Заметим пару интересных фактов в нашей геометрии: 

Длина любой константы равна ее модулю: $||c||=\sqrt{\E(c^{2})}=\sqrt{c^{2}}=|c|$, или $||c||^{2}=c^{2}$.

Если $\E(Y)=0$, то случайная величина $Y$ перпендикулярна любой константе: $<Y,c>=\E(Yc)=c\E(Y)=0$.

Рисунок 1. числовая прямая, ей перпендикулярная величина $Y$ и неперпендикулярная $X$

Применим теорему Пифагора чтобы увидеть дисперсию...

Пусть $X$ - произвольная случайная величина. У случайной величины $Y=X-\E(X)$ матожидание равно нулю, $\E(Y)=\E(X)-\E(X)=0$, поэтому $Y$ перпендикулярна любой константе, в частности, $Y\bot \E(X)$. Заметим кстати, что $||X-\E(X)||^{2}=\E((X-\E(X))^{2})=\Var(X)$.

Применяя теорему Пифагора к $X-\E(X)\bot \E(X)$ получаем:
$||X||^{2}=||X-\E(X)||^{2}+||\E(X)||^{2}$

Переходя к ожиданиям, получаем $\E(X^{2})=\Var(X)+(\E(X))^{2}$.

Рисунок 2. числовая прямая, ей неперпендикулярная $X$, подписи $\E(X)^{2}$...

Из рисунка видно, что $\E(X)$ это проекция случайной величины $X$ на множество констант! Это действительно так в нашей геометрии: 

Во-первых, $X-\E(X)\bot \E(X)$.

Во-вторых, если взять любую другую константу $c\neq \E(X)$, то расстояние от $X$ до этой константы $c$ будет больше, чем до константы $\E(X)$: $||X-c||>||X-\E(X)||$. Доказательство: Рассмотрим функцию $\E((X-c)^2)=\E(X^{2})+c^{2}-2c\E(X)$. Относительно $c$ это парабола с ветвями вверх и вершиной при $c=\E(X)$. Значит наименьшое значение функции равно $||X-\E(X)||$.



\section{Связь со школьной геометрией}

\section{Условное ожидание - это проекция!}


\section{Геометрия ковариации}
Пусть скалярное произведение задано $<X,Y>=\Cov(X,Y)$.

Все требования к скалярному произведению кроме ... выполнены.

Требование ... выполнено с оговорками.

Эта геометрия наглядна тем, что некоррелированные случайные величины в ней перпендикулярны.
Если $Corr(X,Y)=0$, то $\Cov(X,Y)=0$ и, следовательно, $X$ и $Y$ ортогональны.

Напомним, что независимость означает некоррелированность любых\footnote{Любых борелевских функций (для знакомых с теорией меры)} функций $f(X)$ и $g(Y)$.



\section{Частная корреляция}



В анализе временных рядов при изучении процессов ARMA используется понятие частной корреляции. При этом указывается некий шаманский способ ее подсчета (как правило это система уравнений Юла-Воркера) и изредка - ее интуитивная интерпретация. Мы же беремся рассказать что это такое на самом деле в рамках геометрии ковариации! Станет ясна связь между формулой расчета и интуивной интерпретацией!

Рассмотрим три случайных величины $X$, $Y$ и $Z$. Обычная корреляция $Corr(X,Y)$ - это косинус угла между $X$ и $Y$, $cos(X,Y)$. Что же такое частная корреляция $X$ $Y$ при фиксированном (????) $Z$, $Corr(X,Y|Z)$ (????).

Очень просто! Изначально $X$ коррелировано с $Z$ и $Y$ коррелировано с $Z$. Возьмем и <<очистим>> $X$ и $Y$ от воздействия $Z$, то есть найдем самые похожие на них величины $\hat{X}$ и $\hat{Y}$ не коррелированные с $Z$. От случайной величины $\hat{X}$ требуется чтобы она была некоррелирована с $Z$ и максимально похожа на $X$, то есть чтобы квадрат расстояния $||X-\hat{X}||^{2}=\Var(X-\hat{X})$ был минимальным. Геометрически это означает следующее: есть $Z^{\bot}$ - множество случайных величин, некоррелированных с $Z$ (ортогональных $Z$). Просто спроецируем $X$ и $Y$ на это множество $Z^{\bot}$. Так частная корреляция между $X$ и $Y$ при фиксированном $Z$ это просто обычная корреляция между $\hat{X}$ и $\hat{Y}$, $Corr(X,Y|Z)=Corr(\hat{X},\hat{Y})$ или косинус угла между проекциями $X$ и $Y$ на плоскость $Z^{\bot}$. 

Давайте попробуем посчитать на простом примере, не связанном с временными рядами.
Пусть $X$, $Y$, $Z$




Из этого определения следует формула расчета, предлагаемая процедурой Юла-Воркера.








Чтобы изложения было законченным - проиллюстрируем интуитивную интерпретацию 
% стрелочки для AR и MA процессов - нужно ли, все таки уже прямо не связано?



 то есть спроецируем величины $X$ и $Y$ на множество случайных величин
% продумать обозначение и перевод





