\documentclass[12pt]{article}

\usepackage{tikz} % картинки в tikz
\usepackage{microtype} % свешивание пунктуации

\usepackage{array} % для столбцов фиксированной ширины

\usepackage{indentfirst} % отступ в первом параграфе

\usepackage{sectsty} % для центрирования названий частей
\allsectionsfont{\centering}

\usepackage{amsmath} % куча стандартных математических плюшек

\usepackage{comment}

\usepackage[top=2cm, left=1.2cm, right=1.2cm, bottom=2cm]{geometry} % размер текста на странице

\usepackage{lastpage} % чтобы узнать номер последней страницы

\usepackage{enumitem} % дополнительные плюшки для списков
%  например \begin{enumerate}[resume] позволяет продолжить нумерацию в новом списке
\usepackage{caption}


\usepackage{fancyhdr} % весёлые колонтитулы
\pagestyle{fancy}
\lhead{Теория вероятностей}
\chead{}
\rhead{2017-10-24, праздник номер раз :)}
\lfoot{}
\cfoot{}
\rfoot{\thepage/\pageref{LastPage}}
\renewcommand{\headrulewidth}{0.4pt}
\renewcommand{\footrulewidth}{0.4pt}



\usepackage{todonotes} % для вставки в документ заметок о том, что осталось сделать
% \todo{Здесь надо коэффициенты исправить}
% \missingfigure{Здесь будет Последний день Помпеи}
% \listoftodos --- печатает все поставленные \todo'шки


% более красивые таблицы
\usepackage{booktabs}
% заповеди из докупентации:
% 1. Не используйте вертикальные линни
% 2. Не используйте двойные линии
% 3. Единицы измерения - в шапку таблицы
% 4. Не сокращайте .1 вместо 0.1
% 5. Повторяющееся значение повторяйте, а не говорите "то же"



\usepackage{fontspec}
\usepackage{polyglossia}

\setmainlanguage{russian}
\setotherlanguages{english}

% download "Linux Libertine" fonts:
% http://www.linuxlibertine.org/index.php?id=91&L=1
\setmainfont{Linux Libertine O} % or Helvetica, Arial, Cambria
% why do we need \newfontfamily:
% http://tex.stackexchange.com/questions/91507/
\newfontfamily{\cyrillicfonttt}{Linux Libertine O}

\AddEnumerateCounter{\asbuk}{\russian@alph}{щ} % для списков с русскими буквами
\setlist[enumerate, 2]{label=\asbuk*),ref=\asbuk*}

%% эконометрические сокращения
\DeclareMathOperator{\Cov}{Cov}
\DeclareMathOperator{\Corr}{Corr}
\DeclareMathOperator{\Var}{Var}
\DeclareMathOperator{\E}{E}
\def \hb{\hat{\beta}}
\def \hs{\hat{\sigma}}
\def \htheta{\hat{\theta}}
\def \s{\sigma}
\def \hy{\hat{y}}
\def \hY{\hat{Y}}
\def \v1{\vec{1}}
\def \e{\varepsilon}
\def \he{\hat{\e}}
\def \z{z}
\def \hVar{\widehat{\Var}}
\def \hCorr{\widehat{\Corr}}
\def \hCov{\widehat{\Cov}}
\def \cN{\mathcal{N}}


\begin{document}


Ровно 272 года назад императрица Елизавета повелела завезти во дворцы котов для ловли мышей.


\begin{enumerate}

\item В отсутствии кота Леопольда мыши Белый и Серый подкидывают по очереди игральный додекаэдр
%\footnote{Леопольд подсказывает по случаю праздника, что у додекаэдра 12 граней :)}
.
Сыр достаётся тому, кто первым выкинет число 6. Начинает подкидывать Белый.

\begin{enumerate}
  \item Какова вероятность того, что сыр достанется Белому?
  \item Сколько в среднем бросков продолжается игра?
  \item Какова дисперсия числа бросков?
\end{enumerate}

\item Микки Маус, Белый и Серый решили устроить труэль из любви к мышки Мии. Сначала стреляет Микки, затем Белый, затем Серый, затем снова Микки и так до тех пор, пока в живых не останется только один.

Прошлые данные говорят о том, что Микки попадает с вероятностью $1/3$, Белый — с вероятностью $2/3$, а Серый стреляет без промаха.

Найдите оптимальную стратегию каждого мыша.

\item Микки Маус, Белый и Серый пойманый злобным котом Леопольдом до начала труэли. И теперь Леопольд будет играть с ними в странную игру.

В комнате три закрытых внешне неотличимых коробки: с золотом, серебром и платиной. Общаться после начала игры мыши не могут, но могут заранее договориться о стратегии.

Правила игры таковы. Кот Леопольд будет заводить мышей в комнату по очереди. Каждый из мышей может открыть
две коробки по своему выбору. Перед следующим мышом коробки закрываются.

Если Микки откроет коробку с золотом, Белый
— с серебром, а Серый — с платиной, то они выигрывают. Если
хотя бы один из мышей не найдёт свой металл, то Леопольд их съест.
\begin{enumerate}
\item Какова оптимальная стратегия?
\item Какова вероятность выигрыша при использовании оптимальной стратегии?
\end{enumerate}

\item Накануне войны Жестокий Тиран Мышь очень большой страны издал указ. Отныне за каждого новорождённого мыша-мальчика семья получает денежную премию, но если в семье рождается вторая мышка-девочка, то всю семью убивают. Бедные жители страны запуганы и остро нуждаются в деньгах, поэтому в каждой семье мыши будут появляться до тех пор, пока не родится первая мышка-девочка.

\begin{enumerate}
  \item Каким будет среднее число детей в мышиной семье?
  \item Какой будет доля мышей-мальчиков в стране?
  \item Какой будет средняя доля мышей-мальчиков в случайной семье?
  \item Сколько в среднем мышей-мальчиков в случайно выбираемой семье?
\end{enumerate}

\item Вальяжный кот Василий положил на счёт в банке на Гаити один гурд. Сумма на счету растёт непрерывно с постоянной ставкой в течение очень длительного промежутка времени. В случайный момент этого промежутка кот Василий закрывает свой вклад.

Каков закон распределения первой цифры полученной Василием суммы?

\begin{comment}
\item Начинающий трейдер Афанасий совершает не более одной сделки в день.

Если в какой-то день у трейдера Афанасия есть акция, то за этот день он равновероятно продаёт или не продаёт её. Если в какой-то день у трейдера Афанасия нет акций, то он равновероятно покупает или не покупает одну акцию.

Найдите ожидаемую прибыль Афанасия, если известно, что реализовывал он свою стратегию 100 дней, в начале акции стоили по 50 рублей, в конце — по 80 рублей, максимум составил 120 рублей, а минимум — 30.


\item Страшный Мейн-кун разрубает палочку единичной длины на $10$ частей в случайных и независимых местах, равномерно распределённых по всех длине. Затем Страшный Мейн-кун выбирает случайно один из кусочков и возводит его длину в $24$ степень.

Какое в среднем число он получит?
\end{comment}

\end{enumerate}



\end{document}
