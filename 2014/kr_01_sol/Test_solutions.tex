\documentclass[a4paper,12pt]{article}

%%% Работа с русским языком
\usepackage{cmap}				            	% поиск в PDF
\usepackage{mathtext} 			              % русские буквы в формулах
\usepackage[T2A]{fontenc}		          	% кодировка
\usepackage[utf8]{inputenc}		         	% кодировка исходного текста
\usepackage[english,russian]{babel}	    % локализация и переносы

%%% Дополнительная работа с математикой (по идее, если подключен mathtools, amsmath можно не юзать) 
\usepackage{amsmath, amsfonts, amssymb, amsthm, mathtools} % AMS
\usepackage{icomma}                            % "Умная" запятая: $0,2$ --- число, $0, 2$ --- перечисление

%% Номера формул
%\mathtoolsset{showonlyrefs=true}      % Показывать номера только у тех формул, на которые есть \eqref{} в тексте.
%\usepackage{leqno}                             % Нумерация формул слева

% % Работа с цветом
\usepackage[usenames, dvipsnames, svgnames, table, rgb]{xcolor}
\usepackage{colortbl}


%% Свои команды
\DeclareMathOperator{\sgn}{\mathop{sgn}}

%% Перенос знаков в формулах (по Львовскому)
\newcommand*{\hm}[1]{#1\nobreak\discretionary{}
{\hbox{$\mathsurround=0pt #1$}}{}}

%%% Работа с картинками
\usepackage{graphicx}                                   % Для вставки рисунков
\graphicspath{{images/}{images2/}}               % папки с картинками
\setlength\fboxsep{3pt}                                 % Отступ рамки \fbox{} от рисунка
\setlength\fboxrule{1pt}                                 % Толщина линий рамки \fbox{}
\usepackage{wrapfig}                                      % Обтекание рисунков текстом

%%% Работа с таблицами
\usepackage{array, tabularx, tabulary, booktabs}  % Дополнительная работа с таблицами
\usepackage{longtable}                                      % Длинные таблицы
\usepackage{multirow}                                       % Слияние строк в таблице


%%% Теоремы
\theoremstyle{plain}                         % Это стиль по умолчанию, его можно не переопределять.
\newtheorem{theorem}{Теорема}[section]
\newtheorem{proposition}[theorem]{Утверждение}
 
\theoremstyle{definition}                  % "Определение"
\newtheorem{corollary}{Следствие}[theorem]
\newtheorem{problem}{Задача}[section]
 
\theoremstyle{remark}                      % "Примечание"
\newtheorem*{nonum}{Решение}

%%% Программирование
\usepackage{etoolbox}                     % логические операторы

%%% Страница
\usepackage{extsizes}                      % Возможность сделать 14-й шрифт
\usepackage{geometry}                    % Простой способ задавать поля
	\geometry{top=25mm}
	\geometry{bottom=35mm}
	\geometry{left=25mm}
	\geometry{right=20mm}
 
%\usepackage{fancyhdr} % Колонтитулы
% 	\pagestyle{fancy}
%\renewcommand{\headrulewidth}{0pt}   % Толщина линейки, отчеркивающей верхний колонтитул
% 	\rhead{Верхний правый}
% 	\chead{Верхний в центре}
% 	\lhead{Верхний левый}
% 	\lfoot{Нижний левый}
%	\cfoot{Нижний в центре} % По умолчанию здесь номер страницы
% 	\rfoot{Нижний правый}



\usepackage{setspace}   % Интерлиньяж
\onehalfspacing             % Интерлиньяж 1.5
%\doublespacing            % Интерлиньяж 2
%\singlespacing              % Интерлиньяж 1

\usepackage{lastpage}                   % Узнать, сколько всего страниц в документе.
\usepackage{soul}                         % Модификаторы начертания

\usepackage{hyperref}
\hypersetup{			                   	   % Гиперссылки
    unicode=true,                             % русские буквы в раздела PDF
    pdftitle={Заголовок},                  % Заголовок
    pdfauthor={Автор},                     % Автор
    pdfsubject={Тема},                      % Тема
    pdfcreator={Создатель},             % Создатель
    pdfproducer={Производитель},  % Производитель
    pdfkeywords={keyword1} {key2} {key3}, % Ключевые слова
    colorlinks=true,                 	      % false: ссылки в рамках; true: цветные ссылки
    linkcolor=black,                         % внутренние ссылки
    citecolor=black,                          % на библиографию
    filecolor=magenta,                      % на файлы
    urlcolor=blue                               % на URL
}

\usepackage{csquotes}                    % Еще инструменты для ссылок
\usepackage{cite}

\usepackage{multicol}                      % Писать документ в несколько колонок

%% Работа с графикой
\usepackage{tikz}
\usetikzlibrary{arrows}                           
\usepackage{pgfplots}
\usepackage{pgfplotstable}

\begin{document}

\begin{center}
\section*{Часть 1}
\end{center}
Не претендуя на единственность, решения претендуют на правильность!

\subsection*{Задача 1}
\renewcommand{\labelenumi}{(\alph{enumi})}
\begin{enumerate}
\item $P(\cdot) = 0.9^2\cdot 0.7 = 0.567  $
\item $A$ = \{случайно выбранный арбуз --- от тети Маши\}; $B$ = \{случайно выбранный арбуз оказался спелым\}. Формула условной вероятности:
$$P(A|B) = \dfrac{P(AB)}{P(B)}= \dfrac{2/3 \cdot 0.9}{2/3 \cdot 0.9 + 1/3 \cdot 0.7} =\dfrac{18}{25}$$
\item $A$ = \{второй и третий съеденные арбузы --- от тети Маши\}; $B$ = \{все три арбуза --- спелые\}. Дает ли нам что-то о принадлежности арбузов к тете Маше или тете Оле то, что все арбузы --- спелые? События независимы!
$$P(A|B) = P(A) = \dfrac{1}{3}$$
\end{enumerate}

\subsection*{Задача 2}
\begin{enumerate}
\item $\mathbb{E}(X) = \sum P(X_i) X_i = 1.9$

$\mathbb{V}ar(X) = \mathbb{E}(X^2) - (\mathbb{E}(X))^2 = 0\cdot 0.1 + 1 \cdot 0.3 + 4 \cdot 0.2 + 9 \cdot 0.4 - 1.9^2 = 1.09$

\item Раз ребенок выбран, значит, в его семье дети есть! Всего детей $n\mathbb{E}(X) = 1.9n$. Семей с одним ребенком --- $0.3n$, значит, детей из семей с одним ребенком --- $0.3n$. Аналогично, детей из семей с двумя детьми --- $0.4n$; детей из семей с тремя детьми --- $1.2n$. 

Теперь легко построить закон распределения случайной величины $Y$:
\begin{table}[h]
\begin{tabular}{l|c|c|c}
$Y$    & $1$                & $2$                & $3$                \\ \hline
$P(Y)$ & $3/19$ & $4/19$ & $12/19$
\end{tabular}
\end{table}

$$\mathbb{E}(Y) = \dfrac{3}{19} + \dfrac{8}{19} + \dfrac{36}{19} = \dfrac{47}{19} > \mathbb{E}(X)$$

\end{enumerate}

\newpage

\subsection*{Задача 3}
Любителям (или нелюбителям) интегралов:
\begin{enumerate}
\item Да это же интеграл от функции плотности на всей числовой прямой! Ответ: единица!
\item $$\mathbb{E}(X) = \int \limits_0^2 x f(x) dx = \int \limits_0^2 \dfrac{3}{8} x^3 dx = \dfrac{3}{32} x^4 |_0^2 = \dfrac{3}{2}$$

$$\mathbb{E}(X^2) = \int \limits_0^2 x^2 f(x) dx = \int \limits_0^2 \dfrac{3}{8} x^4 dx = \dfrac{3}{40} x^5 |_0^2 = \dfrac{12}{5}$$

Формула дисперсии:
$$\mathbb{V}ar(X) = \mathbb{E}(X^2) - \left(\mathbb{E}(X) \right)^2 = \dfrac{12}{5} - \dfrac{9}{4} = \dfrac{3}{20} $$

\item $$P(X>1.5) = \int \limits_{1.5}^2 f(x) dx = \int \limits_{1.5}^2 \dfrac{3}{8} x^2 dx = \dfrac{1}{8} x^3 |_{1.5}^2 = \dfrac{37}{64}$$

Вычислим вероятность условия:

$$P(X>1) = \int \limits_1^2 f(x) dx = \int \limits_1^2 \dfrac{3}{8} x^2 dx = \dfrac{1}{8} x^3 |_1^2 = \dfrac{7}{8}$$

$$P(X>1.5 | X>1) = \dfrac{P(X>1.5)}{P( X>1)} = \dfrac{37/64}{7/8} = \dfrac{37}{56}$$

\item Должно выполниться следующее соотношение:
$$\int \limits_{-\infty}^{+\infty} c x f(x) dx  = 1$$

Применительно к нашей задаче:
$$\dfrac{3c}{8} \int \limits_0^2 x^3 dx  = \dfrac{3c}{32} x^4 |_0^2 = \dfrac{3c}{2} = 1 \Rightarrow c = \dfrac{2}{3}$$
\end{enumerate}

\subsection*{Задача 4}

You have to learn the rules of the game. And then you have to play better than anyone else. (А. Эйнштейн)

\begin{enumerate}
\item $$\mathbb{V}ar(Z) = \mathbb{E}(Z^2) - (\mathbb{E}(Z))^2 = 15 - 9 = 6$$
$$\mathbb{V}ar(4 - 3Z) = 9\mathbb{V}ar(Z) = 54$$
$$\mathbb{E}(5 + 3Z - Z^2) = 5 + 3\cdot (-3)  - 15 = -19 $$

\item $$\mathbb{V}ar(X \pm Y) = \mathbb{V}ar(X) + \mathbb{V}ar(Y) \pm 2 \cdot \mathbb{C}ov(X, Y)$$
Отсюда получаем:
$$\mathbb{V}ar(X + Y) - \mathbb{V}ar(X - Y) = 4 \mathbb{C}ov(X, Y) \Rightarrow \mathbb{C}ov(X, Y) = 2.5$$
$$\mathbb{C}ov(6 - X, 3Y) = -3\cdot 2.5 = -7.5$$

\item $$\mathbb{C}ov(X, Y) = 2.5 \ne 0$$ Случайные величины действительно независимы.
\end{enumerate}


\subsection*{Задача 5}
 Предположим, что правильный ответ 0.25. Но это невозможно, потому что вариантов ответа 0.25 --- два (1 и 4), значит ответ 0.5 тоже был бы правильный. Предположим, что правильный 0.5. Тогда 0.25 тоже правильный --- таких вариантов два из четырех, значит вероятность попасть в 0.25, выбрав ответ наугад, равна 0.5. Ответ 0.6, очевидно, неверен, потому что вероятность попасть в него равна 0.25. \\
\textbf{Правильный ответ:} 0

\newpage
\subsection*{Задача 6}
Удобно рассуждать следующим образом: предположим, что каждая опечатка наугад (\textcolor{red}{с равными вероятностями и независимо от других опечаток}) выбирает, на какую страницу ей попасть\footnote[1]{Ну очень самостоятельные!}. 

\begin{enumerate}
\item Пусть $X$ - число опечаток на 13 странице. $$P(X \geqslant 2) = 1 - P(X=0) - P(X=1) $$
$P(X=0) = \left( \frac{499}{500} \right)^{400}$ --- каждая из 400 опечаток не доложна попасть на 13 страницу.\\
$P(X=1) = 400\cdot\frac{1}{500}\cdot\left( \frac{499}{500} \right)^{399}$ --- ровно одна опечатка (а есть 400 вариантов) должна попасть на 13 страницу, а остальные --- мимо. Соответственно:
$$
P(X \geqslant 2) = 1 - \left( \frac{499}{500} \right)^{400} - 400\cdot\frac{1}{500}\cdot\left( \frac{499}{500} \right)^{399} \approx 0.1911357
$$
Это если считать в явном виде. А если пользоваться приближением Пуассона: 
$$
p(k) = P(X = k) = \frac{\lambda^k}{k!}e^{-\lambda}
$$
неплохо бы вспомнить, что парамер $\lambda$ это матожидание $X$, поэтому расчеты здесь пока оставим до лучших времен.

\item Пусть $X$ - число опечаток на 13 странице. Введем случайную величину 
$$X_i =
\begin{cases}
1 & \text{если } i\text{-ая опечатка попала на 13 страницу}\\
0 & \text{если нет}
\end{cases}
$$
Тогда $X = \sum\limits_{i=1}^{400}X_i$. Рассмотрим отдельно $X_i$: \hspace{0.5cm}
\begin{minipage}{0.3\linewidth}

\begin{tabular}{c|c|c}
$X_i$ & 1 & 0 \\
\hline
$P(X_i = \cdot)$ & $\frac{1}{500} $ & $\frac{499}{500}$
\end{tabular}
\end{minipage}


Так как $i$-ая опечатка наугад выбирает одну страницу из 500 и это должна быть именно 13. 

Тогда:
$$
\mathbf{E}[X_i] = \frac{1}{500} = \mathbb{E}[X^2_i] \Rightarrow 
$$
$$
\Rightarrow \mathbf{Var}(X_i) = \mathbf{E}[X^2_i] - (\mathbf{E}[X_i])^2 = \frac{1}{500} - \left(\frac{1}{500}\right)^2 = \frac{499}{500^2}
$$
Значит 
$$
\mathbb{E}[X] = \mathbb{E}\left[\sum\limits_{i=1}^{400}X_i\right] = \sum\limits_{i=1}^{400}\mathbf{E}[X_i]  = \frac{400}{500} = 0.8
$$

$$
\mathbf{Var}(X) = \mathbf{Var}\left(\sum\limits_{i=1}^{400}X_i\right) = \sum\limits_{i=1}^{400}\mathbf{Var}(X_i) = 400\cdot\frac{499}{500^2} = 0.8\cdot\frac{499}{500}
$$

Теперь мы знаем, что $\lambda = \mathbb{E}[X] = 0.8$ поэтому можем вернуться к пункту (а):
$$
P(X \geqslant 2) = 1 - P(X=0) - P(X=1)  = 1 - \frac{0.8^0}{0!}e^{-0.8} - \frac{0.8^1}{1!}e^{-0.8} = 0.1912079
$$ \vspace{-1.2cm}

\hspace{13cm} \fcolorbox{ForestGreen}{white}{So close!}

Осталось найти наиболее вероятное число опечаток на 13 странице:
$$
P(X=k) = \frac{0.8^k}{k!}e^{-0.8} \rightarrow \max \limits_k
$$
Очевидно, что эта функция убывает по $k$, ведь с ростом $k$:\\
 $k!$ растет, а $0.8^k$ убывает. Значит наиболее вероятное число ошибок --- $X = 0$


\item \href{https://en.wikipedia.org/wiki/Triskaidekaphobia}{Ох уж эти предрассудки!} 13-я страница точно такая же как и все остальные, ведь везде в решении можно просто заменить номер 13 на любой другой и ничего не изменится.
\begin{center}
\includegraphics[width=6cm]{13}
\end{center}
\end{enumerate}
\newpage

\subsection*{Задача 7}
Пусть $A = \{\text{<<Лекция полезна>>}\}$, $B = \{\text{<<Лекция интересна>>}\}$. Заметим, что лекции вообще независимы друг от друга. 

\begin{enumerate}
\item Пусть $X_A$ --- число полезных лекций, прослушанных Васей,  $X_B$ --- число интересных лекций, прослушанных Васей. Введем случайную величину:
$$X_i =
\begin{cases}
1 & \text{если } i\text{-ая лекция была полезна}\\
0 & \text{если нет}
\end{cases}
$$

Тогда $X_A = \sum\limits_{i=1}^{30}X_i$. Рассмотрим отдельно $X_i$: \hspace{0.5cm}
\begin{minipage}{0.3\linewidth}

\begin{tabular}{c|c|c}
$X_i$ & 1 & 0 \\
\hline
$P(X_i = \cdot)$ & $0.9$ & $0.1$
\end{tabular}
\end{minipage}

Вероятность 0.9 дана. Тогда:
$$
\mathbf{E}[X_i] = 0.9 = \mathbb{E}[X^2_i] \Rightarrow 
$$
$$
\Rightarrow \mathbf{Var}(X_i) = \mathbf{E}[X^2_i] - (\mathbf{E}[X_i])^2 = 0.9 - 0.9^2 = 0.09
$$

Значит 
$$
\mathbb{E}[X_A] = \mathbb{E}\left[\sum\limits_{i=1}^{30}X_i\right] = \sum\limits_{i=1}^{30}\mathbf{E}[X_i]  = 0.9\cdot30 = 27
$$
$$
\mathbf{Var}(X_A) = \mathbf{Var}\left(\sum\limits_{i=1}^{30}X_i\right) = \sum\limits_{i=1}^{30}\mathbf{Var}(X_i) = 0.09\cdot30 = 2.7
$$

Аналогично для числа интересных лекций можем получить:
$$
\mathbb{E}[X_B] = 0.7\cdot 30 = 21
$$
$$
\mathbf{Var}(X_B) = 0.21\cdot 30 = 6.3
$$


\item Так как интересность и полезность --- независимые свойства лекций, то:\\
 $P(\overline{A} \cap \overline{B}) = P(\overline{A})\cdot P(\overline{B}) = 0.3\cdot0.1 = 0.03$, где $\overline{A}$ значит <<не $A$>>. В свою очередь:\\
 $P(A\cup B) = P(A\cap\overline{B}) + P(B\cap\overline{A}) + P(A\cap B) = 1 - P(\overline{A})\cdot P(\overline{B}) = 0.97$ , где $(A\cup B)$ значит <<$A$  или $B$>>. Аналогично, путем введения бинарной случайной величины можем получить:
 $$
 \mathbb{E}[X_{\overline{A} \cap \overline{B}}] = 0.03 \cdot  30 = 0.9
 $$
 $$
 \mathbb{E}[X_{A\cup B}] = 0.97\cdot30 = 29.1
$$

\subsection*{Задача 8}
Будем пользоваться свойствами функций распределения и плотности. Для начала:
$$
\lim\limits_{x \rightarrow +\infty} F(X) = 1, \hspace{0.5cm} \lim\limits_{x \rightarrow -\infty} F(X) = 0,
$$
$$
\lim\limits_{x \rightarrow +\infty} \left(\frac{ae^x}{1+e^x}+b\right) = a+b := 1
$$
$$
\lim\limits_{x \rightarrow -\infty} \left(\frac{ae^x}{1+e^x}+b\right) = b :=0
$$
Откуда сразу получаем $$a =1, b = 0 \Rightarrow F(x) = \frac{e^x}{1+e^x}$$
Для дальнийших развлечений нам понадобится функция плотности:
$$
f(x) = F'(x) = \frac{e^x}{(1+e^x)^2}
$$
 Заметим, что она симметрична относительно нуля:
 $$
 f(-x) = \frac{\frac{1}{e^x}}{\left(1+\frac{1}{e^x}\right)^2} = \frac{e^x}{(1+e^x)^2} = f(x)
 $$
 Из того этого следует, что \textbf{математическое ожидание, а так же мода и медиана равны нулю}. Более того, так как функция плотности симметрична относительно \textbf{нулевого} матожидания, \textbf{центральный и начальный моменты третьего порядка равны между собой и равны нулю.} Можно было выписать интегралы для матожидания и третьего начального момента и сослаться на нечетность функции.
 
\end{enumerate}

\newpage

\begin{center}
\section*{Часть 2}
\end{center}

\begin{flushright}
 --- Это невозможно! \\
--- Нет. Это необходимо.\\
\textcopyright \hspace{0.1cm} Interstellar
\end{flushright}
\subsection*{Задача 1}
Алгоритм решения: рисуешь дерево $\rightarrow$ PROFIT 

\begin{center}

 \begin{tikzpicture}[->,>=stealth',shorten >=1pt,auto,node distance=3cm,
  thick,main node/.style={circle,fill=blue!20,draw,font=\sffamily\Large\bfseries}]

  \node[main node] (1) {1};
  \node[main node] (2) [below of=1] {2};
  \node[main node] (3) [below of=2] {3};

  \path[every node/.style={font=\sffamily\small}]
    (1) edge [loop right] node {<<не 6>>} (1)
        edge [bend right] node[left] {<<6>>} (2)
    (2) edge [bend right] node[right] {<<не 6>>} (1)
        edge [above] node[left] {<<6>>} (3);
\end{tikzpicture}

\end{center}

Комментарии к построению дерева: состояние 1 --- начальное, состояние 3 --- конец игры, когда выпало две <<шестерки>> подряд. Заметим, что выпадение любой <<нешестерки>> в процессе игры приводит нас к состоянию, эквивалентному начальному. 

Вероятность выпадения <<шестерки>> равна 1/6, <<нешестерки>> --- 5/6.

Теперь мы готовы оседлать коня!

\begin{enumerate}
\item $P(N = 1) = 0$ --- невозможно за ход закончить игру.

$P(N = 2) = \dfrac{1}{36}$

$P(N = 3) = \dfrac{5}{6} \cdot \dfrac{1}{6} \cdot \dfrac{1}{6} = \dfrac{5}{216}$

\item А теперь будет видна вся сила рисования дерева:

Пусть $\mathbb{E}_1$ --- число ходов, за которое мы ожидаем закончить игру, если игра начинается в состоянии 1, $\mathbb{E}_2$ --- число ходов, за которое мы ожидаем закончить игру, если игра начинается в состоянии 2. 

Получим два уравнения:
$\begin{cases} \mathbb{E}_2 = \frac{1}{6} \cdot 1 +  \frac{5}{6} (\mathbb{E}_1 + 1)   \\ \mathbb{E}_1 =  \frac{5}{6} (\mathbb{E}_1 + 1) + \frac{1}{6} ( \mathbb{E}_2 + 1) \end{cases} $

Решив эту систему, получим, что $\mathbb{E}_1 = 42$. А ведь это и есть $\mathbb{E}(N)$.


Аналогична логика для оставшихся мат. ожиданий.

Найдем математическое ожидание суммы набранных очков. Ясно, что если выпадает <<не 6>>, то мы ждем 3 очка. Тогда переопределив $\mathbb{E}_1$ и $\mathbb{E}_2$ следующим образом: пусть $\mathbb{E}_1$ --- число набранных очков, которое мы ожидаем получить за игру, если игра начинается в состоянии 1, $\mathbb{E}_2$ --- число набранных очков, которое мы ожидаем получить за игру, если игра начинается в состоянии 2.

Новые два уравнения:
$\begin{cases} \mathbb{E}_2 = \frac{1}{6} \cdot 6 +  \frac{5}{6} (\mathbb{E}_1 + 3)   \\ \mathbb{E}_1 =  \frac{5}{6} (\mathbb{E}_1 + 3) + \frac{1}{6} ( \mathbb{E}_2 + 6) \end{cases} $

Решаем и получаем: $\mathbb{E}(S) = \mathbb{E}_1 = 147$

А можно было сделать еще круче! Выше показано, что  $\mathbb{E}(N) = 42$. А сколько мы ждем очков за 1 ход? 3.5! Тогда $\mathbb{E}(S) = \mathbb{E}(N) \cdot 3.5 = 147$

Применяя схожую логику для $\mathbb{E}(N^2)$:\\
$$\mathbb{E}(N^2) = \dfrac{5}{6} \cdot \mathbb{E}\left((N + 1)^2\right) + \dfrac{1}{6} \cdot \dfrac{5}{6}  \cdot \mathbb{E}\left((N + 2)^2\right) + \dfrac{1}{6} \cdot \dfrac{1}{6} \cdot 2^2$$

Учитывая, что $\mathbb{E}(N) = 42$, получим: $\mathbb{E}(N^2) = 3414$.

\item Veni, vidi, vici
\begin{center}

\begin{tabular}{l|l}
$X_n$ & 6 \\ \hline
$P(X_n)$ & 1
\end{tabular}

\end{center}

\end{enumerate}

\subsection*{Задача 2}
\begin{enumerate}
\item $P(V = 1) = 1/30$, т.к. именно этому равна вероятность того, что Вовочка стоит ровно вторым в очереди;

$M = 1$ значит, что между Машенькой и Вовочкой ровно один человек в очереди. Если Вовочка находится от 3 (включительно) до 28 позиции в очереди, то для Машеньки есть две благоприятные позиции для события $M = 1$ (например, если Вовочка стоит на 15 месте, то благоприятные позиции для Машеньки --- стоять либо 13-ой, либо 17-ой). Если же Вовочка стоит на других позициях в очереди, то для Машеньки существует ровно одна благоприятная позиция:

$$P(M = 1) = \dfrac{26}{30} \cdot \dfrac{2}{29} +  \dfrac{4}{30} \cdot \dfrac{1}{29} = \dfrac{56}{30\cdot 29} = \dfrac{28}{435}$$

$M = V$ произойдет только, если Машенька стоит за Вовочкой. При этом для Машеньки существует только одна благоприятная позиция и только в том случае, что Вовочка стоит до 15 позиции (включительно):
$$P(M = V) = \dfrac{1}{2} \cdot \dfrac{1}{29} = \dfrac{1}{58} $$

\item $$\mathbb{E}(V) = \dfrac{0 + 1 + ... + 29}{30} = \dfrac{30\cdot 14 + 15}{30} = 14.5$$

Для $\mathbb{E}(M)$ можно решить в лоб, и получится красивая сумма, а можно вот так:

Сначала случайно кинем Вовочку и Машеньку на две из 30 позиций в очереди. Образуется три отрезка: точки между Вовочкой и Машенькой и два крайних отрезка (может быть, отрезок из 0 точек). Затем будем закидывать в очередь на оставшиеся позиции случайно 28 оставшихся людей (назовем их <<пропавшими>>). Т.к. все броски были случайны (или из соображений симметрии, как хотите), вероятность попасть в отрезок между Машенькой и Вовочкой для <<пропавшего>> равна $1/3$, вне отрезка --- соответственно $2/3$, и независима от остальных бросков (!).

Введем случайную величину $X_i$ для $i$-го <<пропавшего>>, которая равна $1$, если он попал в отрезок между Машенькой и Вовочкой, $0$, если не попал:


\begin{center}
\begin{tabular}{l|c|c}
$X_i$ & 1 & 0 \\ \hline
$P(X_i)$ & 1/3 & 2/3
\end{tabular}

\end{center}


Легко считается: $\mathbb{E}(X_i) = 1/3$, $\mathbb{E}(X^2_i) = 1/3$, $\mathbb{V}ar(X_i) = 1/3 - 1/9 = 2/9$.
Ясно, что $M = \sum_1^{28} X_i$. Тогда учитывая независимость $X_i$:

$$\mathbb{E}(M) = \dfrac{28}{3}$$
$$\mathbb{V}ar(M) = \dfrac{56}{9}$$
\end{enumerate}
\subsection*{Задача 3}

Биномиальное распределение --- \textit{À l’abordage!}.

Задача интерпретируется так: последний ход --- это когда мы обратились к коробку, в котором нет спичек (т.е. к одному коробку нужно обратиться $n+1$ раз). 

\begin{enumerate}
\item Если $0<k \leqslant n$, будем считать успехом --- попадание в коробок, к которому мы на последнем ходу игры (пустому коробку) обратились. До этого момента из негомбыло вытащено $n$ спичек, а из другого $2n-k$ спичек, т.е. всего в игре было $2n - k + 1$ шагов. Успехов --- $n + 1$ (вытащено $n$ спичек, и на последнем ходу мы к нему обратились). По формуле Бернулли получаем следующее ($X$ --- случайная величина, показывающая сколько спичек осталось в коробке, к которому мы не обратились на последнем ходу игры):
$$P(X = k) = C^{n+1}_{2n-k+1} \left(\dfrac{1}{2} \right)^{2n-k+1}$$

Если $k = 0$, то мы вытащили все спички из обоих коробков к последнему ходу, и нам без разницы к какому коробку мы обратимся на последнем шагу, т.е.:
$$P(X = 0) =2 C^{n+1}_{2n+1} \left(\dfrac{1}{2} \right)^{2n+1}$$

\item Среднее спичек в другом коробке:

$$\mathbb{E}(X) = \sum \limits_{k=1}^{n} k \cdot C^{n+1}_{2n-k+1} \left(\dfrac{1}{2} \right)^{2n-k+1}$$


\end{enumerate}







\subsection*{Задача 4}

Для того чтобы количество упаковок, которые необходимо купить, равнялось 50, нужно чтобы ни одну из наклеек Покупатель не встретил дважды, поэтому:
$$
P(X=50) = 1\cdot\frac{49}{50}\cdot\frac{48}{50}\cdot\dots\cdot\frac{1}{50} = \frac{49!}{50^{49}} \approx 3.4\cdot
10^{-21}$$ \vspace{-1cm}

\hspace{10.5cm}\fcolorbox{ForestGreen}{white}{Dum spiro, spero!}\footnote[2]{Надежда умирает последней!}

Теперь введем понятие <<шаг>>. Переход на новый шаг происходит в тот момент, когда покупатель получил наклейку, которой у него раньше не было. Начинаем с шага 0, когда нет ни одной наклейки, и шагать будем до 49, потому что в момент перехода на шаг 50 Покупатель получит последнюю необходимую наклейку и <<прогулка>> закончится. Введем случайную величину $X_q$ равную количеству покупок в течение шага номер $q$. Тогда $X = \sum \limits_{q=0}^{49}X_q$.  Найдем математическое ожидание $X_q$:
$$
\mathbb{E}[X_q] = \frac{n-q}{n}\cdot 1 + \frac{q}{n}\cdot\frac{n-q}{n}\cdot 2 + \left(\frac{q}{n}\right)^2\cdot\frac{n-q}{n}\cdot 3 + ...
$$
здесь $\dfrac{n-q}{n}$ ---  это вероятность найти наклейку, которой еще нет, а $\dfrac{q}{n}$, соответственно --- вероятность повториться. Вопрос теперь в том, как посчитать сумму:
$$
\mathbb{E}[X_q] = \frac{n-q}{n}\left( 1 + \frac{q}{n}\cdot 2 + \left(\frac{q}{n}\right)^2 \cdot 3 + ...\right) = \frac{n-q}{n}\cdot\sum\limits_{k=0}^{\infty}\left(\frac{q}{n}\right)^k(k+1)
$$

Можем выписать в столбик несколько первых членов вышестоящей суммы:
$$
\begin{array}{l}
\hspace{0.3cm}1 \\
\vspace{0.2cm}
\left(\dfrac{q}{n}\right)^1  + \left(\dfrac{q}{n}\right)^1  \\
\vspace{0.2cm}
\left(\dfrac{q}{n}\right)^2 + \left(\dfrac{q}{n}\right)^2  + \left(\dfrac{q}{n}\right)^2 \\
\left(\dfrac{q}{n}\right)^3 + \left(\dfrac{q}{n}\right)^3 + \left(\dfrac{q}{n}\right)^3 + \left(\dfrac{q}{n}\right)^3 \\    
\hspace{0.1cm}\cdots\cdots\cdots\cdots\cdots\cdots\cdots\cdots\cdots\cdots\cdots   
\end{array}
$$
Достаточно! Можем скомпоновать всю сумму другим способом, а именно --- по столбцам. Заметим, что сумма элементов в каждом столбце это сумма бесконечно убывающей геометрической прогрессии с одним и тем же знаменателем $\dfrac{q}{n}$ и различными первыми членами. Соответственно:

$$
\sum\limits_{k=0}^{\infty}\left(\frac{q}{n}\right)^k(k+1) = \dfrac{1}{1-\frac{q}{n}} + \dfrac{\frac{q}{n}}{1-\frac{q}{n}} + \dfrac{\left(\frac{q}{n}\right)^2}{1-\frac{q}{n}} + \dfrac{\left(\frac{q}{n}\right)^3}{1-\frac{q}{n}} + \dots =
$$
$$
= \frac{1}{1-\frac{q}{n}}\left( 1 + \frac{q}{n} + \left(\frac{q}{n}\right)^2 + \left(\frac{q}{n}\right)^3 + \dots\right) = \frac{n}{n-q}\cdot\frac{n}{n-q} = \left( \frac{n}{n-q}\right)^2
$$

Таким образом, получаем, что:
$$
\mathbb{E}[X_q] = \frac{n-q}{n}\cdot \left( \frac{n}{n-q}\right)^2 = \frac{n}{n-q}
$$
\hspace{10cm} и это верно для любого q!

$$
\mathbb{E}[X] = \mathbb{E}\left[\sum \limits_{q=0}^{49}X_q\right] = \sum \limits_{q=0}^{49}\mathbb{E}[X_q] = 
\frac{50}{50-0} + \frac{50}{50-1} + \dots + \frac{50}{50-49} = 50\left(\frac{1}{50} + \frac{1}{49} + \dots + 1\right) \approx
$$
$$
\approx 50\int\limits_{1}^{50}\frac{1}{x}\mathbf{d}x = 50\ln(50) \approx 195.5
$$

А теперь \textbf{простое решение от Бориса Борисовича:}
Величины $X_q$ независимы (но по разному распределены). Если долго пришлось ждать $i$-го шага, это ничего не говорит о $j$-ом шаге. Величины $X_q$ имеют известный закон распределения --- это число опытов до первого успеха при заданной вероятности успеха. Это геометрическое распределение, математическое ожидание которого равно $\dfrac{1}{p}$, а дисперсия: $\dfrac{1-p}{p^2}$, где $p$ --- \vspace{0.2cm}  вероятность успеха.

А те, кто забыл, могут \textbf{проще решить} методом первого шага:
Если $X$ --- число опытов до успеха при вероятности успеха $p$, то 
\[
\mathbb{E}[X]=p\cdot 1 + (1-p)\cdot \mathbb{E}[X+1]
\]
Откуда $\mathbb{E}[X]=\frac{1}{p}$ и дело в шляпе :)
Аналогично:
\[
\mathbb{E}[X^2]=p\cdot 1^2 + (1-p) \cdot \mathbb{E}[(X+1)^2]
\]
и решая, находим $\mathbb{E}[X^2]$. 



\begin{center}
\subsection*{Задача 5}
\end{center}

\begin{enumerate}
\item Необходимое и достаточное условие --- старушка не должна занять чужое место. С вероятностью $\frac{1}{n}$ она угадает свое место, значит, для каждого входящего его место будет свободно и он туда сядет.\\
\textbf{Ответ:} $\dfrac{1}{n}$
\item Будем искать вероятность того, что последний человек не сядет на свое место. 

Пусть $A_i = \{\text{Старушка села на  место } i\text{-го} \}$,  $B_{(i,j)} = \{i \text{-ый  пассажир сел на место } j\text{-ого} \}$

$$
P[n\text{-ый не сядет на свое место}] = P(A_n) + P[A_{n-1}]P[B_{(n-1,n)}]+ 
$$
$$
+ P[A_{n-2}](P[B_{(n-2,n)}] 
+ P[B_{(n-2,n-1)}]P[B_{(n-1,n)}]  ) + \dots 
$$

Можем заметить, что:
\renewcommand{\labelitemi}{$\checkmark$}
\begin{itemize}
\item $P[A_i] = P[A_j] = \dfrac{1}{n}$ $\forall \hspace{0.1cm} i, j$
\item $P[B_{(n-1,n)}]=\dfrac{1}{2}$, потому что $n-1$-ый выбирает из двух оставшихся мест
\item $P[B_{(n-2,n)}] 
+ P[B_{(n-2,n-1)}]P[B_{(n-1,n)}]  = \dfrac{1}{3} + \dfrac{1}{3}\cdot \dfrac{1}{2} = \dfrac{1}{2}$
\item $P[B_{(n-3,n)}] + P[B_{(n-3,n-2)}](P[B_{(n-2,n)}]+P[B_{(n-2,n-1)}]P[B_{(n-1,n)}])+\vspace{0.5cm} \\
\hspace*{3cm} + P[B_{(n-3,n-1)}]P[B_{(n-1,n)}] =   \dfrac{1}{4}+\dfrac{1}{4}\left(\dfrac{1}{3}+\dfrac{1}{3}\cdot\dfrac{1}{2}\right) +\dfrac{1}{4}\cdot \dfrac{1}{2} = \dfrac{1}{2}$
\item И так далее до того момента, пока старушка не сядет на место первого человека, который заходит после нее, --- всего $n-2$ вариантов.
\end{itemize}

Таким образом мы получаем сумму:
$$
P[n\text{-ый не сядет на свое место}] = \dfrac{1}{n} + \dfrac{1}{n}\cdot \dfrac{1}{2} + \dfrac{1}{n}\cdot \dfrac{1}{2} + \dots = \dfrac{1}{n}  + \frac{1}{2n}(n-2) = \dfrac{1}{2}
$$
\begin{center}
\fcolorbox{ForestGreen}{white}{
Значит вероятность $P[n\text{-ый сядет на свое место}] = 1 -\dfrac{1}{2} = \dfrac{1}{2}$}
\end{center}

А вот красивое более \textbf{простое решение от Бориса Борисовича:}\\
\underline{Метод математической индукции:} допустим что это утверждение доказано для одного, двух и так далее до $k$ человек. Рассмотрим $k+1$ человека. Когда последний сядет на своё место? Если старушка сядет на своё место, а вероятность этого равна $\dfrac{1}{k+1}$ или, с вероятностью $\dfrac{1}{2}$ (по индукции), если старшука сядет на любое место кроме своего и последнего, то есть $\dfrac{1}{2}\cdot\dfrac{k-1}{k+1}$. В этом случае тот\vspace{0.2cm} пассажир, чье место  она заняла, становится старушкой, и мы получаем задачу при меньшем $k$. Складывая эти две дроби, получаем $\dfrac{1}{2} $.

Чтобы найти среднее число пассажиров, разобъем эту величину в сумму индикаторов: $Y_1$ --- сел ли первый на место, $\dots$, $Y_n$ --- сел ли $n$-ый на место (индикатор равен единице, если сел). 

Стало быть $E(Y)=E(Y_1)+E(Y_2)+...+E(Y_n)$. $E(Y_n)=\frac{1}{2}$.

Почти аналогично можем рассуждать для предпоследнего:\\
База индукции: если пассажиров трое ($n=3$ включая старушку), то для предпоследнего вероятность сесть на своё место равна $\frac{2}{3}$.\\
Шаг индукции: допустим что для $3, 4, ... n$ пассажиров эта вероятность равна $\frac{2}{3}$.
Рассмотрим случай $(n+1)$-го пассажира. 
Предпоследний сядет на своё место, если:

\renewcommand{\labelitemi}{\textbullet}

\begin{itemize}
\item старушка сядет на своё место или на место последнего $\frac{2}{n+1}$
\item в $\frac{2}{3}$ тех случаев, когда старушка сядет на место $2, 3, ..., (n-1)$, т.е. $\frac{2}{3}\cdot \frac{n-2}{n+1}$
складываем, получаем $\frac{2}{3}$.
То есть по индукции вероятность того, что предпоследний сядет на своё место равна $\frac{2}{3}$
\end{itemize}
И по аналогии можно увидеть, что вероятность того, что $k$-ый с конца пассажир сядет на своё место равна $k/(k+1)$

Если у нас $n$ пассажиров включая СС, то среднее количество севших на свои места (раскладывая с конца) равно $$\frac{1}{2}+\frac{2}{3}+\frac{3}{4}+\dots+\frac{n-1}{n}+\frac{1}{n}$$

\end{enumerate}

\begin{center}
\section*{Разбалловка}
\subsection*{Часть 1}
\end{center}

\textbf{Задание 1}
\begin{itemize}
\item 2 балла
\item 4 балла
\item 4 балла
\end{itemize}

\textbf{Задание 2}
\begin{itemize}
\item мат. ожидание —- 3 балла, дисперсия —- 3 балла
\item распределение —- 3 балла, знак неравенства —- 1 балл

\end{itemize}

\textbf{Задание 3}
\begin{itemize}
\item 2 балла
\item все по 1 баллу
\item каждая вероятность —- 1.5 балла
\item 2 балла
\textbf{Задание 4}
\begin{itemize}
\item все по 1 баллу
\item первая ковариация —- 3 балла, вторая —- 2 балла
\item 2 балла
\end{itemize}

\end{itemize}

\textbf{Задание 5}
\begin{itemize}
\item 10 или 0
\end{itemize}

\textbf{Задание 6}
\begin{enumerate}
\item явный вид 2, приближение Пуассона 2
\item наиболее вероятное число 2, математическое ожидание 1, дисперсия 1
\item 2
\end{enumerate}

\textbf{Задание 7}
\begin{enumerate}
\item математические ожидание 2.5, дисперсии 2.5
\item для бесполезных и неинтересных 2.5, с хотя бы одним из свойств 2.5

\end{enumerate}

\textbf{Задание 8}
\begin{itemize}
\item константы $a$ и $b$ 5
\item математическое ожидание 2
\item третий начальный момент, медиана и мода —- по 1
\end{itemize}

\begin{center}
\subsection*{Часть 2}
\end{center}

\textbf{Задание 1}
\begin{itemize}
\item каждая вероятность —- 1 балл
\item каждое мат. ожидание —- 2 балла
\item 1 балл
\end{itemize}

\textbf{Задание 2}
\begin{itemize}
\item каждая вероятность —- 1.5 балла
\item первое мат. ожидание —- 1.5 балла, последние два мат. ожидания —- по 2 балла

\end{itemize}

\textbf{Задание 3}
\begin{itemize}
\item 6 баллов
\item 4 балла
\end{itemize}
\textbf{Задание 4}
\begin{itemize}
\item $\mathbb{P}(X=50)$ —- 2
\item $\mathbb{E}[X]$ —- 4
\item $\mathbf{Var}(X)$ —- 4
\end{itemize}

\textbf{Задание 5}
\begin{enumerate}
\item 2
\item 4
\item 4
\end{enumerate}
\end{document}


