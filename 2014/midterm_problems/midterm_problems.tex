\documentclass[12pt,a4paper]{article}
\usepackage[utf8]{inputenc}
\usepackage[russian]{babel}

\newcommand{\E}{\mathbb{E}}
\renewcommand{\P}{\mathbb{P}}
\newcommand{\Var}{\mathrm{Var}}
\newcommand{\Cov}{\mathrm{Cov}}
\newcommand{\Corr}{\mathrm{Corr}}

\usepackage{amsmath}
\usepackage{amsfonts}
\usepackage{amssymb}
\usepackage[left=2cm,right=2cm,top=2cm,bottom=2cm]{geometry}
\begin{document}

Несколько задач для устного экзамена

\begin{enumerate}
\item За неделю Аннушка семь раз пролила масло. 
\begin{enumerate}
\item Какова вероятность того, что она проливала масло каждый день?
\item Какова вероятность того, что Аннушка пролила масло в среду, если в четверг Аннушка масла не проливала?
\end{enumerate}

\item Погода завтра может быть ясной с вероятностью $0.3$ и пасмурной с вероятностью $0.7$. Вне зависимости от того, какая будет погода, Маша даёт верный прогноз с вероятностью $0.8$. Вовочка, не разбираясь в погоде, делает свой прогноз по принципу: с вероятностью $0.9$ копирует Машин прогноз, и с вероятностью $0.1$ меняет его на противоположный. 
\begin{enumerate}
\item Какова вероятность того, что Маша спрогнозирует ясный день?
\item Какова вероятность того, что Машин и Вовочкин прогнозы совпадут?
\item Какова вероятность того, что день будет ясный, если Маша спрогнозировала ясный?
\item Какова вероятность того, что день будет ясный, если Вовочка спрогнозировал ясный?
\end{enumerate} 

\item Два охотника выстрелили в одну утку. Первый попадает с вероятностью 0,4, второй --- с вероятностью 0,7.  
\begin{enumerate}
\item Какова вероятность того, что оба охотника попали в утку?
\item Какова вероятность того, что утка была убита первым охотником, если в утку попал только один из охотников?
\end{enumerate}




\item Функция плотности случайной величины $X$ имеет вид $f(x)=\left\{\begin{array}{l}
\frac{3}{7}x^2,\, x\in[1;2] \\
0,\, x\notin [1,2] 
\end{array}\right.$
\begin{enumerate}
\item Не производя вычислений найдите $\int_{-\infty}^{+\infty}f(x)\,dx$
\item Найдите $\E(X)$, $\E(X^2)$ и дисперсию $\Var(X)$
\item Найдите $\P(X>1.5)$
\item Найдите функцию распределения $F(x)$ и постройте её график
\end{enumerate}


\item Совместное распределение случайных величин $X$ и $Y$ задано таблицей

\begin{tabular}{c|ccc}
 & $X=-2$ & $X=0$ & $X=2$ \\ 
\hline 
$Y=1$ & 0.2 & 0.3 & 0.1 \\ 
$Y=2$ & 0.1 & 0.2 & $a$ \\ 
\end{tabular} 

\begin{enumerate}
\item Определите неизвестную вероятность $a$. 
\item Найдите вероятности $\P(X>-1)$, $\P(X>Y)$, $\P(Y=1 | X+1>0)$
\item Найдите математические ожидания $\E(X)$, $\E(X^2)$
\item Найдите корреляцию $\Corr(X,Y)$
\end{enumerate}

\item В треугольнике с вершинами $(0,0)$, $(0,1)$ и $(1,0)$ равновероятно выбирается точка. Величины $X$ и $Y$ --- абсцисса и ордината этой случайной точки. 
\begin{enumerate}
\item Найдите совместную функцию плотности пары $(X,Y)$
\item Найдите $\P(2X>1)$, $\P(Y>X)$, $\P(Y>X | 2X>1)$
\item Найдите частную функцию плотности величины $X$
\item Найдите $\E(X)$, $\E(Y)$, $\Var(X)$, $\Var(Y)$, $\Cov(X,Y)$
\item Являются ли величины $X$ и $Y$ одинаково распределенными и независимыми?
\end{enumerate}

\item Совместная функция плотности величин $X$ и $Y$ имеет вид
\begin{equation}
f(x,y)=\begin{cases}
2(x^3+y^3), \mbox{ если } x\in [0;1], y\in [0;1] \\
0, \mbox{ иначе}
\end{cases} 
\end{equation}
\begin{enumerate}
\item  Найдите $\P(X+Y>1)$, $\P(X+Y>1 | X>Y)$
\item Найдите $\E(X)$, $\E(Y)$, $\Var(X)$, $\Var(Y)$, $\Cov(X,Y)$
\item  Являются ли величины $X$ и $Y$ независимыми? 
\item Являются ли величины $X$ и $Y$ одинаково распределенными?
\end{enumerate}



\item Для случайных величин $X$ и $Y$ заданы следующие значения: $\E(X)=1$, $\E(Y)=4$, $\E(XY)=8$, $\Var(X)=\Var(Y)=9$. Для случайных величин $U=X+Y$ и $V=X-Y$ вычислите: 
\begin{enumerate}
\item $\E(U)$, $\Var(U)$, $\E(V)$, $\Var(V)$, $\Cov(U,V)$ 
\item Можно ли утверждать, что случайные величины U и V независимы? 
\end{enumerate}

\item С помощью неравенства Чебышева, укажите границы, в которых
находятся величины; рассчитайте также их точное значение 
\begin{enumerate}
\item  $\P(-2\sigma<X-\mu<2\sigma)$, $X\sim N(\mu;\sigma^{2})$ 
\item  $\P(8<X<12)$, $X\sim U[0;20]$ 
\item $\P(-2<X-\E(X)<2)$, $X$ имеет экспоненциальное распределение с
$\lambda=1$
\end{enumerate}


\item Сейчас акция стоит 1000 рублей. Каждый день цена может равновероятно либо возрасти на 3 рубля, либо упасть на 5 рублей. 
\begin{enumerate}
\item Чему равно ожидаемое значение цены через 60 дней? Дисперсия?
\item Какова вероятность того, что через 60 дней цена будет больше 900 рублей? 
\end{enumerate}


\item Предположим, что истинная вероятность рождения мальчика равна $0.5$. Каким должен быть размер выборки, чтобы с вероятностью $0.95$ можно было утверждать, что выборочная доля отличается от истинной вероятности не более, чем на $0.02$?

\item Допустим, что срок службы пылесоса имеет экспоненциальное распределение. В среднем один пылесос бесперебойно работает 7 лет. Завод предоставляет гарантию 5 лет на свои изделия. Предположим также, что примерно 80\% потребителей аккуратно хранят все бумаги, необходимые, чтобы воспользоваться гарантией. 
\begin{enumerate}
\item Какой процент потребителей в среднем обращается за гарантийным ремонтом? 
\item Какова вероятность того, что из 1000 потребителей за гарантийным ремонтом обратится более 35\% покупателей? 
\end{enumerate}
Подсказка: $\exp(5/7)\approx 2$ 

\end{enumerate}


\end{document}