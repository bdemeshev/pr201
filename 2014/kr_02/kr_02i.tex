\documentclass[12pt,a4paper]{article}
\usepackage[utf8]{inputenc}
\usepackage[russian]{babel}

\usepackage{amsmath}
\usepackage{amsfonts}
\usepackage{amssymb}
\usepackage[left=2cm,right=2cm,top=2cm,bottom=2cm]{geometry}

\DeclareMathOperator{\Var}{Var}
\DeclareMathOperator{\Cov}{Cov}
\newcommand{\E}{\mathbb{E}}
\renewcommand{\P}{\mathbb{P}}

\begin{document}
Праздник номер 2 по теории вероятностей! 15.12.2014

\begin{enumerate}
\item Вася может получить за экзамен равновероятно либо $8$ баллов, либо $7$ баллов. Петя может получить за экзамен либо $8$ баллов --- с вероятностью $1/3$; либо $7$ баллов --- с вероятностью $2/3$. Известно, что корреляция их результатов равна $0.7$. 

Какова вероятность того, что Петя и Вася покажут одинаковый результат? 

\item В  городе Туме проводят демографическое исследование семейных пар. Стандартное отклонение возраста мужа оказалось равным 5 годам, а стандартное отклонение возраста жены --- 4 годам. Найдите корреляцию возраста жены и возраста мужа, если стандартное отклонение разности возрастов оказалось равным 2 годам. В каких пределах лежит вероятность того, что возраст случайно выбираемого женатого мужчины отклоняется от своего математического ожидания больше чем на 10 лет?


\item На окружности единичной длины случайным образом равномерно и независимо друг от друга выбирают две дуги: длины $0.3$ и длины $0.4$. 
\begin{enumerate}
\item  Найдите функцию распределения длины пересечения этих отрезков
\item Найдите среднюю длину пересечения
\end{enumerate}


\item  Совместная функция плотности величин $X$ и $Y$ имеет вид
\begin{equation}
\nonumber
f(x,y)=\begin{cases}
2(x^3+y^3), \text{ если } x\in [0;1], y\in [0;1] \\
0, \mbox{ иначе}
\end{cases} 
\end{equation}
\begin{enumerate}
\item $[1]$ Найдите $\P(X+Y>1)$
\item $[6]$ Найдите $\Cov(X,Y)$
\item $[1]$ Являются ли величины $X$ и $Y$ независимыми? 
\item $[1]$ Являются ли величины $X$ и $Y$ одинаково распределенными?
\end{enumerate}



\item Изначально цена акций компании <<Пумперникель>> равна $X_0=1000$ рублей. Каждый последующий день в течение 100 дней цена равновероятно может вырасти на 2 рубля или упасть на 1 рубль. Обозначим цену акции через $n$ дней как $X_n$.
\begin{enumerate}
\item Чему равны математическое ожидание и дисперсия изменения цены за отдельный день?
\item Найдите $\E(X_n)$, $\Var(X_n)$, $\Cov(X_n, X_k)$
\item Сформулируйте центральную предельную теорему
\item Примерно найдите вероятность $\P(X_{100}>1060)$
\item Биржевой игрок Вениамин утверждает, что через 100 дней с вероятностью 95\% цена акций <<Пумперникель>> не опустится ниже $a$. Чему равно $a$?
\end{enumerate}
\item  Сэр Фрэнсис Гальтон ---  учёный XIX-XX веков, один из основоположников как
генетики, так и статистики --- изучал, среди всего прочего, связь между ростом детей и родителей.  Он исследовал данные о росте 928 индивидов. Обозначим $X_1$ --- рост случайного человека, а $X_2$ --- среднее арифметическое роста его отца и матери. По результатам исследования Гальтона:


\[
\begin{pmatrix}
X_1 \\
X_2
\end{pmatrix}
\sim
N\left[
\begin{pmatrix}
68.1 \\
68.3
\end{pmatrix};
\begin{pmatrix}
6.3 & 2.1 \\
2.1 & 3.2
\end{pmatrix}
\right]
\]

\begin{enumerate}
\item Обратите внимание на то, что дисперсия роста детей выше дисперсии среднего роста
родителей. С чем это
может быть связано? Учтите, что рост детей измерялся уже по достижении
зрелости, так что разброс не должен быть связан с возрастными различиями.
\item Рассчитайте корреляцию между $X_1$ и $X_2$
\item Один дюйм примерно равен $2.54$ сантиметра. Пусть $X_1'$ и $X_2'$ --- это те же $X_1$ и $X_2$, только измеренные в сантиметрах. Найдите вектор математических ожиданий и ковариационную матрицу вектора $X'=(X_1', X_2')$.
\item Определите, каков ожидаемый рост и дисперсия роста человека, средний рост родителей которого составляет 72 дюйма?
\item Найдите вероятность того, что рост человека превысит 68 дюймов, если средний рост его родителей равен 72 дюймам. Подсказка: используйте предыдущий пункт и нормальность распределения!
%\item Пуассоновское казино. В пуассоновском казино сигнальная лампочка загорается согласно пуассоновскому потоку в среднем один раз в минуту. В любой момент азартный игрок Вениамин может нажать кнопку. Нажать кнопку Вениамин может только один раз. Вениамин срывает джек-пот, 
\end{enumerate}

\item Звонки поступают в пожарную часть согласно пуассоновскому потоку в среднем $2$~раза в час. Предположим, что после получения звонка пожарная часть занята тушением пожара случайное время равномерно распределённое от получаса до часа. В это время звонки перенаправляются в соседнюю пожарную часть. 

Пожарная часть только-только начала работать и готова принимать звонки. 
\begin{enumerate}
\item Какова вероятность того, что за ближайший час не поступит звонков?
\item Какова вероятность того, что за ближайший час не будет перенаправленных звонков?
\item Найдите закон распределения количества звонков до первого перенаправленного звонка.
\end{enumerate}

\item  Судьба Дон-Жуана


У Дон-Жуана $n$  знакомых девушек, и их всех зовут по-разному. Он пишет
им $n$  писем, но по рассеянности раскладывает их в конверты
наугад. Случайная величина $X$ обозначает количество девушек, получивших письма, адресованные лично им.

\begin{enumerate}
\item Найдите $\E(X)$, $\Var(X)$
\item Какова при большом $n$ вероятность того, что хотя бы одна девушка получит письмо, адресованное ей?
\end{enumerate}

\item Игла Бюффона

Плоскость разлинована параллельными линиями через каждый сантиметр. Случайным образом на эту плоскость бросается иголка длины $a<1$. 

\begin{enumerate}
\item Какова вероятность того, что иголка пересечёт какую-нибудь линию?
\item Предложите вероятностный способ оценки числа $\pi$
\end{enumerate}


\end{enumerate}

\end{document}